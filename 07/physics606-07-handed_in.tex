\documentclass[11pt, english, fleqn, DIV=15, headinclude, BCOR=1.5cm]{scrartcl}

\usepackage[
    bibatend,
    color,
]{../header}

\usepackage{tikz}
\usepackage{pdflscape}

\usepackage[tikz]{mdframed}
\newmdtheoremenv[%
    backgroundcolor=black!5,
    innertopmargin=\topskip,
    splittopskip=\topskip,
]{theorem}{Theorem}[section]

\hypersetup{
    pdftitle=
}

\newcounter{totalpoints}
\newcommand\punkte[1]{#1\addtocounter{totalpoints}{#1}}

\newcounter{problemset}
\setcounter{problemset}{7}

\subject{physics606 -- Advanced Quantum Theory}
\ihead{physics606 -- Problem Set \arabic{problemset}}

\title{Problem Set \arabic{problemset}}

\publishers{Group 2 -- Dilege Gülmez}
\ofoot{Group 2 -- Dilege Gülmez}

\newmdenv[%
    backgroundcolor=black!0,
    frametitlebackgroundcolor=black!0,
    roundcorner=5pt,
    skipabove=\topskip,
    innertopmargin=\topskip,
    splittopskip=\topskip,
    frametitle={Problem statement},
    frametitlerule=true,
    nobreak=true,
]{problem}

\newmdenv[%
    backgroundcolor=white,
    frametitlebackgroundcolor=black!0,
    roundcorner=5pt,
    skipabove=\topskip,
    innertopmargin=\topskip,
    innerbottommargin=8cm,
    splittopskip=\topskip,
    frametitle={Side question},
    frametitlerule=true,
]{question}

\newcommand\an{^\text{angular}}
\newcommand\ra{^\text{radial}}


\author{
    Martin Ueding \\ \small{\href{mailto:mu@martin-ueding.de}{mu@martin-ueding.de}}
    \and
    Lino Lemmer \\ \small{\href{mailto:l2@uni-bonn.de}{l2@uni-bonn.de}}
}
\ifoot{Martin Ueding, Lino Lemmer}

\ohead{\rightmark}

\begin{document}

\maketitle

\vspace{3ex}

\begin{center}
    \begin{tabular}{rrr}
        problem number & achieved points & possible points \\
        \midrule
        1 & & \punkte{25} \\
        \midrule
        total & & \arabic{totalpoints}
    \end{tabular}
\end{center}

\section{Rate for (nP) to (n'S) transitions}

\subsection{Orthogonality}

\newcommand\tg{\theta_\gammaup}
\newcommand\phg{\phi_\gammaup}
\newcommand\te{\theta_\text e}
\newcommand\phe{\phi_\text e}
\newcommand\re{r_\text e}

Given is the wave vector
\[
    \vec k_\gammaup = k_\gammaup
    \begin{pmatrix}
        \sin(\tg) \sin(\phg) \\
        \sin(\tg) \cos(\phg) \\
        \cos(\tg)
    \end{pmatrix}
\]
where we have used the notation where a vector in non-bold type is its absolute
value, consistent with its look as a scalar. We hope this becomes apparent on
this ink jet printout. Then there are also two polarization basis vectors
given:
\[
    \vec \epsilon_1 =
    \begin{pmatrix}
        \cos(\phg) \\
        - \sin(\phg) \\
        0
    \end{pmatrix}
    \eqnsep
    \vec \epsilon_2 =
    \begin{pmatrix}
        \cos(\tg) \sin(\phg) \\
        \cos(\tg) \cos(\phg) \\
        - \cos(\tg)
    \end{pmatrix}.
\]

Checking for orthonormality among the $\vec \epsilon_i$ is $X - X = 0$. We done
that, it checks out. It is also easy to see that $\vec \epsilon_1 \vec
k_\gammaup = 0$. The last one requires application of the so called
trigonometric identities. We present all the steps here in painstaking detail:
\begin{align*}
    \vec \epsilon_2 \vec k_\gamma
    &= \cos(\tg) \sin(\tg) \sin(\phg)^2 + \sin(\tg) \cos(\tg) \cos(\phg)^2 -
    \sin(\tg) \cos(\tg) \\
    \intertext{%
        The first two terms contain common factors. We factor those out to get
        a more compact expression.
    }
    &= \cos(\tg) \sin(\tg) \sbr{ \sin(\phg)^2 + \cos(\phg)^2} - \sin(\tg) \cos(\tg)
    \\
    \intertext{%
        Here we use $\sin(\phg)^2 + \cos(\phg)^2 = 1$. This leaves us with
    }
    &= \cos(\tg) \sin(\tg) - \sin(\tg) \cos(\tg). \\
    \intertext{%
        After using commutation law of the product we have to equal terms. They
        cancel each other and just gives us
    }
    &= 0.
\end{align*}

The more interesting question, which was not asked in this problem, is “Why do
the polarization vectors have to be perpendicular to the wave vector?”. This is
because electromagnetic radiation is vector-polarized and a transverse wave.
Therefore the polarization is perpendicular to the direction of propagation.
This also means that the photon has spin 1. Gravitational waves are
tensor-polarized and the graviton would have spin 2.

\subsection{Angular part of integrands}

We begin by copying the terms.
\begin{align*}
    \vec x_\text e \vec \epsilon_1
    &= r_\text e
    \begin{pmatrix}
        \sin(\te) \sin(\phe) \\
        \sin(\te) \cos(\phe) \\
        \cos(\te)
    \end{pmatrix}
    \begin{pmatrix}
        \cos(\phg) \\
        - \sin(\phg) \\
        0
    \end{pmatrix} \\
    &= r_\text e \sbr{
        \sin(\te) \sin(\phe) \cos(\phg) - \sin(\te) \sin(\phe) \sin(\phg)
    } \\
    &= r_\text e \sin(\te) \sbr{
        \sin(\phe) \cos(\phg) - \sin(\phe) \sin(\phg)
    } \\
    \intertext{%
        Now we apply the addition theorem.
    }
    &= r_\text e \sin(\te) \sin(\phe - \phg)
\end{align*}

The second one is very similar.
\begin{align*}
    \vec x_\text e \vec \epsilon_2
    &= r_\text e
    \begin{pmatrix}
        \sin(\te) \sin(\phe) \\
        \sin(\te) \cos(\phe) \\
        \cos(\te)
    \end{pmatrix}
    \begin{pmatrix}
        \cos(\tg) \sin(\phg) \\
        \cos(\tg) \cos(\phg) \\
        - \cos(\tg)
    \end{pmatrix} \\
    &= r_\text e \sbr{
        \sin(\te) \sin(\phe) \cos(\tg) \sin(\phg)
        + \sin(\te) \cos(\phe) \cos(\tg) \cos(\phg)
        - \cos(\te) \sin(\tg)
    }
    \intertext{%
        We also apply the addition theorem here.
    }
    &= r_\text e \sbr{
        \sin(\te) \cos(\tg) \cos(\phe - \phg)
        - \cos(\te) \sin(\tg)
    }
\end{align*}

\subsection{Evaluation of the integrals}

Now we got the middle of the braket in terms of the angles. Next are initial
and final states. Since there are a lot of combinations of $m_i$ and $\vec
\epsilon$, we will do this in sections.

The radial part, $r_\text e$ will be set to $1$ here. Since part five of this
problem takes care of the radial part, we will only look at the angular part
here. Luckily, radial and angular part are independent, so that the actual
transition rates are the product of both:
\[
    \Gamma_{fi} = \Gamma_{fi}\ra \Gamma_{fi}\an
    \eqnsep
    \mathcal M_{fi} = \mathcal M_{fi}\ra \mathcal M_{fi}\an.
\]
In this part, we will take care of
all the factors from the equation, such that in part five we only have to carry
the factors that come from the Laguerre polynomials later on.

\subsubsection{Case $\vec \epsilon = \vec \epsilon_1$}

\paragraph{Case $m_i = 0$}

We start by calculating the matrix element
\begin{align*}
    \mathcal M_{fi}\an
    &= - \frac{\omega_{if}}c \braket{f|\vec \epsilon_1 \vec x_e | i}. \\
    \intertext{%
        In the first step, we insert the knowns. $\bra f$ just gives a factor.
        The scalar product is inserted, $r_\text e$ is omitted since this
        belongs to the radial part. $\ket i$ gives us a factor, the exponential
        function is 1 since $m_i = 0$ is assumed. In $f$, we get another
        $\cos(\te)$ and a factor.
    }
    &= - \frac{\sqrt 3}{4 \piup} \frac{\omega_{if}}c 
    \intop_{\Omega_\text e} \dif \Omega_\text e \, \sin(\te) \sin(\phe - \phg) \cos(\te)
    \intertext{%
        The volume element is $\dif \Omega = \dif \theta \, \dif \phi \,
        \sin(\theta)$.
    }
    &= - \frac{\sqrt 3}{4 \piup} \frac{\omega_{if}}c 
    \int_0^\piup \dif \te \, \int_0^{2\piup} \dif \phe \, \sin(\te)^2 \sin(\phe - \phg) \cos(\te)
    \intertext{%
        We move the second integral as far back as possible.
    }
    &= - \frac{\sqrt 3}{4 \piup} \frac{\omega_{if}}c 
    \int_0^\piup \dif \te \, \sin(\te)^2 \cos(\te) \int_0^{2\piup} \dif \phe \, \sin(\phe - \phg)
    \intertext{%
        The last integral will give zero.
    }
    &= 0
\end{align*}
From this, $\Gamma_{if}\ra = 0$ as well.

\paragraph{Case $|m_i| = 1$}

For this case, we start similarly.
\begin{align*}
    \mathcal M_{fi}\an
    &= - m_i \frac{\omega_{if}}c \braket{f|\vec \epsilon_1 \vec x_e | i}. \\
    \intertext{%
        $\ket i$ gives us the exponential function. In $f$, we get another $m_i
        \sin(\te)$. There are three $\sin(\te)$, one from the integration
        measure, one from $f$ and another from $\vec \epsilon_1 \vec x_\text
        e$.
    }
    &= m_i \frac{\sqrt 3}{4 \piup \sqrt 2} \frac{\omega_{if}}c 
    \int_0^\piup \dif \te \, \int_0^{2\piup} \dif \phe \, \sin(\te)^3 \sin(\phe
    - \phg) \exp(-\iup m_i \phe)
    \intertext{%
        We do the same reordering.
    }
    &= m_i \frac{\sqrt 3}{4 \piup \sqrt 2} \frac{\omega_{if}}c 
    \int_0^\piup \dif \te \, \sin(\te)^3 \int_0^{2\piup} \dif \phe \, \sin(\phe
    - \phg) \exp(-\iup m_i \phe) 
    \intertext{%
        The first and second integral are completely independent now. We just
        insert the result for the first one, $4/3$, to get a bit more space on
        the line without loosing the chain of equalities.
    }
    &= m_i \frac{1}{\piup \sqrt 6} \frac{\omega_{if}}c 
    \int_0^{2\piup} \dif \phe \, \sin(\phe
    - \phg) \exp(-\iup m_i \phe) 
    \intertext{%
        Now we expand the sine in terms of the exponential functions.
    }
    &= m_i \frac{1}{2 \iup \piup \sqrt 6} \frac{\omega_{if}}c 
    \int_0^{2\piup} \dif \phe \,
    \sbr{
        \exp\del{\iup [\phe - \phg]} - \exp\del{- \iup [\phe - \phg]}
    } \exp(-\iup m_i \phe) 
    \intertext{%
        The last exponential can be moved into the first two.
    }
    &= m_i \frac{1}{2 \iup \piup \sqrt 6} \frac{\omega_{if}}c 
    \int_0^{2\piup} \dif \phe \,
    \sbr{
        \exp\del{\iup \sbr{[1-m_i]\phe - \phg}} - \exp\del{- \iup \sbr{[1+m_i]\phe - \phg}}
    }
\end{align*}

From here, we have to discuss two cases. In either case, only one of the most
inner square brackets will be nonzero.

\begin{description}
    \item[Sub case $m_i = 1$] 

        \begin{align*}
            \mathcal M_{fi}\an
            &= \frac{1}{2 \iup \piup \sqrt 6} \frac{\omega_{if}}c 
            \int_0^{2\piup} \dif \phe \,
            \sbr{
                \exp\del{- \iup \phg} - \exp\del{- \iup \sbr{2\phe - \phg}}
            } \\
            \intertext{%
                We use the linearity in the integration.
            }
            &= \frac{1}{2 \iup \piup \sqrt 6} \frac{\omega_{if}}c 
            \sbr{
                \int_0^{2\piup} \dif \phe \,
                \exp\del{- \iup \phg}
                -
                \int_0^{2\piup} \dif \phe \,
                \exp\del{- \iup \sbr{2\phe - \phg}}
            } \\
            \intertext{%
                The first integrand does not depend on the integration
                variable, the integral is therefore trivial. In the second
                integral, we pull out the term that does not depend on the
                integration variable.
            }
            &= \frac{1}{2 \iup \piup \sqrt 6} \frac{\omega_{if}}c 
            \sbr{
                2 \piup \exp\del{- \iup \phg}
                -
                \exp(\iup \phg)
                \int_0^{2\piup} \dif \phe \,
                \exp\del{- 2 \iup \phe}
            } \\
            \intertext{%
                The last integral will give zero, since the exponential will
                give 1 at 0 and $2\piup$. Only the first term will contribute.
            }
            &= \frac{1}{\iup \sqrt 6} \frac{\omega_{if}}c \exp\del{- \iup \phg}
        \end{align*}

    \item[Sub case $m_i = -1$]

        Here, the terms are the other way around.
        \begin{align*}
            \mathcal M_{fi}\an
            &= - \frac{1}{2 \iup \piup \sqrt 6} \frac{\omega_{if}}c 
            \int_0^{2\piup} \dif \phe \,
            \sbr{
                \exp\del{\iup [2\phe - \phg]} - \exp(\iup \phg)
            }
            \intertext{%
                We use the same linearity.
            }
            &= - \frac{1}{2 \iup \piup \sqrt 6} \frac{\omega_{if}}c 
            \sbr{
                \int_0^{2\piup} \dif \phe \,
                \exp\del{\iup [2\phe - \phg]}
                -
                \int_0^{2\piup} \dif \phe \,
                \exp(\iup \phg)
            } \\
            \intertext{%
                We do the same steps, the first integral will yield zero. The
                second gives an additional factor of $2 \piup$.
            }
            &= \frac{1}{\iup \sqrt 6} \frac{\omega_{if}}c \exp(\iup \phg)
        \end{align*}
\end{description}

We can now see that the two results we have are almost complex conjugates.
Their absolute value will be the same. We continue to compute the transition
ratio.
\begin{align*}
    \Gamma_{fi}\an
    &= \frac{\alpha}{2\piup} \omega_{if} \int \dif \Omega_\gammaup \,
    \abs{\mathcal M_{fi}\an}^2. \\
    \intertext{%
        The modulus squared of the matrix element does not have any angular
        dependence now. The angular integration will yield a factor of $4
        \piup$.
    }
    &= \frac{\alpha}{3c^2} \omega_{if}^3
\end{align*}

\subsubsection{Case $\vec \epsilon = \vec \epsilon_2$}

\paragraph{Case $m_i = 0$}

We start by calculating the matrix element
\begin{align*}
    \mathcal M_{fi}\an
    &= - \frac{\omega_{if}}c \braket{f|\vec \epsilon_2 \vec x_e | i}. \\
    \intertext{%
        In the first step, we insert the knowns. $\bra f$ just gives a factor.
        The scalar product is inserted. $\ket i$ gives us a factor, the
        exponential function is 1 since $m_i = 0$ is assumed. In $f$, we get
        another $\cos(\te)$ and a factor.
    }
    &= - \frac{\sqrt 3}{4 \piup} \frac{\omega_{if}}c 
    \int \dif \Omega_\text e \,
    \sbr{
        \sin(\te) \cos(\tg) \cos(\phe - \phg)
        - \cos(\te) \sin(\tg)
    } \cos(\te)
    \intertext{%
        The volume element is $\dif \Omega = \dif \theta \, \dif \phi \,
        \sin(\theta)$. We insert this.
    }
    &= - \frac{\sqrt 3}{4 \piup} \frac{\omega_{if}}c 
    \int_0^\pi \dif \te \, \sin(\te) \\&\qquad \cdot \int_0^{2\piup} \dif \phe \,
    \sbr{
        \sin(\te) \cos(\tg) \cos(\phe - \phg)
        - \cos(\te) \sin(\tg)
    } \cos(\te)
    \intertext{%
        The integral over $\phe$ will give zero for the first summand in the
        square bracket. For the second, it will just contribute a factor of $2
        \piup$ since there is no dependence on $\phe$.
    }
    &= \frac{\sqrt 3}{2} \frac{\omega_{if}}c 
    \int_0^\pi \dif \te \, \sin(\te) \cos(\te)^2
    \intertext{%
        This integral can be solved using the substitution $z := \cos(\te)$
        and gives the value $2/3$. Our final result is
    }
    &= \frac{1}{\sqrt 3} \frac{\omega_{if}}c 
\end{align*}

From here, we compute the rate
\begin{align*}
    \Gamma_{fi}\ra
    &= \frac{\alpha}{2\piup} \omega_{if} \int \dif \Omega_\gammaup \,
    \abs{\mathcal M_{fi}\an}^2. \\
    \intertext{%
        We insert our previous result
    }
    &= \frac{\alpha}{2\piup} \omega_{if} \int \dif \Omega_\gammaup \,
    \abs{\frac{1}{\sqrt 3} \frac{\omega_{if}}c \sin(\tg)}^2 \\
    \intertext{%
        and simplify
    }
    &= \frac{\alpha}{6\piup} \frac{\omega_{if}^3}{c} 
    \int \dif \Omega_\gammaup \, \sin(\tg)^2. \\
    \intertext{%
        The integration has to be rewritten in terms of the chosen coordinates,
        the spherical polar coordinates. We already carry out the integration
        in $\phg$ which only gives a factor $2\piup$.
    }
    &= \frac{\alpha}{3} \frac{\omega_{if}^3}{c} 
    \int \dif \tg \, \sin(\tg)^3. \\
    \intertext{%
        This integral can be solved with the power reduction formulas. Those
        can be derived using the binomial theorem and the exponential
        representation of the sine. All in all, the integral is $4/3$. That
        gives the final result for this choice of $\vec\epsilon$ and $m_i$.
    }
    &= \frac{4\alpha}{9c} \omega_{if}^3 
\end{align*}

\paragraph{Case $|m_i| = 1$}

% TODO This part is missing.

\subsection{Summation of contributions}

\begin{problem}
    In order to compute the total decay rate, the contributions from both
    polarization states of the photon have to be added incoherently (why?).
\end{problem}

The decays of different polarizations are independent of each other. The
photons themselves might interfere if they are emitted at the same time from
neighboring atoms, but that does not change the rates.

\begin{problem}
    Show that after this summation, which simply means adding the two
    contributions evaluated in the previous step, the decay rate is independent
    of $m_i$.
\end{problem}

\begin{problem}
    Give a physical reason for this result.
\end{problem}

When the atom is in a state that has a certain orientation as given my the $m$,
which is $L_3 / \hbar$, it has certain emission characteristics. It cannot
emit light in any direction. However, when all values of $L_3$---and therefore
all orientations in space---are accounted for, it will be spherically
symmetric.

\subsection{Radial part}

As hinted in the third part of this problem, we took care of all the factors in
equations~(1) and (2) from the problem set already in the angular part. The
radial part will then only have these equations:
\[
    \Gamma_{fi}\ra = \intop_{\Omega_\gammaup} \dif \Omega_\gammaup \,
    \abs{\mathcal M_{fi}\ra}^2
    \eqnsep
    \mathcal M_{fi}\ra = \braket{f|r_\text e|i}
\]

\begin{problem}
    Prove (by induction) and use the following result when evaluation the
    integrals:
    \[
        \int_0^\infty \dif x \, x^n \exp\del{- \frac{x}{x_0}} = n! x_0^{n+1}.
    \]
\end{problem}

\begin{proof}
    We first show that this holds for $n = 0$.
    \[
        \int_0^\infty \dif x \, \exp\del{- \frac{x}{x_0}}
        = \sbr{-x_0 \exp\del{- \frac{x}{x_0}}}_0^\infty
        = x_0
    \]

    Next is the induction step. We start with the left hand side at $n + 1$ and
    try to recover the right hand side at $n+1$ by using the equation at $n$.
    \begin{align*}
        \operatorname{LHS}(n+1)
        &= \int_0^\infty \dif x \, x^{n+1} \exp\del{- \frac{x}{x_0}} \\
        \intertext{%
            Now we use partial integration.
        }
        &= \sbr{- x_0 n^{x+1} \exp\del{- \frac{x}{x_0}}}_0^\infty + [n+1] x_0
        \int_0^\infty \dif x \, x^{n} \exp\del{- \frac{x}{x_0}} \\
        \intertext{%
            The surface term vanishes.
        }
        &= [n+1] x_0 \int_0^\infty \dif x \, x^{n} \exp\del{- \frac{x}{x_0}} \\
        \intertext{%
            Now we use the theorem at $n$.
        }
        &= [n+1] x_0 n! x_0^{n+1} \\
        &= [n+1]! x_0^{[n+1]+1} \\
        &= \operatorname{RHS}(n+1)
    \end{align*}
    By the principle of induction, this equation holds for any $n \geq 0$.
\end{proof}

\subsubsection{Transition $(2P) \to (1S)$}

The radial part of the matrix element is
\begin{align*}
    \mathcal M_{fi}\ra
    &= \int_0^\infty \dif \re \, \re^2 \re R_{21}(\re)^* R_{10}(\re).
    \intertext{%
        The first $\re^2$ comes from the integration measure, the second $\re$
        comes from the braket that we are supposed to compute. Another one will
        come from $R_{21}$. Then there are the radial wave functions according
        to the Hilbert $L^2$ scalar product. Since they are completely real,
        the complex conjugation does not have any effect. We insert the wave
        functions.
    }
    &= \frac{2}{a_0^{3/2}} \frac{1}{2 \sqrt 6 a_0^{5/2}} \int_0^\infty \dif \re
    \, \re^4 \exp\del{-\frac{\re}{a_0}} \exp\del{-\frac{\re}{2a_0}} \\
    \intertext{%
        Before we look at the integration, we simplify everything.
    }
    &= \frac{1}{\sqrt 6 \, a_0^4} \int_0^\infty \dif \re
    \, \re^4 \exp\del{-\frac{3\re}{2a_0}} \\
    \intertext{%
        To apply the formula we just proved, we move the factor 3 into the
        denominator.
    }
    &= \frac{1}{\sqrt 6 \, a_0^4} \int_0^\infty \dif \re
    \, \re^4 \exp\del{-\frac{\re}{\frac23 a_0}} \\
    \intertext{%
        Now we apply the theorem …
    }
    &= \frac{1}{\sqrt 6 \, a_0^4} 4! \sbr{\frac23 a_0}^5 \\
    \intertext{%
        … and simplify again.
    }
    &= \frac{2^8}{3^4 \sqrt 6} a_0
\end{align*}

The transition rate is this squared:
\[
    \Gamma_{fi}\ra = \frac{2^{15}}{3^9} a_0^2
\]

\subsubsection{Transition $(3P) \to (1S)$}

This one works similarly.
\begin{align*}
    \mathcal M_{fi}\ra
    &= \int_0^\infty \dif \re \, \re^2 \re R_{31}(\re)^* R_{10}(\re) \\
    \intertext{%
        We insert the Laguerre polynomials.
    }
    &= \frac{2}{a_0^{3/2}} \frac{4 \sqrt 2}{27 \sqrt 3 \, a_0^{5/2}}
    \int_0^\infty \dif \re
    \, \re^4 \exp\del{-\frac{\re}{a_0}} \sbr{1 - \frac\re{6a_0}} \exp\del{-\frac{\re}{3a_0}} \\
    \intertext{%
        Then we simplify as much as possible, bringing it into a form that
        makes the application of the theorem easy.
    }
    &= \frac{2^3 \sqrt 2}{3^3 \sqrt 3 \, a_0^4} 
    \int_0^\infty \dif \re
    \, \sbr{\re^4 - \frac{\re^5}{6a_0}} \exp\del{-\frac{\re}{\frac34 a_0}} \\
    \intertext{%
        Applying the theorem …
    }
    &= \frac{2^3 \sqrt 2}{3^3 \sqrt 3 \, a_0^4} 
    \sbr{4! \sbr{\frac34 a_0}^5 - \frac{5!}{6 a_0} \sbr{\frac34 a_0}^6} \\
    \intertext{%
        The order of $a_0$ is the same in both terms, we can pull this out.
    }
    &= \frac{2^3 \sqrt 2}{3^3 \sqrt 3} 
    \sbr{4! \sbr{\frac34}^5 - \frac{5!}{6} \sbr{\frac34}^6} a_0
\end{align*}

% TODO Simplify.

% TODO Compute transition rate.

\subsubsection{Transition $(3P) \to (2S)$}

This one is the most convoluted one to compute.
\begin{align*}
    \mathcal M_{fi}\ra
    &= \int_0^\infty \dif \re \, \re^2 \re R_{31}(\re)^* R_{20}(\re) \\
    \intertext{%
        Inserting the Laguerre polynomials.
    }
    &= \frac{1}{\sqrt 2 \, a_0^{5/2}} \frac{4 \sqrt 2}{27 \sqrt 3 \, a_0^{5/2}}
    \int_0^\infty \dif \re \, \re^5 \sbr{1 - \frac{\re}{2a_0}}
    \sbr{1 - \frac{\re}{6a_0}} \exp\del{- \frac{\re}{2a_0}} \exp\del{- \frac{\re}{3a_0}} \\
    \intertext{%
        We simplify a bit.
    }
    &= \frac{4}{27 \sqrt 3 \, a_0^5}
    \int_0^\infty \dif \re \, \re^5 \sbr{1 - \frac{\re}{2a_0}}
    \sbr{1 - \frac{\re}{6a_0}} \exp\del{- \frac{\re}{6a_0}} \\
    \intertext{%
        Now we factor out the square brackets.
    }
    &= \frac{4}{27 \sqrt 3 \, a_0^5}
    \int_0^\infty \dif \re \,
    \sbr{
        \re^5 - \frac{\re^6}{6 a_0} - \frac{\re^6}{2 a_0} + \frac{\re^7}{12 a_0^2}
    }
    \exp\del{- \frac{\re}{6a_0}} \\
    \intertext{%
        We combine the middle summands.
    }
    &= \frac{4}{27 \sqrt 3 \, a_0^5}
    \int_0^\infty \dif \re \,
    \sbr{
        \re^5 - \frac{2\re^6}{3 a_0} + \frac{\re^7}{12 a_0^2}
    }
    \exp\del{- \frac{\re}{6a_0}} \\
    &= \frac{4}{27 \sqrt 3 \, a_0^5}
    \sbr{
        5![6a_0]^6 - \frac{6!}{6a_0} [6 a_0]^7 + \frac{7!}{12 a_0^2} [6a_0]^8
    }
    \intertext{%
        The order of $a_0$ is the same in each term, we pull this out.
    }
    &= \frac{4}{27 \sqrt 3}
    \sbr{
        5! \, 6^6 - \frac{6!}{6} 6^7 + \frac{7!}{12} 6^8
    } a_0
\end{align*}

% TODO

\subsection{Complete transition rate}

The factors $\omega_{if}$ can be computed like so:
\[
    \frac{E_i - E_f}\hbar.
\]
The energies of the hydrogen atom are
\[
    E_n = \frac{E_\text{Ryd}}{n^2}.
\]
Using all this, we can write the angular frequency as
\[
    \omega_{if} = \frac{E_\text{Ryd}}\hbar \sbr{\frac{1}{i^2} - \frac{1}{f^2}}.
\]

The factor we are looking for is:
\[
    \frac{\omega_{13}}{\omega_{12}}
    = \frac{\frac{1}{1^2} - \frac{1}{3^2}}{\frac{1}{1^2} - \frac{1}{2^2}}
    = \frac{1 - \frac{1}{9}}{1 - \frac{1}{4}}
    = \frac{\frac{8}{9}}{\frac{3}{4}}
    = \frac{8 \cdot 4}{9 \cdot 3}
    = \frac{2^5}{3^3}.
\]
The ratio between the angular frequency of $(3P) \to (1S)$ and $(3P) \to (2S)$
is this cubed,
\[
    \sbr{\frac{\omega_{13}}{\omega_{12}}}^3
    = \frac{2^{15}}{3^9} \approx \num{1.66}.
\]

\end{document}

% vim: spell spelllang=en tw=79
