\documentclass[11pt, english, fleqn, DIV=15, headinclude, BCOR=1.5cm]{scrartcl}

\usepackage[
    bibatend,
    color,
]{../header}

\usepackage{tikz}
\usepackage{pdflscape}

\usepackage[tikz]{mdframed}
\newmdtheoremenv[%
    backgroundcolor=black!5,
    innertopmargin=\topskip,
    splittopskip=\topskip,
]{theorem}{Theorem}[section]

\hypersetup{
    pdftitle=
}

\newcounter{totalpoints}
\newcommand\punkte[1]{#1\addtocounter{totalpoints}{#1}}

\newcounter{problemset}
\setcounter{problemset}{6}

\subject{physics606 -- Advanced Quantum Theory}
\ihead{physics606 -- Problem Set \arabic{problemset}}

\title{Problem Set \arabic{problemset}}

\publishers{Group 2 -- Dilege Gülmez}
\ofoot{Group 2 -- Dilege Gülmez}

\newmdenv[%
    backgroundcolor=black!5,
    frametitlebackgroundcolor=black!10,
    roundcorner=5pt,
    skipabove=\topskip,
    innertopmargin=\topskip,
    splittopskip=\topskip,
    frametitle={Problem statement},
    frametitlerule=true,
    nobreak=true,
]{problem}

\newmdenv[%
    backgroundcolor=white,
    frametitlebackgroundcolor=black!10,
    roundcorner=5pt,
    skipabove=\topskip,
    innertopmargin=\topskip,
    innerbottommargin=8cm,
    splittopskip=\topskip,
    frametitle={Side question},
    frametitlerule=true,
]{question}


\author{
    Martin Ueding \\ \small{\href{mailto:mu@martin-ueding.de}{mu@martin-ueding.de}}
    \and
    Lino Lemmer \\ \small{\href{mailto:l2@uni-bonn.de}{l2@uni-bonn.de}}
}
\ifoot{Martin Ueding, Lino Lemmer}

\ohead{\rightmark}

\begin{document}

\maketitle

\vspace{3ex}

\begin{center}
    \begin{tabular}{rrr}
        problem number & achieved points & possible points \\
        \midrule
        1 & & \punkte{25} \\
        \midrule
        total & & \arabic{totalpoints}
    \end{tabular}
\end{center}

\section{Rate for (nP) to (n'S) transitions}

\subsection{Orthogonality}

\newcommand\tg{\theta_\gammaup}
\newcommand\phg{\phi_\gammaup}
\newcommand\te{\theta_\text e}
\newcommand\phe{\phi_\text e}

Given is the wave vector
\[
    \vec k_\gammaup = k_\gammaup
    \begin{pmatrix}
        \sin(\tg) \sin(\phg) \\
        \sin(\tg) \cos(\phg) \\
        \cos(\tg)
    \end{pmatrix}
\]
where we have used the notation where a vector in non-bold type is its absolute
value, consistent with its look as a scalar. Then there are also two
polarization basis vectors given:
\[
    \vec \epsilon_1 =
    \begin{pmatrix}
        \cos(\phg) \\
        - \sin(\phg) \\
        0
    \end{pmatrix}
    \eqnsep
    \vec \epsilon_2 =
    \begin{pmatrix}
        \cos(\tg) \sin(\phg) \\
        \cos(\tg) \cos(\phg) \\
        - \cos(\tg)
    \end{pmatrix}.
\]

Checking for orthonormality among the $\vec \epsilon_i$ is $X - X = 0$. We done
that, it checks out. It is also easy to see that $\vec \epsilon_1 \vec
k_\gammaup = 0$. The last one requires application of the so called
trigonometric identities. We present all the steps here in painstaking detail:
\begin{align*}
    \vec \epsilon_2 \vec k_\gamma
    &= \cos(\tg) \sin(\tg) \sin(\phg)^2 + \sin(\tg) \cos(\tg) \cos(\phg)^2 -
    \sin(\tg) \cos(\tg) \\
    \intertext{%
        The first two terms contain common factors. We factor those out to get
        a more compact expression.
    }
    &= \cos(\tg) \sin(\tg) \sbr{ \sin(\phg)^2 + \cos(\phg)^2} - \sin(\tg) \cos(\tg)
    \\
    \intertext{%
        Here we use $\sin(\phg)^2 + \cos(\phg)^2 = 1$. This leaves us with
    }
    &= \cos(\tg) \sin(\tg) - \sin(\tg) \cos(\tg). \\
    \intertext{%
        After using commutation law of the product we have to equal terms. They
        cancel each other and just gives us
    }
    &= 0.
\end{align*}

The more interesting question, which was not asked in this problem, is “Why do
the polarization vectors have to be perpendicular to the wave vector?”. This is
because electromagnetic radiation is vector-polarized and a transverse wave.
Therefore the polarization is perpendicular to the direction of propagation.
This also means that the photon has spin 1. Gravitational waves are
tensor-polarized and the graviton would have spin 2.

\subsection{Angular part of integrands}

We begin by copying the terms.
\begin{align*}
    \vec x_\text e \vec \epsilon_1
    &= r_\text e
    \begin{pmatrix}
        \sin(\te) \sin(\phe) \\
        \sin(\te) \cos(\phe) \\
        \cos(\te)
    \end{pmatrix}
    \begin{pmatrix}
        \cos(\phg) \\
        - \sin(\phg) \\
        0
    \end{pmatrix} \\
    &= r_\text e \sbr{
        \sin(\te) \sin(\phe) \cos(\phg) - \sin(\te) \sin(\phe) \sin(\phg)
    } \\
    &= r_\text e \sin(\te) \sbr{
        \sin(\phe) \cos(\phg) - \sin(\phe) \sin(\phg)
    } \\
    \intertext{%
        Now we apply the addition theorem.
    }
    &= r_\text e \sin(\te) \sin(\phe - \phg)
\end{align*}

The second one is very similar.
\begin{align*}
    \vec x_\text e \vec \epsilon_2
    &= r_\text e
    \begin{pmatrix}
        \sin(\te) \sin(\phe) \\
        \sin(\te) \cos(\phe) \\
        \cos(\te)
    \end{pmatrix}
    \begin{pmatrix}
        \cos(\tg) \sin(\phg) \\
        \cos(\tg) \cos(\phg) \\
        - \cos(\tg)
    \end{pmatrix} \\
    &= r_\text e \sbr{
        \sin(\te) \sin(\phe) \cos(\tg) \sin(\phg)
        + \sin(\te) \cos(\phe) \cos(\tg) \cos(\phg)
        - \cos(\te) \sin(\tg)
    }
    \intertext{%
        We also apply the addition theorem here.
    }
    &= r_\text e \sbr{
        \sin(\te) \cos(\tg) \cos(\phe - \phg)
        - \cos(\te) \sin(\tg)
    }
\end{align*}

\subsection{Evaluation of the integrals}

Now we got the middle of the braket in terms of the angles. Next are initial
and final states. Since there are a lot of combinations of $m_1$ and $\vec
\epsilon$, we will do this in sections.

\subsubsection{Case $\vec \epsilon = \vec \epsilon_1$}

\subsubsection{Case $\vec \epsilon = \vec \epsilon_2$}

\paragraph{Case $m_1 = 0$}

We start by calculating the matrix element
\begin{align*}
    \mathcal M_{fi}
    &= - \frac{\omega_{if}}c \braket{f|\vec \epsilon_2 \vec x_e | i}. \\
    \intertext{%
        In the first step, we insert the knowns. $\bra f$ just gives a
        prefactor. The scalar product is inserted. $\ket i$ gives us a
        prefactor, the exponential function is 1 since $m_i = 0$ is assumed. In
        $f$, we get another $\cos(\te)$ and a prefactor.
    }
    &= - \frac{\sqrt 3}{4 \piup} \frac{\omega_{if}}c r_\text e
    \int \dif \Omega_\text e \,
    \sbr{
        \sin(\te) \cos(\tg) \int_0^{2\piup} \dif \phe \cos(\phe - \phg)
        - \cos(\te) \sin(\tg)
    } \cos(\te)
    \intertext{%
        The volume element is $\dif \Omega = \dif \theta \, \dif \phi \,
        \sin(\theta)$. We insert this.
    }
    &= - \frac{\sqrt 3}{4 \piup} \frac{\omega_{if}}c r_\text e
    \int_0^\pi \dif \te \, \sin(\te) \\&\qquad \cdot \int_0^{2\piup} \dif \phe \,
    \sbr{
        \sin(\te) \cos(\tg) \cos(\phe - \phg)
        - \cos(\te) \sin(\tg)
    } \cos(\te)
    \intertext{%
        The integral over $\phe$ will give zero for the first summand in the
        square bracket. For the second, it will just contribute a factor of $2
        \piup$ since there is no dependence on $\phe$.
    }
    &= \frac{\sqrt 3}{2} \frac{\omega_{if}}c r_\text e \sin(\tg)
    \int_0^\pi \dif \te \, \sin(\te) \cos(\te)^2
    \intertext{%
        This integral can be solved using the substitution $z := \cos(\te)$
        and gives the value $2/3$. Our final result is
    }
    &= \frac{1}{\sqrt 3} \frac{\omega_{if}}c r_\text e \sin(\tg).
\end{align*}

\end{document}

% vim: spell spelllang=en tw=79
