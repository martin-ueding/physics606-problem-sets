\documentclass[11pt, english, fleqn, DIV=15, headinclude, BCOR=1.5cm]{scrartcl}

\usepackage[
    bibatend,
    %color,
]{../header}

\usepackage{tikz}
\usepackage{pdflscape}

\usepackage[tikz]{mdframed}
\newmdtheoremenv[%
    backgroundcolor=black!5,
    innertopmargin=\topskip,
    splittopskip=\topskip,
]{theorem}{Theorem}[section]

\hypersetup{
    pdftitle=
}

\newcounter{totalpoints}
\newcommand\punkte[1]{#1\addtocounter{totalpoints}{#1}}

\newcounter{problemset}
\setcounter{problemset}{9}

\subject{physics606 -- Advanced Quantum Theory}
\ihead{physics606 -- Problem Set \arabic{problemset}}

\title{Problem Set \arabic{problemset}}

\publishers{Group 2 -- Dilege Gülmez}
\ofoot{Group 2 -- Dilege Gülmez}

\newmdenv[%
    backgroundcolor=black!0,
    frametitlebackgroundcolor=black!0,
    roundcorner=5pt,
    skipabove=\topskip,
    innertopmargin=\topskip,
    splittopskip=\topskip,
    frametitle={Problem statement},
    frametitlerule=true,
    nobreak=true,
]{problem}

\newmdenv[%
    backgroundcolor=white,
    frametitlebackgroundcolor=black!0,
    roundcorner=5pt,
    skipabove=\topskip,
    innertopmargin=\topskip,
    innerbottommargin=8cm,
    splittopskip=\topskip,
    frametitle={Side question},
    frametitlerule=true,
]{question}

\newcommand\an{^\text{angular}}
\newcommand\ra{^\text{radial}}


\author{
    Martin Ueding \\ \small{\href{mailto:mu@martin-ueding.de}{mu@martin-ueding.de}}
    \and
    Lino Lemmer \\ \small{\href{mailto:l2@uni-bonn.de}{l2@uni-bonn.de}}
}
\ifoot{Martin Ueding, Lino Lemmer}

\ohead{\rightmark}

\begin{document}

\maketitle

\vspace{3ex}

\begin{center}
    \begin{tabular}{rrr}
        problem number & achieved points & possible points \\
        \midrule
        1 & & \punkte{15} \\
        2 & & \punkte{15} \\
        \midrule
        total & & \arabic{totalpoints}
    \end{tabular}
\end{center}

\section{Expansion of a plane wave}

\subsection{Spherical Bessel functions}

We start with the definition.
\begin{align*}
    j_l(x)
    &= - x^l \sbr{\frac 1x \od{}x}^l \frac{\sin(x)}{x} \\
    \intertext{%
        Now we insert the series expansion of the sinc function.
    }
    &= - x^l \sbr{\frac 1x \od{}x}^l \sum_{n=0}^\infty \frac{[-1]^n}{[2n+1]!}
    x^{2n} \\
    \intertext{%
        Commuting terms.
    }
    &= - \sum_{n=0}^\infty \frac{[-1]^n}{[2n+1]!} x^l \sbr{\frac 1x \od{}x}^l
    x^{2n} \\
    \intertext{%
        The big square bracket will decrease the power of $x^{2n}$ by $2l$ in
        total. Only terms with $n \geq l$ will contribute. We will omit higher
        terms. This leaves only one interesting term, we can drop the sum and
        set $n = l$.
    }
    &= - \frac{[-1]^l}{[2l+1]!} x^l \sbr{\frac 1x \od{}x}^l x^{2l} + \mathcal O(x^{l + 2}) \\
    \intertext{%
        The factors that we get by differentiating is every second of $2l,
        2l-2, 2l-4, \ldots =: [2l]!!$.
    }
    &= - \frac{[-1]^l [2l]!!}{[2l+1]!} x^l + \mathcal O(x^{l + 2}) \\
    \intertext{%
        Realizing that
        \[
            [2l]!!
            = [2l][2l-2][2l-4]\ldots
            = 2 l 2 [l-1] 2 [l-2] \ldots
            = 2^l l!
        \]
        gives
    }
    &= [-1]^{l+1} \frac{2^l l!}{[2l+1]!} x^l + \mathcal O(x^{l + 2})
\end{align*}
This result differers from the result on the problem set by the $l$ dependent
sign.

\subsection{Expression of $x^l$}

We only look at the highest term in the Legendre polynomial:
\[
    P_l(x)
    = \frac{1}{2^l l!} \od{^l}{x^l} \sbr{x^{2l} + \ldots}
    = \frac{1}{2^l l!} \frac{[2l]!}{l!} x^{l} + \ldots
\]
The other terms contain lesser powers of $x$ than $x^l$. Then this can be
inverted to
\[
    x^l = \frac{2^l [l!]^2}{[2l]!} P_l(x) + \ldots.
\]

\subsection{Use of orthogonality}

We define $\xi := \cos(\theta)$ for this subsection. It is slightly confusing
that $k$ is used as the wave number and an index, we use $m$ instead.

Start with expansion of plane wave.
\begin{align*}
    \exp(\iup k x \xi) &= \sum_{l = 0}^\infty A_l j_l(kx) P_l(\xi) \\
    \intertext{%
        $L^2$ project this onto $P_m(\xi)$.
    }
    \int_{-1}^1 \dif \xi \, P_m(\xi) \exp(\iup k x \xi) &= \sum_{l = 0}^\infty A_l
    j_l(kx) \int_{-1}^1 \dif \xi \, P_m(\xi) P_l(\xi) \\
    \intertext{%
        Use orthogonality.
    }
    \int_{-1}^1 \dif \xi \, P_m(\xi) \exp(\iup k x \xi) &= \sum_{l = 0}^\infty A_l
    j_l(kx) \frac{2}{2l+1} \delta_{lm} \\
    \intertext{%
        Execute $\delta$.
    }
    \int_{-1}^1 \dif \xi \, P_m(\xi) \exp(\iup k x \xi) &= A_m
    j_m(kx) \frac{2}{2k+1} \\
    \intertext{%
        Move fraction to other side.
    }
    \frac{2m+1}{2} \int_{-1}^1 \dif \xi \, P_m(\xi) \exp(\iup k x \xi) &= A_m
    j_m(kx)
\end{align*}
This is not the result on the problem set, since the integration measure
$\dif\cos(\theta) = \dif\xi$ is missing there.

\subsection{Extraction of $A_l$}

We start with the expression we just derived.
\begin{align*}
    A_l j_l(kx)
    &= \frac{2l+1}{2} \int_{-1}^1 \dif \xi \, P_l(\xi) \exp(\iup k x \xi) \\
    \intertext{%
        We express $j_l$ through its approximation around $kx = 0$ and express
        $P_l$ in terms of $x$.
    }
    A_m [kx]^l \frac{2^l l!}{[2l+1]!}
    &= \frac{2l+1}{2} \int_{-1}^1 \dif \xi \, \frac{[2l]!}{2^l[l!]^2} \xi^l \exp(\iup k x \xi) \\
    \intertext{%
        Next we insert the \emph{definition} of the exponential.
    }
    A_m [kx]^l \frac{2^l l!}{[2l+1]!}
    &= \frac{2l+1}{2} \int_{-1}^1 \dif \xi \, \frac{[2l]!}{2^l[l!]^2} \xi^l
    \sbr{1 + \iup k x \xi + \mathcal O(\xi^2)} \\
    \intertext{%
        We move the Landau $\mathcal O$ out of the integral.
    }
    A_m [kx]^l \frac{2^l l!}{[2l+1]!}
    &= \frac{2l+1}{2} \int_{-1}^1 \dif \xi \, \frac{[2l]!}{2^l[l!]^2} \xi^l
    [1 + \iup k x \xi] + \mathcal O(\xi^{l+2}) \\
    \intertext{%
        Move the factors to the right hand side.
    }
    A_m &= \frac{[2l+1]!}{2^l l!} \frac{2l+1}{2} \frac{1}{[kx]^l}
    \frac{[2l]!}{2^l[l!]^2} \int_{-1}^1 \dif \xi \, \xi^l [1 + \iup k x \xi] +
    \mathcal O(\xi^{l+2}) \\
    \intertext{%
        Simplify.
    }
    A_m &= [2l+1]! [2l+1] \frac{[2l]!}{2^{2l+1} [l!]^3} \frac{1}{[kx]^l}
    \int_{-1}^1 \dif \xi \, \xi^l [1 + \iup k x \xi] + \mathcal O(\xi^{l+2}) \\
    A_m &= [2l+1]^2 \frac{\sbr{[2l]!}^2}{2^{2l+1} [l!]^3} \frac{1}{[kx]^l}
    \int_{-1}^1 \dif \xi \, \xi^l [1 + \iup k x \xi] + \mathcal O(\xi^{l+2})
\end{align*}
One of the integrals will give zero, the other will contribute something that
only depends on $kx$.

\section{Scattering on a dipole}


\end{document}

% vim: spell spelllang=en tw=79
