\documentclass[11pt, english, fleqn, DIV=15, headinclude, BCOR=1cm]{scrartcl}

\usepackage[
    bibatend,
    color,
]{../header}

\usepackage{tikz}

\usepackage[tikz]{mdframed}
\newmdtheoremenv[%
    backgroundcolor=black!5,
    innertopmargin=\topskip,
    splittopskip=\topskip,
]{theorem}{Theorem}[section]

\hypersetup{
    pdftitle=
}

\newcounter{totalpoints}
\newcommand\punkte[1]{#1\addtocounter{totalpoints}{#1}}

\newcounter{problemset}
\setcounter{problemset}{4}

\subject{physics606 -- Advanced Quantum Theory}
\ihead{physics606 -- Problem Set \arabic{problemset}}

\title{Problem Set \arabic{problemset}}

\publishers{Group 2 -- Dilege Gülmez}
\ofoot{Group 2 -- Dilege Gülmez}

\newmdenv[%
    backgroundcolor=black!5,
    frametitlebackgroundcolor=black!10,
    roundcorner=5pt,
    skipabove=\topskip,
    innertopmargin=\topskip,
    splittopskip=\topskip,
    frametitle={Problem statement},
    frametitlerule=true,
]{problem}

\newmdenv[%
    backgroundcolor=white,
    frametitlebackgroundcolor=black!10,
    roundcorner=5pt,
    skipabove=\topskip,
    innertopmargin=\topskip,
    innerbottommargin=8cm,
    splittopskip=\topskip,
    frametitle={Side question},
    frametitlerule=true,
    nobreak=true,
]{question}


\author{
    Martin Ueding \\ \small{\href{mailto:mu@martin-ueding.de}{mu@martin-ueding.de}}
    \and
    Lino Lemmer \\ \small{\href{mailto:l2@uni-bonn.de}{l2@uni-bonn.de}}
}
\ifoot{Martin Ueding, Lino Lemmer}

\ohead{\rightmark}

\begin{document}

\maketitle

\vspace{3ex}

\begin{center}
    \begin{tabular}{rrr}
        problem number & achieved points & possible points \\
        \midrule
        1 & & \punkte{9} \\
        2 & & \punkte{5} \\
        3 & & \punkte{8} \\
        \midrule
        Total & & \arabic{totalpoints}
    \end{tabular}
\end{center}

In order to make this document a bit more self-sufficient, we summarize the
problem statements in front of our solutions. That will hopefully allow for
better presentation of the results.

\section{Propagator of the harmonic oscillator} % GH-11

\begin{problem}
    Let $L$ be the Lagrangian of the harmonic oscillator in one dimension:
    \[
        L = \frac 12 m [\dot x^2 - \omega^2 x^2].
    \]
\end{problem}

\subsection{Classical action}

\begin{problem}
    Show that the classical action $S_\text{cl}(x, x', t)$ has the given form.
\end{problem}

The action is defined as:
\[
    S = \int_0^t \dif t' \, L\del{x(t'), \dot x(t'), t'}.
\]

We use the hint that is given on the problem set and express the trajectory as
a linear combination of basis trajectories:
\[
    x(t') = a \cos(\omega t') + b \sin(\omega t').
\]
The boundary conditions are a little strange. In the wording, it says that the
particle goes from point $x_0$ at $t = 0$ to the point $x$ at time $t$. Then
the action is written with dependency on $x$ and $x'$. We will use $x(0) =:
x_0$ and $x(t) =: x_1$ here. This already is the boundary condition, so $a$ and
$b$ are:
\[
    a = x_0
    \eqnsep
    b = \frac{x_1 - x_0 \cos(\omega t)}{\sin(\omega t)}.
\]

This can now be inserted into the Lagrangian function:
\begin{align*}
    S_\text{cl}(x_0, x_1, t)
    &= \int_0^t \dif t' \, L\del{x(t'), \dot x(t'), t'} \\
    &= \frac{m}{2} \int_0^t \dif t' \,
    \sbr{
        \sbr{- a \omega \sin(\omega t') + b \omega \cos(\omega t')}^2
        - \omega^2 \sbr{a \cos(\omega t') + b \sin(\omega t')}^2
    } \\
    \intertext{%
        We expand all the squares and regroup the terms again. Currently, all
        arguments for sine and cosine are $\omega t'$, so we will omit them for
        a more compact view of the expression. When the arguments change in the
        next steps, we will show them explicitly.
    }
    &= \frac{m}{2} \int_0^t \dif t' \,
    \sbr{
        [a^2 - b^2][\sin^2 - \cos^2] - 4 a b \cos \sin
    } \\
    \intertext{%
        Now we perform the integration. The indefinite integral to $\sin^2 -
        \cos^2$ is given by
        \[
            - \frac{\sin(2\omega t')}{2\omega} = - \frac{2 \cos \sin}{2\omega}.
        \]
        The integral of $\cos\sin$ is given by
        \[
            - \frac{\cos^2}{2\omega}.
        \]
        The use those and insert into the previous equation.
    }
    &= - \frac{m \omega^2}{2}
    \sbr{
        [a^2 - b^2]\frac{\cos\sin}{\omega} - 2 a b \frac{\cos^2}{\omega}
    }_0^t \\
    \intertext{%
        Now we evaluate at $t' = t$ and $t' = 0$. Since all the arguments
        change from $\omega t'$ to $\omega t$, we redefine the omission and
        still omit all the arguments.
    }
    &= - \frac{m \omega^2}{2}
    \sbr{
        [a^2 - b^2]\frac{\cos\sin}{\omega} - 2 a b \frac{\cos^2 - 1}{\omega}
    } \\
    \intertext{%
        Now we insert $a$ and $b$ explicitly.
    }
    &= - \frac{m \omega^2}{2}
    \sbr{
        \sbr{x_0^2 - \sbr{\frac{x_1 - x_0 \cos}{\sin}}^2}\frac{\cos\sin}{\omega} -
        2 x_0 \frac{x_1 - x_0 \cos}{\sin} \frac{\cos^2 - 1}{\omega}
    } \\
    \intertext{%
        We get rid of the minus in front and cancel one $\omega$.
    }
    &= \frac{m \omega}{2}
    \sbr{
        \sbr{- x_0^2 + \sbr{\frac{x_1 - x_0 \cos}{\sin}}^2}\cos\sin +
        2 x_0 \frac{x_1 - x_0 \cos}{\sin} [\cos^2 - 1]
    } \\
\end{align*}

\section{Ordinary path integral from phase space path integral} % GH-12

\section{Path integral with vector potential} % GH-13



\end{document}

% vim: spell spelllang=en tw=79
