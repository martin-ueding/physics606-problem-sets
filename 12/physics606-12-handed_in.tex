\documentclass[11pt, english, fleqn, DIV=15, headinclude, BCOR=1.5cm]{scrartcl}

\usepackage[
    bibatend,
    color,
]{../header}

\usepackage{tikz}
\usepackage{pdflscape}

\usepackage[tikz]{mdframed}
\newmdtheoremenv[%
    backgroundcolor=black!5,
    innertopmargin=\topskip,
    splittopskip=\topskip,
]{theorem}{Theorem}[section]

\hypersetup{
    pdftitle=
}

\newcounter{totalpoints}
\newcommand\punkte[1]{#1\addtocounter{totalpoints}{#1}}

\newcounter{problemset}
\setcounter{problemset}{12}

\subject{physics606 -- Advanced Quantum Theory}
\ihead{physics606 -- Problem Set \arabic{problemset}}

\title{Problem Set \arabic{problemset}}

\publishers{Group 2 -- Dilege Gülmez}
\ofoot{Group 2 -- Dilege Gülmez}

\newmdenv[%
    backgroundcolor=black!0,
    frametitlebackgroundcolor=black!0,
    roundcorner=5pt,
    skipabove=\topskip,
    innertopmargin=\topskip,
    splittopskip=\topskip,
    frametitle={Problem statement},
    frametitlerule=true,
    nobreak=true,
]{problem}

\newmdenv[%
    backgroundcolor=white,
    frametitlebackgroundcolor=black!0,
    roundcorner=5pt,
    skipabove=\topskip,
    innertopmargin=\topskip,
    innerbottommargin=8cm,
    splittopskip=\topskip,
    frametitle={Side question},
    frametitlerule=true,
]{question}

\newcommand\an{^\text{angular}}
\newcommand\ra{^\text{radial}}


\author{
    Martin Ueding \\ \small{\href{mailto:mu@martin-ueding.de}{mu@martin-ueding.de}}
    \and
    Lino Lemmer \\ \small{\href{mailto:l2@uni-bonn.de}{l2@uni-bonn.de}}
}
\ifoot{Martin Ueding, Lino Lemmer}

\ohead{\rightmark}

\begin{document}

\maketitle

\vspace{3ex}

\begin{center}
    \begin{tabular}{rrr}
        problem number & achieved points & possible points \\
        \midrule
        1 & & \punkte{7} \\
        2 & & \punkte{14} \\
        \midrule
        total & & \arabic{totalpoints}
    \end{tabular}
\end{center}

\section{Two-particle operators in second quantization}

\subsection{}

Let $\mathbb P$ be the set of all particles such that $\alpha, \beta \in
\mathbb P$. The sum
\[
    \sum_{\alpha \neq \beta}
\]
means that it is to be summed over all $\alpha$ and $\beta$ but not those where
they are equal. We write this more formally as
\begin{align*}
    F &= \frac 12 \sum_{\alpha \in \mathbb P} \sum_{\beta \in \mathbb P
    \setminus \{\alpha\}} f(\vec x_\alpha, \vec x_\beta).
    \intertext{%
        Now we insert sets of ones.
    }
    &= \frac 12 \sum_{\alpha \in \mathbb P}
    \sum_{\beta \in \mathbb P \setminus \{\alpha\}}
    \sum_{i,j,k,l}
    \ket i_\alpha \bra i_\alpha \ket j_\beta \bra j_\beta
    f(\vec x_\alpha, \vec x_\beta)
    \ket k_\alpha \bra k_\alpha \ket l_\beta \bra l_\beta
    \intertext{%
        We commute the bra and ket vectors of different particles which we can
        do since we consider the product Hilbert space of two different
        particles.
    }
    &= \frac 12 \sum_{\alpha \in \mathbb P}
    \sum_{\beta \in \mathbb P \setminus \{\alpha\}}
    \sum_{i,j,k,l}
    \ket i_\alpha \ket j_\beta \bra i_\alpha \bra j_\beta
    f(\vec x_\alpha, \vec x_\beta)
    \ket k_\alpha \ket l_\beta \bra k_\alpha \bra l_\beta
    \intertext{%
        We can then use the notation which corresponds to the tensor product
        Hilbert space of two particles.
    }
    &= \frac 12 \sum_{\alpha \in \mathbb P}
    \sum_{\beta \in \mathbb P \setminus \{\alpha\}}
    \sum_{i,j,k,l}
    \ket i_\alpha \ket j_\beta \bra{i,j}
    f(\vec x_\alpha, \vec x_\beta)
    \ket{k,l} \bra k_\alpha \bra l_\beta
    \intertext{%
        The term in the middle is a scalar and can be moved anywhere in the
        product.
    }
    &= \frac 12 \sum_{\alpha \in \mathbb P}
    \sum_{\beta \in \mathbb P \setminus \{\alpha\}}
    \sum_{i,j,k,l}
    \bra{i,j} f(\vec x_\alpha, \vec x_\beta) \ket{k,l}
    \ket i_\alpha \ket j_\beta
    \bra k_\alpha \bra l_\beta
\end{align*}
This is the desired result in equation~(2) of the problem set.

\subsection{}

We start with equation~(2) from the problem set.
\begin{align*}
    F &= \frac 12 \sum_{\alpha \in \mathbb P}
    \sum_{\beta \in \mathbb P \setminus \{\alpha\}}
    \sum_{i,j,k,l}
    \bra{i,j} f(\vec x_\alpha, \vec x_\beta) \ket{k,l}
    \ket i_\alpha \ket j_\beta
    \bra k_\alpha \bra l_\beta
    \intertext{%
        Then we change the order of the bra and ket vectors at the right side
        of the equation.
    }
    &= \frac 12
    \sum_{\alpha \in \mathbb P}
    \sum_{\beta \in \mathbb P \setminus \{\alpha\}}
    \sum_{i,j,k,l}
    \bra{i,j} f(\vec x_\alpha, \vec x_\beta) \ket{k,l}
    \ket i_\alpha
    \bra k_\alpha
    \ket j_\beta
    \bra l_\beta
    \intertext{%
        We change the order of the summations such that we can plug in
        equation~(5) from the problem set later on. The underbraces denote the
        outcome suggestively, we still have to work on that, though.
    }
    &= \frac 12
    \sum_{i,j,k,l}
    \bra{i,j} f(\vec x_\alpha, \vec x_\beta) \ket{k,l}
    \underbrace{
        \sum_{\alpha \in \mathbb P}
        \ket i_\alpha
        \bra k_\alpha
    }_{\leadsto b^\dagger_i b_k}
    \underbrace{
        \sum_{\beta \in \mathbb P \setminus \{\alpha\}}
        \ket j_\beta
        \bra l_\beta
    }_{\leadsto b^\dagger_j b_l}
    \intertext{%
        The problem is that the sum over $\beta$ excludes $\alpha$. Therefore,
        the sums are not independent and we cannot apply equation~(5) yet. So
        we just sum over all $\beta \in \mathbb P$ in the sum and subtract that
        particular value again.
    }
    &= \frac 12
    \sum_{i,j,k,l}
    \bra{i,j} f(\vec x_\alpha, \vec x_\beta) \ket{k,l}
    \sum_{\alpha \in \mathbb P}
    \ket i_\alpha
    \bra k_\alpha
    \sbr{
        \sum_{\beta \in \mathbb P}
        \ket j_\beta
        \bra l_\beta
        -
        \ket j_\alpha
        \bra l_\alpha
    }
    \intertext{%
        The one summation over $\beta$ is now independent of $\alpha$ and we
        can use equation~(5).
    }
    &= \frac 12
    \sum_{i,j,k,l}
    \bra{i,j} f(\vec x_\alpha, \vec x_\beta) \ket{k,l}
    \sum_{\alpha \in \mathbb P}
    \ket i_\alpha
    \bra k_\alpha
    \sbr{
        b^\dagger_j b_l
        -
        \ket j_\alpha
        \bra l_\alpha
    }
    \intertext{%
        Now we expand the parentheses.
    }
    &= \frac 12
    \sum_{i,j,k,l}
    \bra{i,j} f(\vec x_\alpha, \vec x_\beta) \ket{k,l}
    \sbr{
        b^\dagger_j b_l
        \sum_{\alpha \in \mathbb P}
        \ket i_\alpha
        \bra k_\alpha
        -
        \sum_{\alpha \in \mathbb P}
        \ket i_\alpha
        \bra k_\alpha
        \ket j_\alpha
        \bra l_\alpha
    }
    \intertext{%
        We apply equation~(5) once again.
    }
    &= \frac 12
    \sum_{i,j,k,l}
    \bra{i,j} f(\vec x_\alpha, \vec x_\beta) \ket{k,l}
    \sbr{
        b^\dagger_i b_k
        b^\dagger_j b_l
        -
        \sum_{\alpha \in \mathbb P}
        \ket i_\alpha
        \bra k_\alpha
        \ket j_\alpha
        \bra l_\alpha
    }
    \intertext{%
        This looks somewhat like equation~(4) from the problem set which we
        should arrive at. The terms are not in the correct order yet, though.
        The bra-ket in the middle is just a Kronecker $\delta$.
    }
    &= \frac 12
    \sum_{i,j,k,l}
    \bra{i,j} f(\vec x_\alpha, \vec x_\beta) \ket{k,l}
    \sbr{
        b^\dagger_i b_k
        b^\dagger_j b_l
        -
        \sum_{\alpha \in \mathbb P}
        \ket i_\alpha
        \delta_{kj}
        \bra l_\alpha
    }
    \intertext{%
        However, the Kronecker $\delta$ is also contained in the commutator of
        the creation and annihilation operators: $\delta_{kj} = [b_k,
        b^\dagger_j]_\mp$
        \parencite[16]{Schwabl/Quantenmechanik_fuer_Fortgeschrittene}. Since it
        is a plain number, we pull it out front for a step.
    }
    &= \frac 12
    \sum_{i,j,k,l}
    \bra{i,j} f(\vec x_\alpha, \vec x_\beta) \ket{k,l}
    \sbr{
        b^\dagger_i b_k
        b^\dagger_j b_l
        -
        \delta_{kj}
        \sum_{\alpha \in \mathbb P}
        \ket i_\alpha
        \bra l_\alpha
    }
    \intertext{%
        Now we apply equation~(5) from the problem set again.
    }
    &= \frac 12
    \sum_{i,j,k,l}
    \bra{i,j} f(\vec x_\alpha, \vec x_\beta) \ket{k,l}
    \sbr{
        b^\dagger_i b_k
        b^\dagger_j b_l
        -
        \delta_{kj}
        b^\dagger_i b_l
    }
    \intertext{%
        Now we move the Kronecker $\delta$ in between those terms.
    }
    &= \frac 12
    \sum_{i,j,k,l}
    \bra{i,j} f(\vec x_\alpha, \vec x_\beta) \ket{k,l}
    \sbr{
        b^\dagger_i b_k
        b^\dagger_j b_l
        -
        b^\dagger_i
        \delta_{kj}
        b_l
    }
    \intertext{%
        We expand the (anti-)commutator there.
    }
    &= \frac 12
    \sum_{i,j,k,l}
    \bra{i,j} f(\vec x_\alpha, \vec x_\beta) \ket{k,l}
    \sbr{
        b^\dagger_i b_k
        b^\dagger_j b_l
        -
        b^\dagger_i
        \sbr{ b_k b^\dagger_j \mp b_k b^\dagger_j }
        b_l
    }
    \intertext{%
        Then we expand the inner square bracket.
    }
    &= \frac 12
    \sum_{i,j,k,l}
    \bra{i,j} f(\vec x_\alpha, \vec x_\beta) \ket{k,l}
    \sbr{
        b^\dagger_i b_k b^\dagger_j b_l
        -
        b^\dagger_i b_k b^\dagger_j b_l
        \pm
        b^\dagger_i b^\dagger_j b_k b_l
    }
    \intertext{%
        The first two terms cancel each other, so we are left with
    }
    &= \pm \frac 12
    \sum_{i,j,k,l}
    \bra{i,j} f(\vec x_\alpha, \vec x_\beta) \ket{k,l}
    b^\dagger_i b^\dagger_j b_k b_l.
\end{align*}
The version on the problem set has $k$ and $l$ switched. Since the annihilation
operators always commute, those two versions are equivalent.

\section{Hartree-Fock approximation for atoms}

We understand equation~(8) from the problem set that there are $N$ states
occupied and those are the states with the lowest numbers from 1 to $N$.

\subsection{}

\begin{problem}
    Show that
    \[
        \braket{\psi|b_i^\dagger b_j|\psi} = \delta_{ij}.
    \]
\end{problem}

We have done this in three similar ways. We will start with the third one that
we came up with because it is really succinct.

\subsubsection{Third way}

$b_j$ will destroy a particle in state $\ket j$. $b_i^\dagger$ will create one
in the state $\ket i$. Since those are fermions, $i$ has to be either the same
as $j$ or $i > N$ because there are no other free spots. If $i > N$, then this
altered $\ket \psi$ will be orthogonal to $\bra \psi$. If $i = j$, then the two
operators are just the occupation number operator $n_j$. Its eigenvalue here is
1 since $j \leq N$ and fermions was assumed. $\ket \psi$ is left unchanged.
Therefore it is $\delta_{ij}$.

\subsubsection{First way}

Now we will use the Ansatz directly.
\begin{align*}
    \braket{\psi|b_i^\dagger b_j|\psi}
    &= \sbr{\prod_{k=1}^N b_k^\dagger \ket 0}^\dagger b_i^\dagger
    b_j \sbr{\prod_{k=1}^N b_k^\dagger \ket 0}
    \intertext{%
        Now we incorporate the two stray operators into the brackets.
    }
    &= \sbr{b_i \prod_{k=1}^N b_k^\dagger \ket 0}^\dagger
    \sbr{b_j \prod_{k=1}^N b_k^\dagger \ket 0}
    \intertext{%
        The $b$ anticommute with the $b^\dagger$ like so:
        \[
            [b_i, b_j^\dagger]_+ = \delta_{ij}.
        \]
        That means that we can move the $b_j$ into the spot in the product over
        $k$ such that we have the $b_j$ in front of the $b_j^\dagger$. We will
        need $i-1$ anticommutations for that. The same applies in the second
        bracket. We do not really like the notation with omissions, but this
        actually seems easier here.
    }
    &= [-1]^{i + j - 2}
    \sbr{
        b_1^\dagger
        b_2^\dagger
        \ldots
        b_{i-1}^\dagger
        b_{i+1}^\dagger
        \ldots
        b_N^\dagger
        \ket 0
    }^\dagger
    \sbr{
        b_1^\dagger
        b_2^\dagger
        \ldots
        b_{j-1}^\dagger
        b_{j+1}^\dagger
        \ldots
        b_N^\dagger
        \ket 0
    }
    \intertext{%
        You can see that almost all creation operators are present, except for
        the ones with $i$ and $j$, respectively. The states that we will get
        from the vacuum are then the following:
    }
    &= [-1]^{i + j} \braket{1, 1, \ldots, 1, \underset i0, 1, \ldots, \underset
    N1, 0, \ldots | 1, 1, \ldots, 1, \underset j0, 1, \ldots, \underset N1, 0,
    \ldots}.
    \intertext{%
        We have marked the index of the unoccupied states. This scalar product
        is only nonzero when $i = j$ are given. So we can write this as
    }
    &= [-1]^{i + j} \delta_{ij}.
    \intertext{%
        And then we just have
    }
    &= \delta_{ij}
\end{align*}
because in the case of $i = j$, we have an even power of the $-1$, such that
does not contribute at all.

\subsubsection{Second way}

This way uses the hint on the problem set. Basically, in the first way we have
derived this hint for the special case that all the lower states are occupied.
So let us use that hint in the way given on the problem set. We let
$b_i^\dagger$ act to the left, $b_j$ to the right.
\[
    \braket{\psi|b_i^\dagger b_j|\psi}
    = n_i n_j [-1]^{\sum_{k<i} n_k + \sum_{k<j} n_k} \braket{1, 1, \ldots, 1,
    \underset i0, 1, \ldots, \underset N1, 0, \ldots | 1, 1, \ldots, 1, \underset
    j0, 1, \ldots, \underset N1, 0, \ldots}
\]
The sums are actually pretty straightforward, they are $i-1$ and $j-1$,
just like above. So this comes to the desired result as well.

\subsection{}

We just insert the result from the previous problem. The trick here is that $i$
and $j$
must be smaller than $N$ as well, because only states up to $N$ are occupied in
the state $\ket \psi$. So we get
\begin{align*}
    \sum_{i = 1}^\infty
    \sum_{j = 1}^\infty
    \braket{i | O | j}
    \braket{\psi|b_i^\dagger b_j|\psi}
    &=
    \sum_{i = 1}^\infty
    \sum_{j = 1}^\infty
    \braket{i | O | j}
    \delta_{ij}
    \Theta(i - N)
    \Theta(j - N).
    \intertext{%
        Then this is rather straightforward. But we will include the steps in
        between because this problem is worth two points.
    }
    &=
    \sum_{i = 1}^\infty
    \braket{i | O | i}
    \Theta(i - N)
    \Theta(i - N)
    \intertext{%
        The Heaviside step function squared it just the function itself.
    }
    &=
    \sum_{i = 1}^\infty
    \braket{i | O | i}
    \Theta(i - N)
    \intertext{%
        Now we restrict the summation according to the Heaviside step function.
    }
    &=
    \sum_{i = 1}^N
    \braket{i | O | i}
\end{align*}

\subsection{}

Given our in-depth coverage of the second part of this problem, we will do this
one really short. Look at
\[
    \braket{\psi|b_i^\dagger b_j^\dagger b_l b_k|\psi}.
\]
If we assume $j = l$ and then $i = k$, this scalar product will give us 1
because the operators will become occupation number operators which are
eigenoperators of state $\ket \psi$. So this is the $\delta_{ik} \delta_{jl}$
term. However, we can do one anticommutation as well. We anticommute the two
creation operators and get an additional minus sign:
\[
    = - \braket{\psi|b_j^\dagger b_i^\dagger b_l b_k|\psi}.
\]
With the same argumentation as above, we get a $- \delta_{il} \delta_{jk}$.
Together we have
\[
    \delta_{ik} \delta_{jl} - \delta_{il} \delta_{jk}
\]
which is the desired result.

\subsection{}

The equations are rather long here, even the result is on two lines already! We
could just go landscape and plow through it like fearless theoreticians or take
the more practical route and look at each summand in the Hamiltonian
separately. Since we are at \thepage\ pages already, we choose the latter way.

First we will look at the kinetic energy term. We write the matrix elements of
the kinetic operator as such to avoid nesting brakets.
\begin{align*}
    \Braket{\psi| \sum_{i,j} b_i^\dagger b_j T_{ij} | \psi}
    &= \sum_{i,j} \braket{\psi| b_i^\dagger b_j T_{ij} | \psi}
    \intertext{%
        Because of the creation and annihilation operators, we need $i = j$
        because of the reasoning we had in the previous few problems. This
        restricts the sum.
    }
    &= \sum_{i=1}^N \braket{\psi| b_i^\dagger b_j T_{ii} | \psi}
    \intertext{%
        Now we insert two complete sets of eigenstates.
    }
    &= \sum_{i=1}^N \int \dif^3 x \, \dif^3 y \,
    \braket{\psi|\vec x} \braket{\vec x| T_{ii} |\vec y} \braket{\vec y|\psi}
    \intertext{%
        Those scalar products are the field operators
        \parencite[(1.5.3)]{Schwabl/Quantenmechanik_fuer_Fortgeschrittene}.
    }
    &= \sum_{i=1}^N \int \dif^3 x \, \dif^3 y \,
    \phi^*(\vec x) \braket{\vec x| T_{ii} |\vec y} \phi(\vec y)
    \intertext{%
        We write out the matrix element.
    }
    &= \sum_{i=1}^N \int \dif^3 x \, \dif^3 y \, \dif^3 z \,
    \phi^*(\vec x) \delta^{(3)}(\vec z - \vec x) T_{ii}(\vec z) \delta^{(3)}(\vec z - \vec y) \phi(\vec y)
    \intertext{%
        The $\vec z$ integration will remove one $\delta$ distribution, the
        $\vec y$ integral the other one such that the only variable left is
        $\vec x$.
    }
    &= \sum_{i=1}^N \int \dif^3 x \,
    \phi^*(\vec x) T_{ii}(\vec x) \phi(\vec x)
    \intertext{%
        As a last step we insert the kinetic energy operator.
    }
    &= - \frac{\hbar^2}{2m} \sum_{i=1}^N \int \dif^3 x \,
    \phi^*(\vec x) \laplace \phi(\vec x)
\end{align*}

That was fun, let us do it again. The exact same steps can be done for the
scalar potential $U$. So we will have this:
\begin{align*}
    \Braket{\psi| \sum_{i,j} b_i^\dagger b_j U_{ij} | \psi}
    &= \sum_{i,j} \int \dif^3 x \,
    \phi^*(\vec x) U(\vec x) \phi(\vec x)
    \intertext{%
        The new thing here is that $U$ is just a regular function, not an
        operator. Therefore it commutes with the field operators and we get a
        modulus squared.
    }
    &= \sum_{i,j} \int \dif^3 x \, U(\vec x) |\phi(\vec x)|^2
\end{align*}

Now the stage is set for the last part, the interaction between different
electrons. $H_{\restriction V}$ is supposed to be a shorthand for the summand
of the Hamiltonian which contains $V$. It is merely introduced for alignment
purposes. This time we did not avoid the nested braket.
\begin{align*}
    \braket{\psi| H_{\restriction V} | \psi}
    &= \bra \psi
    \frac 12 \sum_{i,j} \braket{i,j|V|k,l} b_i^\dagger b_j^\dagger b_l b_k
    \ket \psi
    \intertext{%
        We pull out the sum.
    }
    &= \frac 12 \sum_{i,j} \bra \psi
    \braket{i,j|V|k,l} b_i^\dagger b_j^\dagger b_l b_k
    \ket \psi
\end{align*}

\end{document}

% vim: spell spelllang=en tw=79
