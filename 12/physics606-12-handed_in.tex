\documentclass[11pt, english, fleqn, DIV=15, headinclude, BCOR=1.5cm]{scrartcl}

\usepackage[
    bibatend,
    color,
]{../header}

\usepackage{tikz}
\usepackage{pdflscape}

\usepackage[tikz]{mdframed}
\newmdtheoremenv[%
    backgroundcolor=black!5,
    innertopmargin=\topskip,
    splittopskip=\topskip,
]{theorem}{Theorem}[section]

\hypersetup{
    pdftitle=
}

\newcounter{totalpoints}
\newcommand\punkte[1]{#1\addtocounter{totalpoints}{#1}}

\newcounter{problemset}
\setcounter{problemset}{12}

\subject{physics606 -- Advanced Quantum Theory}
\ihead{physics606 -- Problem Set \arabic{problemset}}

\title{Problem Set \arabic{problemset}}

\publishers{Group 2 -- Dilege Gülmez}
\ofoot{Group 2 -- Dilege Gülmez}

\newmdenv[%
    backgroundcolor=black!0,
    frametitlebackgroundcolor=black!0,
    roundcorner=5pt,
    skipabove=\topskip,
    innertopmargin=\topskip,
    splittopskip=\topskip,
    frametitle={Problem statement},
    frametitlerule=true,
    nobreak=true,
]{problem}

\newmdenv[%
    backgroundcolor=white,
    frametitlebackgroundcolor=black!0,
    roundcorner=5pt,
    skipabove=\topskip,
    innertopmargin=\topskip,
    innerbottommargin=8cm,
    splittopskip=\topskip,
    frametitle={Side question},
    frametitlerule=true,
]{question}

\newcommand\an{^\text{angular}}
\newcommand\ra{^\text{radial}}


\author{
    Martin Ueding \\ \small{\href{mailto:mu@martin-ueding.de}{mu@martin-ueding.de}}
    \and
    Lino Lemmer \\ \small{\href{mailto:l2@uni-bonn.de}{l2@uni-bonn.de}}
}
\ifoot{Martin Ueding, Lino Lemmer}

\ohead{\rightmark}

\begin{document}

\maketitle

\vspace{3ex}

\begin{center}
    \begin{tabular}{rrr}
        problem number & achieved points & possible points \\
        \midrule
        1 & & \punkte{7} \\
        2 & & \punkte{14} \\
        \midrule
        total & & \arabic{totalpoints}
    \end{tabular}
\end{center}

\section{Two-particle operators in second quantization}

\subsection{}

Let $\mathbb P$ be the set of all particles such that $\alpha, \beta \in
\mathbb P$. The sum
\[
    \sum_{\alpha \neq \beta}
\]
means that it is to be summed over all $\alpha$ and $\beta$ but not those where
they are equal. We write this more formally as
\begin{align*}
    F &= \frac 12 \sum_{\alpha \in \mathbb P} \sum_{\beta \in \mathbb P
    \setminus \{\alpha\}} f(\vec x_\alpha, \vec x_\beta).
    \intertext{%
        Now we insert sets of ones.
    }
    &= \frac 12 \sum_{\alpha \in \mathbb P}
    \sum_{\beta \in \mathbb P \setminus \{\alpha\}}
    \sum_{i,j,k,l}
    \ket i_\alpha \bra i_\alpha \ket j_\beta \bra j_\beta
    f(\vec x_\alpha, \vec x_\beta)
    \ket k_\alpha \bra k_\alpha \ket l_\beta \bra l_\beta
    \intertext{%
        We commute the bra and ket vectors of different particles which we can
        do since we consider the product Hilbert space of two different
        particles.
    }
    &= \frac 12 \sum_{\alpha \in \mathbb P}
    \sum_{\beta \in \mathbb P \setminus \{\alpha\}}
    \sum_{i,j,k,l}
    \ket i_\alpha \ket j_\beta \bra i_\alpha \bra j_\beta
    f(\vec x_\alpha, \vec x_\beta)
    \ket k_\alpha \ket l_\beta \bra k_\alpha \bra l_\beta
    \intertext{%
        We can then use the notation which corresponds to the tensor product
        Hilbert space of two particles.
    }
    &= \frac 12 \sum_{\alpha \in \mathbb P}
    \sum_{\beta \in \mathbb P \setminus \{\alpha\}}
    \sum_{i,j,k,l}
    \ket i_\alpha \ket j_\beta \bra{i,j}
    f(\vec x_\alpha, \vec x_\beta)
    \ket{k,l} \bra k_\alpha \bra l_\beta
    \intertext{%
        The term in the middle is a scalar and can be moved anywhere in the
        product.
    }
    &= \frac 12 \sum_{\alpha \in \mathbb P}
    \sum_{\beta \in \mathbb P \setminus \{\alpha\}}
    \sum_{i,j,k,l}
    \bra{i,j} f(\vec x_\alpha, \vec x_\beta) \ket{k,l}
    \ket i_\alpha \ket j_\beta
    \bra k_\alpha \bra l_\beta
\end{align*}
This is the desired result in equation~(2) of the problem set.

\subsection{}

We start with equation~(2) from the problem set.
\begin{align*}
    F &= \frac 12 \sum_{\alpha \in \mathbb P}
    \sum_{\beta \in \mathbb P \setminus \{\alpha\}}
    \sum_{i,j,k,l}
    \bra{i,j} f(\vec x_\alpha, \vec x_\beta) \ket{k,l}
    \ket i_\alpha \ket j_\beta
    \bra k_\alpha \bra l_\beta
    \intertext{%
        Then we change the order of the bra and ket vectors at the right side
        of the equation.
    }
    &= \frac 12
    \sum_{\alpha \in \mathbb P}
    \sum_{\beta \in \mathbb P \setminus \{\alpha\}}
    \sum_{i,j,k,l}
    \bra{i,j} f(\vec x_\alpha, \vec x_\beta) \ket{k,l}
    \ket i_\alpha
    \bra k_\alpha
    \ket j_\beta
    \bra l_\beta
    \intertext{%
        We change the order of the summations such that we can plug in
        equation~(5) from the problem set later on. The underbraces denote the
        outcome suggestively, we still have to work on that, though.
    }
    &= \frac 12
    \sum_{i,j,k,l}
    \bra{i,j} f(\vec x_\alpha, \vec x_\beta) \ket{k,l}
    \underbrace{
        \sum_{\alpha \in \mathbb P}
        \ket i_\alpha
        \bra k_\alpha
    }_{\leadsto b^\dagger_i b_k}
    \underbrace{
        \sum_{\beta \in \mathbb P \setminus \{\alpha\}}
        \ket j_\beta
        \bra l_\beta
    }_{\leadsto b^\dagger_j b_l}
    \intertext{%
        The problem is that the sum over $\beta$ excludes $\alpha$. Therefore,
        the sums are not independent and we cannot apply equation~(5) yet. So
        we just sum over all $\beta \in \mathbb P$ in the sum and subtract that
        particular value again.
    }
    &= \frac 12
    \sum_{i,j,k,l}
    \bra{i,j} f(\vec x_\alpha, \vec x_\beta) \ket{k,l}
    \sum_{\alpha \in \mathbb P}
    \ket i_\alpha
    \bra k_\alpha
    \sbr{
        \sum_{\beta \in \mathbb P}
        \ket j_\beta
        \bra l_\beta
        -
        \ket j_\alpha
        \bra l_\alpha
    }
    \intertext{%
        The one summation over $\beta$ is now independent of $\alpha$ and we
        can use equation~(5).
    }
    &= \frac 12
    \sum_{i,j,k,l}
    \bra{i,j} f(\vec x_\alpha, \vec x_\beta) \ket{k,l}
    \sum_{\alpha \in \mathbb P}
    \ket i_\alpha
    \bra k_\alpha
    \sbr{
        b^\dagger_j b_l
        -
        \ket j_\alpha
        \bra l_\alpha
    }
    \intertext{%
        Now we expand the parentheses.
    }
    &= \frac 12
    \sum_{i,j,k,l}
    \bra{i,j} f(\vec x_\alpha, \vec x_\beta) \ket{k,l}
    \sbr{
        b^\dagger_j b_l
        \sum_{\alpha \in \mathbb P}
        \ket i_\alpha
        \bra k_\alpha
        -
        \sum_{\alpha \in \mathbb P}
        \ket i_\alpha
        \bra k_\alpha
        \ket j_\alpha
        \bra l_\alpha
    }
    \intertext{%
        We apply equation~(5) once again.
    }
    &= \frac 12
    \sum_{i,j,k,l}
    \bra{i,j} f(\vec x_\alpha, \vec x_\beta) \ket{k,l}
    \sbr{
        b^\dagger_i b_k
        b^\dagger_j b_l
        -
        \sum_{\alpha \in \mathbb P}
        \ket i_\alpha
        \bra k_\alpha
        \ket j_\alpha
        \bra l_\alpha
    }
    \intertext{%
        This looks somewhat like equation~(4) from the problem set which we
        should arrive at. The terms are not in the correct order yet, though.
        The bra-ket in the middle is just a Kronecker $\delta$.
    }
    &= \frac 12
    \sum_{i,j,k,l}
    \bra{i,j} f(\vec x_\alpha, \vec x_\beta) \ket{k,l}
    \sbr{
        b^\dagger_i b_k
        b^\dagger_j b_l
        -
        \sum_{\alpha \in \mathbb P}
        \ket i_\alpha
        \delta_{kj}
        \bra l_\alpha
    }
    \intertext{%
        However, the Kronecker $\delta$ is also contained in the commutator of
        the creation and annihilation operators: $\delta_{kj} = [b_k,
        b^\dagger_j]_\mp$
        \parencite[16]{Schwabl/Quantenmechanik_fuer_Fortgeschrittene}. Since it
        is a plain number, we pull it out front for a step.
    }
    &= \frac 12
    \sum_{i,j,k,l}
    \bra{i,j} f(\vec x_\alpha, \vec x_\beta) \ket{k,l}
    \sbr{
        b^\dagger_i b_k
        b^\dagger_j b_l
        -
        \delta_{kj}
        \sum_{\alpha \in \mathbb P}
        \ket i_\alpha
        \bra l_\alpha
    }
    \intertext{%
        Now we apply equation~(5) from the problem set again.
    }
    &= \frac 12
    \sum_{i,j,k,l}
    \bra{i,j} f(\vec x_\alpha, \vec x_\beta) \ket{k,l}
    \sbr{
        b^\dagger_i b_k
        b^\dagger_j b_l
        -
        \delta_{kj}
        b^\dagger_i b_l
    }
    \intertext{%
        Now we move the Kronecker $\delta$ in between those terms.
    }
    &= \frac 12
    \sum_{i,j,k,l}
    \bra{i,j} f(\vec x_\alpha, \vec x_\beta) \ket{k,l}
    \sbr{
        b^\dagger_i b_k
        b^\dagger_j b_l
        -
        b^\dagger_i
        \delta_{kj}
        b_l
    }
    \intertext{%
        We expand the (anti-)commutator there.
    }
    &= \frac 12
    \sum_{i,j,k,l}
    \bra{i,j} f(\vec x_\alpha, \vec x_\beta) \ket{k,l}
    \sbr{
        b^\dagger_i b_k
        b^\dagger_j b_l
        -
        b^\dagger_i
        \sbr{ b_k b^\dagger_j \mp b_k b^\dagger_j }
        b_l
    }
    \intertext{%
        Then we expand the inner square bracket.
    }
    &= \frac 12
    \sum_{i,j,k,l}
    \bra{i,j} f(\vec x_\alpha, \vec x_\beta) \ket{k,l}
    \sbr{
        b^\dagger_i b_k b^\dagger_j b_l
        -
        b^\dagger_i b_k b^\dagger_j b_l
        \pm
        b^\dagger_i b^\dagger_j b_k b_l
    }
    \intertext{%
        The first two terms cancel each other, so we are left with
    }
    &= \pm \frac 12
    \sum_{i,j,k,l}
    \bra{i,j} f(\vec x_\alpha, \vec x_\beta) \ket{k,l}
    b^\dagger_i b^\dagger_j b_k b_l.
\end{align*}
The version on the problem set has $k$ and $l$ switched. Since the annihilation
operators always commute, those two versions are equivalent.

\section{Hartree-Fock approximation for atoms}

\subsection{}

\subsection{}

\subsection{}

\subsection{}


\end{document}

% vim: spell spelllang=en tw=79
