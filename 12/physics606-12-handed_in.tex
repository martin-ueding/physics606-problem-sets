\documentclass[11pt, english, fleqn, DIV=15, headinclude, BCOR=1.5cm]{scrartcl}

\usepackage[
    bibatend,
    color,
]{../header}

\usepackage{tikz}
\usepackage{pdflscape}

\usepackage[tikz]{mdframed}
\newmdtheoremenv[%
    backgroundcolor=black!5,
    innertopmargin=\topskip,
    splittopskip=\topskip,
]{theorem}{Theorem}[section]

\hypersetup{
    pdftitle=
}

\newcounter{totalpoints}
\newcommand\punkte[1]{#1\addtocounter{totalpoints}{#1}}

\newcounter{problemset}
\setcounter{problemset}{12}

\subject{physics606 -- Advanced Quantum Theory}
\ihead{physics606 -- Problem Set \arabic{problemset}}

\title{Problem Set \arabic{problemset}}

\publishers{Group 2 -- Dilege Gülmez}
\ofoot{Group 2 -- Dilege Gülmez}

\newmdenv[%
    backgroundcolor=black!0,
    frametitlebackgroundcolor=black!0,
    roundcorner=5pt,
    skipabove=\topskip,
    innertopmargin=\topskip,
    splittopskip=\topskip,
    frametitle={Problem statement},
    frametitlerule=true,
    nobreak=true,
]{problem}

\newmdenv[%
    backgroundcolor=white,
    frametitlebackgroundcolor=black!0,
    roundcorner=5pt,
    skipabove=\topskip,
    innertopmargin=\topskip,
    innerbottommargin=8cm,
    splittopskip=\topskip,
    frametitle={Side question},
    frametitlerule=true,
]{question}

\newcommand\an{^\text{angular}}
\newcommand\ra{^\text{radial}}


\author{
    Martin Ueding \\ \small{\href{mailto:mu@martin-ueding.de}{mu@martin-ueding.de}}
    \and
    Lino Lemmer \\ \small{\href{mailto:l2@uni-bonn.de}{l2@uni-bonn.de}}
}
\ifoot{Martin Ueding, Lino Lemmer}

\ohead{\rightmark}

\begin{document}

\maketitle

\vspace{3ex}

\begin{center}
    \begin{tabular}{rrr}
        problem number & achieved points & possible points \\
        \midrule
        1 & & \punkte{7} \\
        2 & & \punkte{14} \\
        \midrule
        total & & \arabic{totalpoints}
    \end{tabular}
\end{center}

\section{Two-particle operators in second quantization}

\subsection{}

Let $\mathbb P$ be the set of all particles such that $\alpha, \beta \in
\mathbb P$. The sum
\[
    \sum_{\alpha \neq \beta}
\]
means that it is to be summed over all $\alpha$ and $\beta$ but not those where
they are equal. We write this more formally as
\begin{align*}
    F &= \frac 12 \sum_{\alpha \in \mathbb P} \sum_{\beta \in \mathbb P
    \setminus \{\alpha\}} f(\vec x_\alpha, \vec x_\beta).
    \intertext{%
        Now we insert sets of ones.
    }
    &= \frac 12 \sum_{\alpha \in \mathbb P}
    \sum_{\beta \in \mathbb P \setminus \{\alpha\}}
    \sum_{i,j,k,l}
    \ket i_\alpha \bra i_\alpha \ket j_\beta \bra j_\beta
    f(\vec x_\alpha, \vec x_\beta)
    \ket k_\alpha \bra k_\alpha \ket l_\beta \bra l_\beta
    \intertext{%
        We commute the bra and ket vectors of different particles which we can
        do since we consider the product Hilbert space of two different
        particles.
    }
    &= \frac 12 \sum_{\alpha \in \mathbb P}
    \sum_{\beta \in \mathbb P \setminus \{\alpha\}}
    \sum_{i,j,k,l}
    \ket i_\alpha \ket j_\beta \bra i_\alpha \bra j_\beta
    f(\vec x_\alpha, \vec x_\beta)
    \ket k_\alpha \ket l_\beta \bra k_\alpha \bra l_\beta
    \intertext{%
        We can then use the notation which corresponds to the tensor product
        Hilbert space of two particles.
    }
    &= \frac 12 \sum_{\alpha \in \mathbb P}
    \sum_{\beta \in \mathbb P \setminus \{\alpha\}}
    \sum_{i,j,k,l}
    \ket i_\alpha \ket j_\beta \bra{i,j}
    f(\vec x_\alpha, \vec x_\beta)
    \ket{k,l} \bra k_\alpha \bra l_\beta
    \intertext{%
        The term in the middle is a scalar and can be moved anywhere in the
        product.
    }
    &= \frac 12 \sum_{\alpha \in \mathbb P}
    \sum_{\beta \in \mathbb P \setminus \{\alpha\}}
    \sum_{i,j,k,l}
    \bra{i,j} f(\vec x_\alpha, \vec x_\beta) \ket{k,l}
    \ket i_\alpha \ket j_\beta
    \bra k_\alpha \bra l_\beta
\end{align*}
This is the desired result in equation~(2) of the problem set.

\section{Hartree-Fock approximation for atoms}


\end{document}

% vim: spell spelllang=en tw=79
