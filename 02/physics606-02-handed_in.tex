\documentclass[11pt, ngerman, fleqn, DIV=15, headinclude, BCOR=1cm]{scrartcl}

\usepackage[bibatend]{../header}

\usepackage{tikz}

\usepackage[tikz]{mdframed}
\newmdtheoremenv[%
    backgroundcolor=black!5,
    innertopmargin=\topskip,
    splittopskip=\topskip,
]{theorem}{Theorem}[section]

\hypersetup{
    pdftitle=
}

\newcounter{totalpoints}
\newcommand\punkte[1]{#1\addtocounter{totalpoints}{#1}}

\newcounter{problemset}
\setcounter{problemset}{2}

\subject{physics606 -- Advanced Quantum Theory}
\ihead{physics606 -- Problem Set \arabic{problemset}}

\title{Problem Set \arabic{problemset}}

\publishers{Group 2 -- Dilege Gülmez}
\ofoot{Group 2 -- Dilege Gülmez}



\author{
    Martin Ueding \\ \small{\href{mailto:mu@martin-ueding.de}{mu@martin-ueding.de}}
    \and
    Lino Lemmer \\ \small{\href{mailto:l2@uni-bonn.de}{l2@uni-bonn.de}}
}
\ifoot{Martin Ueding, Lino Lemmer}

\ohead{\rightmark}

\begin{document}

\maketitle

\vspace{3ex}

\begin{center}
    \begin{tabular}{rrr}
        problem number & achieved points & possible points \\
        \midrule
        1 & & \punkte{7} \\
        2 & & \punkte{9} \\
        3 & & \punkte{12} \\
        \midrule
        Total & & \arabic{totalpoints}
    \end{tabular}
\end{center}

\section{Canonical Transformations and Classical Trajectories}

\subsection{}

Let $(q_i(t), p_i(t))$ be a solution to the equations of motion. Then for $q_i$
we have
\begin{align*}
    \dot q_i(t) &= \cbr{q_i(t), H}. \\
    \intertext{%
        We add the same to both sides which will not make any difference:
    }
    \iff \dot q_i(t) + \epsilon \cbr{\dpd{g}{p_i}, H} &= \cbr{q_i(t), H} + \epsilon \cbr{\dpd{g}{p_i}, H}. \\
    \intertext{%
        On the left hand side we expand the Poisson bracket into its
        definition.
    }
    \iff \dot q_i(t) + \epsilon \sum_j \sbr{\dmd{g}{2}{q_j}{}{p_i}{}
        \dpd{H}{p_j}
    - \dmd{g}{2}{p_j}{}{p_i}{} \dpd{H}{q_j}} &= \cbr{q_i(t) + \epsilon \dpd{g}{p_i}, H}. \\
    \intertext{%
        The equations of motion allow us to eliminate $H$ on the left hand
        side.
    }
    \iff \dot q_i(t) + \epsilon \sum_j \sbr{\dmd{g}{2}{q_j}{}{p_i}{} \dot q_j +
    \dmd{g}{2}{p_j}{}{p_i}{} \dot p_j} &= \cbr{q_i(t) + \epsilon \dpd{g}{p_i}, H}. \\
    \intertext{%
        The term in the square bracket is
        \[
            \dod{}t \dpd{}{p_i} g\del{q_1(t), \ldots, q_n(t), p_1(t), \ldots, p_n(t)}.
        \]
        Therefore we can write this as
    }
    \iff \dot q_i(t) + \epsilon \dod{}t \dpd{g}{p_i} &= \cbr{q_i(t) + \epsilon \dpd{g}{p_i}, H}. \\
    \intertext{%
        We factor out the time derivative and yield
    }
    \iff \dod{}t \sbr{q_i(t) + \epsilon \dpd{g}{p_i}} &= \cbr{q_i(t) + \epsilon \dpd{g}{p_i}, H}. \\
    \intertext{%
        Inserting the definition of the transformation gives us
    }
    \iff \dod{}t \bar q_i(t) &= \cbr{\bar q_i(t), H}.
\end{align*}

That is the equation of motion for $\bar q_i$. The same thing can be done with
$\bar p_i$. Therefore, these infinitesimal transformations generate more valid
trajectories.

\subsection{}

\section{Canonical Transformations in Quantum Mechanics}

\section{Gauge Invariance in Classical Electrodynamics}


\end{document}

% vim: spell spelllang=en tw=79
