\documentclass[11pt, english, fleqn, DIV=15, headinclude, BCOR=1.5cm]{scrartcl}

\usepackage[
    bibatend,
    color,
]{../header}

\usepackage{tikz}
\usepackage{pdflscape}

\usepackage[tikz]{mdframed}
\newmdtheoremenv[%
    backgroundcolor=black!5,
    innertopmargin=\topskip,
    splittopskip=\topskip,
]{theorem}{Theorem}[section]

\hypersetup{
    pdftitle=
}

\newcounter{totalpoints}
\newcommand\punkte[1]{#1\addtocounter{totalpoints}{#1}}

\newcounter{problemset}
\setcounter{problemset}{5}

\subject{physics606 -- Advanced Quantum Theory}
\ihead{physics606 -- Problem Set \arabic{problemset}}

\title{Problem Set \arabic{problemset}}

\publishers{Group 2 -- Dilege Gülmez}
\ofoot{Group 2 -- Dilege Gülmez}

\newmdenv[%
    backgroundcolor=black!5,
    frametitlebackgroundcolor=black!10,
    roundcorner=5pt,
    skipabove=\topskip,
    innertopmargin=\topskip,
    splittopskip=\topskip,
    frametitle={Problem statement},
    frametitlerule=true,
    nobreak=true,
]{problem}

\newmdenv[%
    backgroundcolor=white,
    frametitlebackgroundcolor=black!10,
    roundcorner=5pt,
    skipabove=\topskip,
    innertopmargin=\topskip,
    innerbottommargin=8cm,
    splittopskip=\topskip,
    frametitle={Side question},
    frametitlerule=true,
]{question}


\author{
    Martin Ueding \\ \small{\href{mailto:mu@martin-ueding.de}{mu@martin-ueding.de}}
    \and
    Lino Lemmer \\ \small{\href{mailto:l2@uni-bonn.de}{l2@uni-bonn.de}}
}
\ifoot{Martin Ueding, Lino Lemmer}

\ohead{\rightmark}

\begin{document}

\maketitle

\vspace{3ex}

\begin{center}
    \begin{tabular}{rrr}
        problem number & achieved points & possible points \\
        \midrule
        1 & & \punkte{14} \\
        2 & & \punkte{7} \\
        3 & & \punkte{15} \\
        \midrule
        total & & \arabic{totalpoints}
    \end{tabular}
\end{center}

\section{Transitions between general states}

\newcommand\IP{{}_\text I}
\newcommand\SP{{}_\text S}
\newcommand\0{{}^{(0)}}

\subsection{Equivalence of pictures}

\begin{problem}
    Show that $H_0$ is the same in the Schrödinger and interaction pictures.
\end{problem}

In the interaction picture, the operators are defined as
\[
    O\IP(t) = U_0^\dagger(t, t_0) O\SP(t) U_0(t, t_0)
\]
because the states are defined as
\[
    \ket{\psi, t}\IP = U_0^\dagger(t, t_0) \ket{\psi, t}\SP.
\]

In the short form without functional dependence, we have:
\begin{align*}
    H\IP
    &= U_0^\dagger H_0\SP U_0 \\
    \intertext{%
        $U_0$ and $H_0$ commute since $U_0$ is a smooth function of $H_0$ only.
    }
    &= U_0^\dagger U_0 H_0\SP
    \intertext{%
        Now $U_0U_0^\dagger = 1$. We have
    }
    &= H_0\SP.
\end{align*}
So $H\IP = H\SP$.

\subsection{Orthogonality and coefficients}

\begin{problem}
    In order to make contact with the results derived in class, express the
    states $\ket\alpha$ and $\ket\beta$ as linear superpositions of eigenstates
    $\ket{n\0}$ of $H_0$.
\end{problem}

We call those coefficients $c$ and $d$:
\[
    \ket\alpha = \sum_i c_i \ket{i\0}
    \eqnsep
    \ket\beta = \sum_i d_i \ket{i\0}.
\]

Since they are eigenstates to the operator $O$, which seems to be time
independent, the states do not depend on the time, just like the eigenstates of
the Hamiltonian do not either.

The overlap between the two states is
\begin{align*}
    \braket{\alpha|\beta}
    &= \sum_i \sum_j c_i^* d_j \braket{i\0 | j\0} \\
    &= \sum_i \sum_j c_i^* d_j \delta_{ij} \\
    &= \sum_i c_i^* d_i.
\end{align*}
The desired orthogonality of the states $\ket\alpha$ and $\ket\beta$ requires
\[
    \sum_i c_i^* d_i = 0
\]
since the $c$ and $d$ are considered to belong to two different states.

\subsection{Schrödinger picture}

\begin{problem}
    Show that the probability that $\ket{\psi, t}\SP = \ket\beta$ is in general
    non-zero for $t > t_0$ even if $H_1 = 0$ if $[H_0, O\SP] \neq 0$.
\end{problem}

The state $\ket{\psi, t}\SP$ is not a basis vector in the energy basis (the
eigenbasis of $H_0$). Since $[H_0, O\SP] \neq 0$, $O\SP$ does not need to be
diagonal in the general case. (Only commuting operators have the same
eigenfunctions and therefore the same eigenbasis and can be diagonalized by the
same basis transformation.) Since $\ket{\psi, t}\SP$ is not an eigenvector of
$H_0$, the time evolution---that is generated by $H_0$---will map
this to a different vector, which might have an overlap integral with
$\ket\beta$.

This can also be seen in the calculation. The probability is given by:
\begin{align*}
    \abs{\braket{\beta | \psi\SP(t) }}^2
    &= \abs{\braket{\beta | U_0(t, t_0) | \alpha }}^2 \\
    \intertext{%
        Expand into coefficients.
    }
    &= \abs{ \sum_i \sum_j d^*_i c_j \braket{i\0 | U_0(t, t_0) | j\0 }}^2 \\
    \intertext{%
        Insert $U$. It is important to note that $H_0[t-t_0]$ is a product of
        $H_0$ and $t - t_0$, not a function evaluation. I use square brackets
        for grouping of this sort for this exact reason, and only round
        parentheses for function application.
    }
    &= \abs{ \sum_i \sum_j d^*_i c_j \Braket{i\0 | \exp\del{-\frac\iup\hbar
    H_0[t-t_0]} | j\0 }}^2 \\
    \intertext{%
        Apply $H$ to the state $\ket{j\0}$. If this is unclear, expand the
        exponential function, apply each summand individually and wrap it up in
        an exponential again.
    }
    &= \abs{ \sum_i \sum_j d^*_i c_j \Braket{i\0 | \exp\del{-\frac\iup\hbar
    E_j[t-t_0]} | j\0 }}^2 \\
    \intertext{%
        We can pull this up front now.
    }
    &= \abs{ \sum_i \sum_j d^*_i c_j \exp\del{-\frac\iup\hbar
    E_j[t-t_0]} \braket{i\0 | j\0 }}^2 \\
    \intertext{%
        That is a $\delta_{ij}$ in the back.
    }
    &= \abs{ \sum_i d^*_i c_i \exp\del{-\frac\iup\hbar
    E_i[t-t_0]}}^2
\end{align*}

For $t = t_0$, this reproduces the previous equation, and we concluded that
this had to be $0$ since $\delta_{\alpha\beta} = 0$ for $\alpha \neq \beta$.
For $t > t_0$ the exponential will change the summands in a way that they do
not cancel out together, giving a finite possibility.

\subsection{Interaction picture}

Here, the probability is
\begin{align*}
    \abs{\braket{\beta | \psi\IP(t)}}^2
    &= \abs{ \sum_i d_i^* \braket{i\0 | \psi\IP(t)}}^2. \\
    \intertext{%
        Since the interaction picture is defined such that the action of $U_0$
        is reversed on the states, the states here are actually time
        independent now:
    }
    &= \abs{ \sum_i d_i^* \braket{i\0 | \alpha}}^2. \\
    \intertext{%
        We expand again.
    }
    &= \abs{ \sum_i \sum_j d_i^* c_j \braket{i\0 | j\0}}^2 \\
    \intertext{%
        Same as in the above problem.
    }
    &= \abs{ \sum_i d_i^* c_i}^2
    \intertext{%
        For $\alpha \neq \beta$, this is
    }
    &= 0.
\end{align*}
That is different than the result of the previous part.

\begin{problem}
    The discrepancy to the result of the previous subsection clearly shows that
    the wave function, or state vector, by itself has no direct physical
    meaning, even though one of the axioms of quantum mechanics state that it
    contains all the information about the system that can be known!
\end{problem}

Well, $\ket{\psi, t}\IP \neq \ket{\psi, t}\SP$. There is a $U$ or $U^\dagger$
of difference, depending on your point of view. Since a unitary transformation
can be thought of a basis change, it is not too surprising that the scalar
product changes when one of the vectors is transformed only ($\ket\beta$ is
left unchanged). I'd be worried if it was the same when time evolution is taken
into account and disregarded.

\subsection{Physical meaning}

The expectation value of an operator in the Schrödinger picture is:
\begin{align*}
    \bracket{O\SP}
    &= \braket{\psi\SP(t) | O\SP | \psi\SP(t)} \\
    \intertext{%
        We write down the time evolution explicitly.
    }
    &= \braket{\alpha | U\SP^\dagger(t, t_0) O\SP U\SP(t, t_0)| \alpha} \\
    \intertext{%
        We insert complete set of eigenstates. We also use Greek letters here
        to distinguish the eigenstates of $O$ from the eigenstates of $H_0$.
    }
    &= \sum_\gamma \sum_\eta \braket{\alpha | U\SP^\dagger(t, t_0) | \gamma}
    \braket{\gamma | O\SP | \eta} \braket{\eta | U\SP(t, t_0)| \alpha} \\
    \intertext{%
        The middle part gives us $o_\eta \delta_{\gamma\eta}$.
    }
    &= \sum_\gamma o_\gamma \braket{\alpha | U\SP^\dagger(t, t_0) | \gamma}
    \braket{\gamma | U\SP(t, t_0)| \alpha} \\
    \intertext{%
        We prepare …
    }
    &= \sum_\gamma o_\gamma \braket{\gamma | U\SP(t, t_0) | \alpha}^*
    \braket{\gamma | U\SP(t, t_0)| \alpha} \\
    \intertext{%
        … and write this as the modulus squared
    }
    &= \sum_\gamma o_\gamma \abs{\braket{\gamma | U\SP(t, t_0)| \alpha}}^2 \\
    \intertext{%
        This could also be written as the probability that the system will go
        to the state $\ket\gamma$ on the measurement.
    }
    &= \sum_\gamma o_\gamma \abs{\braket{\gamma | \psi\SP(t)}}^2
\end{align*}

That is more or less one of the axioms of quantum mechanics, that the
measurement will turn out to be an eigenvalue of the operator (here $o_\gamma$)
and that the probabilities are the overlap integrals modulus squared.

Now we can re-complexify the probability part to match the version on
the problem set.
\begin{align*}
    P_{\beta\alpha}
    &= \abs{\braket{\beta | U\SP(t, t_0)| \alpha}}^2 \\
    \intertext{%
        We insert another set of eigenstates, now the eigenstates of $H_0$.
    }
    &= \abs{\sum_i \sum_f \braket{\beta | f\0} \braket{f\0 | U\SP(t, t_0)| i\0}
    \braket{i\0 | \alpha}}^2 \\
    \intertext{%
        We rearrange those scalar product, they are just scalars and commute.
    }
    &= \abs{\sum_i \sum_f \braket{i\0 | \alpha} \braket{\beta | f\0}
    \braket{f\0 | U\SP(t, t_0)| i\0} }^2 \\
    \intertext{%
        Using the given name for the matrix element, we can write this as
    }
    &= \abs{\sum_i \sum_f \braket{i\0 | \alpha} \braket{\beta | f\0}
    \tilde{\mathcal A}_{fi}(t, t_0) }^2.
\end{align*}

\subsection{Same result}

Physical reality should not change with a (passive) basis transformation. If
that would be the case, the different pictures would not be legitimate and not
taught at university. (Occam's razor.)

Mathematically, the expectation value is a scalar product where an operator is
pre- and postmultiplied by the same wavefunction. Unitary transformations of
all kinds cannot change this scalar product, since the basis of everything is
changed consistently such that the scalar product does not change. Hence the
definition of the operators the way we wrote in the very first part of this
problem.

\section{Selection rules}

\subsection{Forbidden transitions}

$[H_1, O] = 0$ means that both operators have the same set of eigenfunctions.
Since the state on the right has eigenvalue $o_i$ of the operator $O$, they are
eigenfunctions of $O$. Therefore, they are also eigenfunctions to $H_1$. $H_1$
cannot change the $o_i$ therefore, since it is diagonal in this eigenbasis of
$O$. So $\braket{o_f | H_1 | o_i} = \lambda \braket{o_f | o_i} = 0$, where
$\lambda$ is the unknown eigenvalue of $H_1$ to $\ket{o_i}$.

\subsection{All orders of perturbation theory}

“All orders” sounds like the Dyson series. There, the perturbation is
exponentiated. Since the perturbation itself is diagonal in the space spanned
by the eigenstates of $O$, there is no way (infinitely) repeated application
could get it off-diagonal. I do not see how our argument is limited to first
order.

\subsection{Angular momentum}

\begin{question}
    Is that really $[H_0, \vec L] = 0$? If so, it should be the null vector,
    like $[H_0, \vec L] = \vec 0$ in the first place. Then, this would imply
    that all the components of the angular momentum can be measured at the same
    time with arbitrary precision. This would only work if there is no
    potential. To us, $[H_0, \vec L^2] = 0$ looks more familiar. Is it really
    $[H_0, \vec L]$?
\end{question}

If it was $[H_0, \vec L^2] = 0$, then the eigenstates would be $\ket{n, l,
m_l}$. With a perturbation $H_1 = f(z, t)$ the commutator $[H_1, L_3]$ would be
zero, forcing $m_l$ to stay unchanged in transitions. The second perturbation
depends only on the radius and would $l$ and $m_l$ unchanged.

\section{Atomic radiative transitions}


\end{document}

% vim: spell spelllang=en tw=79
