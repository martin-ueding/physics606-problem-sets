\documentclass[11pt, english, fleqn, DIV=15, headinclude, BCOR=1.5cm]{scrartcl}

\usepackage[
    bibatend,
    color,
]{../header}

\usepackage{tikz}
\usepackage{pdflscape}

\usepackage[tikz]{mdframed}
\newmdtheoremenv[%
    backgroundcolor=black!5,
    innertopmargin=\topskip,
    splittopskip=\topskip,
]{theorem}{Theorem}[section]

\hypersetup{
    pdftitle=
}

\newcounter{totalpoints}
\newcommand\punkte[1]{#1\addtocounter{totalpoints}{#1}}

\newcounter{problemset}
\setcounter{problemset}{5}

\subject{physics606 -- Advanced Quantum Theory}
\ihead{physics606 -- Problem Set \arabic{problemset}}

\title{Problem Set \arabic{problemset}}

\publishers{Group 2 -- Dilege Gülmez}
\ofoot{Group 2 -- Dilege Gülmez}

\newmdenv[%
    backgroundcolor=black!5,
    frametitlebackgroundcolor=black!10,
    roundcorner=5pt,
    skipabove=\topskip,
    innertopmargin=\topskip,
    splittopskip=\topskip,
    frametitle={Problem statement},
    frametitlerule=true,
    nobreak=true,
]{problem}

\newmdenv[%
    backgroundcolor=white,
    frametitlebackgroundcolor=black!10,
    roundcorner=5pt,
    skipabove=\topskip,
    innertopmargin=\topskip,
    innerbottommargin=8cm,
    splittopskip=\topskip,
    frametitle={Side question},
    frametitlerule=true,
]{question}


\author{
    Martin Ueding \\ \small{\href{mailto:mu@martin-ueding.de}{mu@martin-ueding.de}}
    \and
    Lino Lemmer \\ \small{\href{mailto:l2@uni-bonn.de}{l2@uni-bonn.de}}
}
\ifoot{Martin Ueding, Lino Lemmer}

\ohead{\rightmark}

\begin{document}

\maketitle

\vspace{3ex}

\begin{center}
    \begin{tabular}{rrr}
        problem number & achieved points & possible points \\
        \midrule
        1 & & \punkte{14} \\
        2 & & \punkte{7} \\
        3 & & \punkte{15} \\
        \midrule
        total & & \arabic{totalpoints}
    \end{tabular}
\end{center}

\section{Transitions between general states}

\newcommand\IP{{}_\text I}
\newcommand\SP{{}_\text S}
\newcommand\0{{}^{(0)}}

\subsection{Equivalence of pictures}

\begin{problem}
    Show that $H_0$ is the same in the Schrödinger and interaction pictures.
\end{problem}

In the interaction picture, the operators are defined as
\[
    O\IP(t) = U_0^\dagger(t, t_0) O\SP(t) U_0(t, t_0)
\]
because the states are defined as
\[
    \ket{\psi, t}\IP = U_0^\dagger(t, t_0) \ket{\psi, t}\SP.
\]

In the short form without functional dependence, we have:
\begin{align*}
    H\IP
    &= U_0^\dagger H_0\SP U_0 \\
    \intertext{%
        $U_0$ and $H_0$ commute since $U_0$ is a smooth function of $H_0$ only.
    }
    &= U_0^\dagger U_0 H_0\SP
    \intertext{%
        Now $U_0U_0^\dagger = 1$. We have
    }
    &= H_0\SP.
\end{align*}
So $H\IP = H\SP$.

\subsection{Orthogonality and coefficients}

\begin{problem}
    In order to make contact with the results derived in class, express the
    states $\ket\alpha$ and $\ket\beta$ as linear superpositions of eigenstates
    $\ket{n\0}$ of $H_0$.
\end{problem}

We call those coefficients $c$ and $d$:
\[
    \ket\alpha = \sum_i c_i \ket{i\0}
    \eqnsep
    \ket\beta = \sum_i d_i \ket{i\0}.
\]

Since they are eigenstates to the operator $O$, which seems to be time
independent, the states do not depend on the time, just like the eigenstates of
the Hamiltonian do not either.

The overlap between the two states is
\begin{align*}
    \braket{\alpha|\beta}
    &= \sum_i \sum_j c_i^* d_j \braket{i\0 | j\0} \\
    &= \sum_i \sum_j c_i^* d_j \delta_{ij} \\
    &= \sum_i c_i^* d_i.
\end{align*}
The desired orthogonality of the states $\ket\alpha$ and $\ket\beta$ requires
\[
    \sum_i c_i^* d_i = 0
\]
since the $c$ and $d$ are considered to belong to two different states.

\section{Selection rules}

\section{Atomic radiative transitions}


\end{document}

% vim: spell spelllang=en tw=79
