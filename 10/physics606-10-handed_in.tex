\documentclass[11pt, english, fleqn, DIV=15, headinclude, BCOR=1.5cm]{scrartcl}

\usepackage[
    bibatend,
    color,
]{../header}

\usepackage{tikz}
\usepackage{pdflscape}

\usepackage[tikz]{mdframed}
\newmdtheoremenv[%
    backgroundcolor=black!5,
    innertopmargin=\topskip,
    splittopskip=\topskip,
]{theorem}{Theorem}[section]

\hypersetup{
    pdftitle=
}

\newcounter{totalpoints}
\newcommand\punkte[1]{#1\addtocounter{totalpoints}{#1}}

\newcounter{problemset}
\setcounter{problemset}{9}

\subject{physics606 -- Advanced Quantum Theory}
\ihead{physics606 -- Problem Set \arabic{problemset}}

\title{Problem Set \arabic{problemset}}

\publishers{Group 2 -- Dilege Gülmez}
\ofoot{Group 2 -- Dilege Gülmez}

\newmdenv[%
    backgroundcolor=black!0,
    frametitlebackgroundcolor=black!0,
    roundcorner=5pt,
    skipabove=\topskip,
    innertopmargin=\topskip,
    splittopskip=\topskip,
    frametitle={Problem statement},
    frametitlerule=true,
    nobreak=true,
]{problem}

\newmdenv[%
    backgroundcolor=white,
    frametitlebackgroundcolor=black!0,
    roundcorner=5pt,
    skipabove=\topskip,
    innertopmargin=\topskip,
    innerbottommargin=8cm,
    splittopskip=\topskip,
    frametitle={Side question},
    frametitlerule=true,
]{question}

\newcommand\an{^\text{angular}}
\newcommand\ra{^\text{radial}}


\author{
    Martin Ueding \\ \small{\href{mailto:mu@martin-ueding.de}{mu@martin-ueding.de}}
    \and
    Lino Lemmer \\ \small{\href{mailto:l2@uni-bonn.de}{l2@uni-bonn.de}}
}
\ifoot{Martin Ueding, Lino Lemmer}

\ohead{\rightmark}

\begin{document}

\maketitle

\vspace{3ex}

\begin{center}
    \begin{tabular}{rrr}
        problem number & achieved points & possible points \\
        \midrule
        1 & & \punkte{15} \\
        2 & & \punkte{13} \\
        \midrule
        total & & \arabic{totalpoints}
    \end{tabular}
\end{center}

\section{Scattering on a hard sphere}

\subsection{Phase shift}

\begin{problem}
    Write down the radial wave function.
\end{problem}

Inside the sphere, the potential is infinite, so it is exactly zero for any
angular momentum $l$:
\[
    R_l^<(r) = 0.
\]
The part outside the sphere is independent of the potential,
\[
    R_l^>(r) = \frac12 \sbr{h_l^*(kr) + \eup^{2\iup \delta_l} h_l(kr)},
\]
where $h_l := h_l^{(1)}$ \parencite[(18.59)]{Schwabl/Quantenmechanik}.

\begin{problem}
    Use the continuity of the wave function at $r = r_0$ together with the
    asymptotic expression for the wave function at $r \to \infty$ to show that
    \[
        \tan(\delta_l) = \frac{j_l(kr_0)}{n_l(kr_0)}.
    \]
\end{problem}

At the boundary $r_0$ the wave function $\psi$ has to be both continuous and
differentiable. Both criteria can be checked at the same time by the virtue of
the logarithmic derivative, which is defined in
\parencite[(18.60)]{Schwabl/Quantenmechanik} as
\[
    \alpha_l := \eval{\od{\,\log\del{R_l^<(r)}}r}_{r = r_0}
    = \eval{\frac{1}{R_l^<(r)} \od{R_l^<(r)}r}_{r = r_0}.
\]
As mentioned before, the wave function $\psi$ and its radial part $R$ will be
zero within this sphere of radius $r_0$. Also, since the potential makes an
infinite jump, $R$ will not be differentiable, just like the energy
eigenfunctions of the infinite potential well. Both factors in the logarithmic
derivative will be infinite, such that $\alpha_l = \infty$ is result. The
logarithmic derivative of the inner and outer solutions have to match up. This
leads to \parencite[(18.61)]{Schwabl/Quantenmechanik}:
\[
    \frac{1}{R_l^>(r)} \od{R_l^>(r)}r \overset != \alpha_l.
\]
The spherical Hankel functions can be split up into spherical Bessel $j_l$ and
spherical von~Neumann $n_l$ functions. Then the previous equation can be
transformed into \parencite[(18.62)]{Schwabl/Quantenmechanik}:
\[
    \eup^{2 \iup \delta_l} -1 = 2 \eval{\frac{\od{j_l}r - \alpha_l j_l}{\alpha_l h_l
    - \od{h_l}r}}_{r=r_0}.
\]
Solving for $\delta_l$ gives \parencite[(18.62)]{Schwabl/Quantenmechanik}:
\[
    \tan(\delta_l) = \eval{\frac{\od{j_l}r - \alpha_l j_l}{\od{n_l}r - \alpha_l
    n_l}}_{r=r_0}.
\]
Since $\alpha_l$ is infinite, the remainder of this fraction is
\[
    \tan(\delta_l) = \frac{j_l(kr_0)}{n_l(kr_0)},
\]
which is also the result mentioned for the hard sphere in
\parencite[(18.63)]{Schwabl/Quantenmechanik}.

\subsection{Negative shift}

\begin{problem}
    Use the explicit expressions for $j_0(x)$ and $n_0(x)$ to show that
    $\delta_0 = - kr_0$.
\end{problem}

The first order of those spherical functions are given by
\parencite{MathWorld/BesselFunctionoftheSecondKind,MathWorld/BesselFunctionoftheFirstKind}:
\[
    j_0(x) = \frac{\sin(x)}{x}
    \eqnsep
    n_0(x) = - \frac{\cos(x)}{x}.
\]
Inserting this into the previous expression for the phase shift gives us
\[
    \tan(\delta_l) = - \frac{\sin(kr_0)}{\cos(kr_0)}
\]
which simplifies to $\delta_l = - kr_0$. The minus sign can be put into both
sine and cosine, then the tangent can be inverted.

\begin{problem}
    Why is the phase shift negative here?
\end{problem}

The shift is negative since the wave is scattered on a repulsive potential.

\subsection{Expansion for small arguments}

\begin{problem}
    Use the expansions of $j_l$ and $n_l$ at small argument to show that
    $\delta_l \propto [kr_0]^{2l+1}$, just as for scattering on a spherical
    potential well.
\end{problem}

In the ninth exercise, we derived this approximation, it was also given as a
control result:
\[
    j_l(x) = x^l \frac{2^l l!}{[2l+1]!} + \mathcal O(x^{l+2}).
\]
The von~Neumann functions have a similar representation using repeated
differentiation \parencite{wikipedia/bessel_dgl}:
\[
    j_l(x) = [-x]^l \sbr{\frac 1x \od{}x}^l \frac{\sin(x)}{x}
    \eqnsep
    n_l(x) = - [-x]^l \sbr{\frac 1x \od{}x}^l \frac{\cos(x)}{x}.
\]

The derivation is similar. We expand the cosine in terms of its power series:
\begin{align*}
    n_l(x)
    &= - \sum_{n=0}^l \frac{[-1]^n}{[2n]!} [-x]^l \sbr{\frac 1x \od{}x}^l
    x^{2n-1}.
    \intertext{%
        Now we drop all but the lowest power, which comes at $n = 0$. This is
        the same as just using the first order expansion of the cosine.
    }
    &\approx - [-x]^l \sbr{\frac 1x \od{}x}^l x^{-1}.
    \intertext{%
        Assuming that $l \geq 1$, we can perform the derivatives and obtain
    }
    &= - [2l-1]!! \, x^{-1-l}.
\end{align*}

Then we simply have
\[
    \frac{j_l(x)}{n_l(x)} \propto \frac{x^l}{x^{-1-l}} = x^{2l+1}.
\]
However, this is $\tan(\delta_l)$, not $\delta_l$ itself. Do we make the
approximation $\tan(x) = x + \mathcal O(x^3)$ here as well?

\subsection{Asymptotic expansion}

\begin{problem}
    Use the asymptotic expressions for $j_l$ and $n_l$ for large arguments to
    show that
    \[
        \lim_{k r_0 \to \infty} \delta_l(k) = - kr_0 + \frac{l \piup}2.
    \]
\end{problem}

\parencite{wikipedia/bessel_dgl} gives the following large argument expressions
for the non-spherical functions:
\[
    J_\nu(x) = \sqrt{\frac{2}{\piup x}} \cos\del{x - \frac{\nu\piup}2 -
    \frac\piup4}
    \eqnsep
    N_\nu(x) = \sqrt{\frac{2}{\piup x}} \sin\del{x - \frac{\nu\piup}2 -
    \frac\piup4}.
\]
The relation to the spherical ones is given by
\[
    j_l(x) = \sqrt{\frac\piup{2x}} J_{l + 1/2}(x)
    \eqnsep
    n_l(x) = \sqrt{\frac\piup{2x}} N_{l + 1/2}(x).
\]
We now plug this together and get
\[
    j_l(x) = \frac1x \cos\del{x - \frac{l\piup}2}
    \eqnsep
    n_l(x) = \frac1x \sin\del{x - \frac{l\piup}2}.
\]
The ratio gives us
\[
    \tan(\delta_l) = \tan\del{kr_0 - \frac{l\piup}2}
    \iff
    \delta_l = kr_0 - \frac{l\piup}2.
\]
The problem is that the sign is wrong.

\subsection{Total cross section}

We start with equation \parencite[(18.40)]{Schwabl/Quantenmechanik}:
\begin{align*}
    \sigma
    &= \frac{4\piup}{k^2} \sum_{l=0}^{kr_0} [2l+1] \sin(\delta_l)^2. \\
    \intertext{%
        Now we insert the result of $\delta_l$ from the previous part.
    }
    &= \frac{4\piup}{k^2} \sum_{l=0}^{kr_0} [2l+1] \sin\del{-kr_0 +
    \frac{l\piup}2}^2
    \intertext{%
        We now split the sum into even and odd values of $l$, introducing $m$.
    }
    &= \frac{4\piup}{k^2} \sum_{m=0}^{kr_0/2} \sbr{
        [2[2m]+1] \sin\del{-kr_0 + \frac{2m\piup}2}^2
        + [2[2m+1]+1] \sin\del{-kr_0 + \frac{[2m+1]\piup}2}^2
    } \\
    &= \frac{4\piup}{k^2} \sum_{m=0}^{kr_0/2} \sbr{
        [4m+1] \sin\del{-kr_0 + m\piup}^2
    + [4m+3] \sin\del{-kr_0 + \frac{[2m+1]}2 \piup}^2
    }
    \intertext{%
        The first sine will always be a sine for any value of $m$. The square
        will take care of the sign, so we can just write this as a pure sine of
        the first summand. The second sine will always be a cosine.
    }
    &= \frac{4\piup}{k^2} \sum_{m=0}^{kr_0/2} \sbr{
        [4m+1] \sin(kr_0)^2 + [4m+3] \cos(kr_0)^2
    }
    \intertext{%
        We combine sine and cosine.
    }
    &= \frac{4\piup}{k^2} \sum_{m=0}^{kr_0/2} [8m+4]
    \intertext{%
        The sum can now be evaluated and gives
    }
    &= \frac{4\piup}{k^2} \sbr{[kr_0]^2 + 4kr_0 + 4}.
    \intertext{%
        We simplify more and only keep the leading term in $k$ since $k$ is
        supposed to be large.
    }
    &= 4 \piup r_0^2
\end{align*}
So this is a factor 4 larger than the geometrical cross section.

\section{Scattering length}

\subsection{Relation to scattering length}

We start with equation~(3) from the problem set:
\begin{align*}
    \tan(\delta_0) &= - k r_0 \frac{\alpha_0}{1 + \alpha_0}. \\
    \intertext{%
        Then we invert this equation.
    }
    \iff \cot(\delta_0) &= - \frac{1}{k r_0 \frac{\alpha_0}{1 + \alpha_0}} \\
    \intertext{%
        Now we multiply with $k$ on both sides.
    }
    \iff k \cot(\delta_0) &= - \frac{1}{r_0 \frac{\alpha_0}{1 + \alpha_0}} \\
    \intertext{%
        As a last step, we identify $R_0 = r_0 \frac{\alpha_0}{1 + \alpha_0}$.
    }
    \iff k \cot(\delta_0) &= - \frac{1}{R_0}
\end{align*}

\end{document}

% vim: spell spelllang=en tw=79
