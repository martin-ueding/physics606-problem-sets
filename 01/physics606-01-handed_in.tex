\documentclass[11pt, ngerman, fleqn, DIV=15, headinclude, BCOR=1cm]{scrartcl}

\usepackage[bibatend]{../header}

\usepackage{tikz}

\usepackage[tikz]{mdframed}
\newmdtheoremenv[%
    backgroundcolor=black!5,
    innertopmargin=\topskip,
    splittopskip=\topskip,
]{theorem}{Theorem}[section]

\hypersetup{
    pdftitle=
}

\newcounter{totalpoints}
\newcommand\punkte[1]{#1\addtocounter{totalpoints}{#1}}

\newcounter{problemset}
\setcounter{problemset}{1}

\subject{physics606 -- Advanced Quantum Theory}
\ihead{physics606 -- Problem Set \arabic{problemset}}

\title{Problem Set \arabic{problemset}}

\publishers{Group 2 -- Dilege Gülmez}
\ofoot{Group 2 -- Dilege Gülmez}



\author{
    Martin Ueding \\ \small{\href{mailto:mu@martin-ueding.de}{mu@martin-ueding.de}}
    \and
    Lino Lemmer \\ \small{\href{mailto:l2@uni-bonn.de}{l2@uni-bonn.de}}
}
\ifoot{Martin Ueding, Lino Lemmer}

\ohead{\rightmark}

\begin{document}

\maketitle

\vspace{3ex}

\begin{center}
    \begin{tabular}{rrr}
        problem number & achieved points & possible points \\
        \midrule
        1 & & \punkte{8} \\
        2 & & \punkte{8} \\
        3 & & \punkte{5} \\
        4 & & \punkte{10} \\
        \midrule
        Total & & \arabic{totalpoints}
    \end{tabular}
\end{center}

\vspace{5ex}

I, Martin Ueding, would like to scan and upload the problem sets with your
corrections to my website \href{http://martin-ueding.de}{martin-ueding.de}.
There, the original problem set as well as the reviewed one will be licensed
under the “\href{http://creativecommons.org/licenses/by-sa/4.0/}{Creative
Commons Attribution-ShareAlike 4.0 International License}”. Is that okay with
you?

Yes $\Box$ \hspace{2cm} No $\Box$

\newpage

\section{Hermitean Operators}

\subsection{Real Eigenvalues}

Let the eigenvalue of the operator $Q$ on $\psi_2$ be $q_2$. For now, we assume
$q_2 \in \C$. We start with the equation~(1) from the problem set:
\begin{align*}
    \int \dif x \, \psi_1^*(x) Q \psi_2^*(x)
    &= \int \dif x \, \sbr{Q \psi_1(x)}^* \psi_2^*(x). \\
    \intertext{%
        We expand the complex conjugate in the square bracket:
    }
    \int \dif x \, \psi_1^*(x) Q \psi_2^*(x)
    &= \int \dif x \, \psi_1^*(x) Q^\dagger \psi_2^*(x). \\
    \intertext{%
        The eigenvalue of $Q^\dagger$ is the complex conjugate of the
        eigenvalue of $Q$:
    }
    q_2 \int \dif x \, \psi_1^*(x) \psi_2^*(x)
    &= q_2^* \int \dif x \, \psi_1^*(x) \psi_2^*(x) \\
    \iff q_2 &= q_2^*.
\end{align*}

This restricts the eigenvalues to $\Im(q) = 0$ which means that $q \in \R$. The
eigenvalues are all real.

\subsection{Orthogonal Eigenfunctions}

We will use the bra-ket notation here. Let $\ket n$ for $n \in \N$ (perhaps
bounded to some $N$) be the eigenstates for $Q$ with eigenvalues $q_n$. All the
$q_n$ are assume to be pairwise different.

First we have for $m, n \in \N$ which are assumed to be different:
\begin{align*}
    \braket{n | Q | m} &= q_m \braket{n | m}, \\
    \braket{m | Q | n} &= q_n \braket{m | n}.
\end{align*}

Since
\[
    \braket{n | Q | m} = \braket{m | Q | n}^*,
\]
the equations above are the complex conjugate of each other. This statement
will be proven in the next part of this problem. Including the previously
proven fact that the eigenvalues are real, we can derive the following
equation:
\[
    q_m \braket{n|m} = q_n \braket{n|m}.
\]
This leaves us with:
\[
    [q_m - q_n] \braket{n|m} = 0.
\]
We assume non-degenerate eigenvalues, which means that $\braket{n|m} = 0$. The
definition of orthogonality is a vanishing scalar product, which is the case
here. Therefore $\ket n$ and $\ket m$ are orthogonal.

\subsection{Matrix Representation}

Basically we have to show that $Q_{ij} = Q_{ji}^*$ actually holds here. The
definition of the matrix element is:
\begin{align*}
    Q_{ij}
    &= \int \dif x \, \psi_i^*(x) Q \psi_j(x). \\
    \intertext{%
        From equation~(1) from the problem set we know that we can write this
        like this as well:
    }
    &= \int \dif x \, \sbr{Q^\dagger \psi_i(x)}^* \psi_j^*(x). \\
    \intertext{%
        We include the last factor into the complex conjugation:
    }
    &= \int \dif x \, \sbr{\psi_j^*(x) Q^\dagger \psi_i(x)}^*. \\
    \intertext{%
        Integration and complex conjugation commutes, leaving
    }
    &= \sbr{ \int \dif x \, \psi_j^*(x) Q^\dagger \psi_i(x)}^*. \\
    \intertext{%
        This, however, is just the definition of
    }
    &= Q_{ji}^*.
\end{align*}

\section{Decomposition of a Wave Function}

\subsection{Coefficients}

We have
\[
    \psi(x, t) = \sum_n u_n(t) \psi_n(x).
\]
There $\psi$ are different, one has an index, the other does not. We will use
this notation although it is a little overloaded. From there, we project the
$\psi$ onto a $\psi_m$:
\begin{align*}
    \int \dif x \, \psi_n^*(x) \psi(x, t)
    &= \sum_m u_m (t) \int \dif x \, \psi_n^*(x) \psi_m(x) \\
    &= u_n(t).
\end{align*}

Therefore, the equation~(4) from the problem set holds.

\subsection{Normalization}

For this problem, we just insert the definition of $\psi$:
\begin{align*}
    \int \dif x \, |\psi(x, t)|^2
    &= \int \dif x \, \sbr{\sum_n u_n^*(t) \psi_n^*(x)} \sbr{\sum_m u_m(t)
    \psi_m(x)} \\
    &= \sum_{n,m} u_n^*(t) u_m(t) \int \dif x \, \psi_n^*(x) \psi_m(x) \\
    \intertext{%
        Eigenfunctions are orthogonal, so this reduces to the $m = n$ terms:
    }
    &= \sum_{n,m} u_n^*(t) u_m(t) \braket{n|m} \\
    &= \sum_{n} |u_n(t)|^2
\end{align*}

Since $\psi$ had to be normalized, the condition
\[
    \sum_{n} |u_n(t)|^2 = 1
\]
follows.

\subsection{Expectation value}

Just inserting again:
\begin{align*}
    \bracket Q
    &= \int \dif x \, \psi^*(x) Q \psi(x) \\
    &= \sum_{n, m} u_m^*(t) u_n(t) \int \dif x \, \psi_m^*(x) Q \psi_n(x) \\
    &= \sum_{n, m} u_m^*(t) u_n(t) q_n \delta_{nm} \\
    &= \sum_{n} |u_n(t)|^2 q_n .
\end{align*}

\section{Angular Momentum Operator}

\subsection{Eigenvalue}

We have
\begin{align*}
    L_z &= -\iup\hbar\dpd{}{\phi}
    \intertext{and}
    \psi_m(\phi) &= \frac1{\sqrt{2\piup}}\exp\del{\iup m\phi}.
\end{align*}

First the eigenvalues:
\begin{align*}
    L_z \psi_m(\phi) &=
    -\iup\hbar \dpd{}{\phi} \sbr{\frac1{\sqrt{2\piup}}\exp\del{\iup m\phi}} \\
    &= -\iup\hbar \frac{\iup m}{\sqrt{2\piup}}\exp\del{\iup m\phi} \\
    &= \hbar m \psi_m(\phi).
\end{align*}

Now we show, that the eigenfunctions are normalized:
\begin{align*}
    \abs{\psi_m}^2 &= \int_0^{2\piup}\!\dif\phi\, \psi_m^*\psi_m \\
                   &= \frac1{2\piup} \int_0^{2\piup}\!\dif\phi\, \exp\del{-\iup
m\phi}\exp\del{\iup m\phi} \\
&= \frac1{2\piup} \int_0^{2\piup}\!\dif\phi \\
&= 1.
\end{align*}

\subsection{Integer $m$}

We want 
\begin{align*}
    \psi_m(\phi) &= \psi_m(\phi + 2\piup).
    \intertext{This implies}
    \exp(\iup m\phi) &= \exp\left(\iup m(\phi + 2\piup)\right) \\
                     &= \exp(\iup m\phi)
    \underbrace{\exp(\iup m2\piup)}_{=1,\text{ for }m\in\mathbb{N}}.
\end{align*}
Therefor $m$ has to be integer.

\subsection{Orthonormal eigenfunctions}

\begin{align*}
    \int_0^{2\piup}\!\dif\phi\, \psi_l^*\psi_m &= \frac1{2\piup}
    \int_0^{2\piup}\!\dif\phi\, \exp(-\iup l\phi)\exp(\iup m\phi) \\
    &= \frac1{2\piup} \int_0^{2\piup}\!\dif\phi\, \exp\left(\iup
    \phi(m-l)\right).
    \intertext{We look at $m\neq l$:}
    &= \frac1{2\piup} \frac1{\iup(m-l)} \sbr{\exp\left(\iup
    \phi(m-l)\right)}_0^{2\piup},
    \intertext{from $m-l\in\mathbb{N}$ we get}
    &= 0.
    \intertext{%
        The other case $l=m$ gives because of the normalization
    }
    \int_0^{2\piup}\!\dif\phi\, \psi_m^*\psi_m &= 1.
    \intertext{%
        These two cases show us, that the eigenfunctions are orthonormal:
    }
    \int_0^{2\piup}\!\dif\phi\, \psi_l^*\psi_m &= \delta_{lm}.
\end{align*}


\section{Canonical Transformations}

\subsection{Form-Invariance of Equations of Motion}

\begin{theorem}
    Let $A$ and $B$ be two functions of all the $q_i$ and $p_j$. A canonical
    transformation is defined via the new coordinates $\bar q_i$ and $\bar p_j$
    which depend on all the old coordinates and momenta. The functions $A$ and
    $B$ are transformed to $\bar A$ and $\bar B$ and only depend on the new
    coordinates and momenta.
\end{theorem}

\begin{proof}
    \begin{align*}
        \{
    \end{align*}
\end{proof}

\subsection{Check for canonicality}

\subsection{Invariance of Poisson brackets}

As the problem is stated, we believe that it actually is trivial. It says that
one should proof that
\[
    \cbr{A(q, p), B(q, p)}_{q,p} = \cbr{A(\bar q, \bar p), B(\bar q, \bar
    p)}_{\bar q,\bar p}
\]
holds. We mean this in the sense that both sides contain the functions $A$ and
$B$ which take two arguments. Both sides only depend on dummy variables, the
$q$ and $p$ with and without a bar. Since $f(x) = f(y)$ if $x = y$ (as is the
definition of a function), there cannot be any difference whether the Poisson
brackets are spelled out with $q$ or $\bar q$. That said, we think that the
problem actually implies that the left and the right $A$ are two different
functions. So we will call them $A$ and $\bar A$ to make this a non-trivial
problem.

We now assume that the functions $A$ and $\bar A$ relate in the following way:
\[
    \bar A\del{\bar q(q, p), \bar p(q, p)} = A(q, p).
\]
So for a given $q$ and $p$ both functions take the same \emph{value}, although
the arguments for $\bar A$ and $A$ differ. If this is to be the case, the total
differential with respect to $q$ (and $p$) has to be the same. This brings us
to
\[
    \dpd{\bar A}{\bar q} \dpd{\bar q}q + \dpd{\bar A}{\bar p} \dpd{\bar p}q =
    \dpd Aq.
\]

We start off with the Poisson bracket of $A$ and $B$. Since there might be more
than one coordinate, we start indexing them again.
\begin{align*}
    \{ A, B \}_{q,p}
    &= \sum_k \sbr{
        \dpd{A}{{q_k}} \dpd{B}{{p_k}} - \dpd{A}{{p_k}} \dpd{B}{{q_k}}
    } \\
    \intertext{%
        We use the relation that we derived using the total differential to
        expand the partial differentials. This is going to be messy.
    }
    &= \sum_{j,k}
    \sbr{
        \dpd{\bar A}{\bar q_j} \dpd{\bar q_j}{q_k} % 1
        + \dpd{\bar A}{\bar p_j} \dpd{\bar p_j}{q_k} % 2
    }
    \sbr{
        \dpd{\bar B}{\bar q_j} \dpd{\bar q_j}{p_k} % 3
        + \dpd{\bar B}{\bar p_j} \dpd{\bar p_j}{p_k} % 4
    } \\
    &\phantom=- \sum_{j,k}
    \sbr{
        \dpd{\bar A}{\bar q_j} \dpd{\bar q_j}{p_k} % 5
        + \dpd{\bar A}{\bar p_j} \dpd{\bar p_j}{p_k} % 6
    }
    \sbr{
        \dpd{\bar B}{\bar q_j} \dpd{\bar q_j}{q_k} % 7
        + \dpd{\bar B}{\bar p_j} \dpd{\bar p_j}{q_k} % 8
    } \\
    \intertext{%
        We now expand this whole thing, such that it becomes even messier.
    }
    &= \sum_{j,k} \Bigg[
        \dpd{\bar A}{\bar q_j} \dpd{\bar q_j}{q_k} % 1
        \dpd{\bar B}{\bar q_j} \dpd{\bar q_j}{p_k} % 3
        + \dpd{\bar A}{\bar q_j} \dpd{\bar q_j}{q_k} % 1
        \dpd{\bar B}{\bar p_j} \dpd{\bar p_j}{p_k} % 4
        + \dpd{\bar A}{\bar p_j} \dpd{\bar p_j}{q_k} % 2
        \dpd{\bar B}{\bar q_j} \dpd{\bar q_j}{p_k} % 3
        \\
        &\qquad
        + \dpd{\bar A}{\bar p_j} \dpd{\bar p_j}{q_k} % 2
        \dpd{\bar B}{\bar p_j} \dpd{\bar p_j}{p_k} % 4
        - \dpd{\bar A}{\bar q_j} \dpd{\bar q_j}{p_k} % 5
        \dpd{\bar B}{\bar q_j} \dpd{\bar q_j}{q_k} % 7
        - \dpd{\bar A}{\bar q_j} \dpd{\bar q_j}{p_k} % 5
        \dpd{\bar B}{\bar p_j} \dpd{\bar p_j}{q_k} % 8
        \\
        &\qquad
        - \dpd{\bar A}{\bar p_j} \dpd{\bar p_j}{p_k} % 6
        \dpd{\bar B}{\bar q_j} \dpd{\bar q_j}{q_k} % 7
        - \dpd{\bar A}{\bar p_j} \dpd{\bar p_j}{p_k} % 6
        \dpd{\bar B}{\bar p_j} \dpd{\bar p_j}{q_k} % 8
    \Bigg]
    \intertext{%
        We will now group them again by the derivatives of $\bar A$ and $\bar
        B$ into four groups.
    }
    &=
        \dpd{\bar A}{\bar q_j} \dpd{\bar q_j}{q_k}
        \dpd{\bar B}{\bar q_j} \dpd{\bar q_j}{p_k}
        - \dpd{\bar A}{\bar q_j} \dpd{\bar q_j}{p_k}
        \dpd{\bar B}{\bar q_j} \dpd{\bar q_j}{q_k}

        + \dpd{\bar A}{\bar p_j} \dpd{\bar p_j}{q_k}
        \dpd{\bar B}{\bar p_j} \dpd{\bar p_j}{p_k}
        - \dpd{\bar A}{\bar p_j} \dpd{\bar p_j}{p_k}
        \dpd{\bar B}{\bar p_j} \dpd{\bar p_j}{q_k}

        + \dpd{\bar A}{\bar q_j} \dpd{\bar q_j}{q_k} % 1
        \dpd{\bar B}{\bar p_j} \dpd{\bar p_j}{p_k} % 4
        - \dpd{\bar A}{\bar q_j} \dpd{\bar q_j}{p_k} % 5
        \dpd{\bar B}{\bar p_j} \dpd{\bar p_j}{q_k} % 8

        + \dpd{\bar A}{\bar p_j} \dpd{\bar p_j}{q_k} % 2
        \dpd{\bar B}{\bar q_j} \dpd{\bar q_j}{p_k} % 3
        - \dpd{\bar A}{\bar p_j} \dpd{\bar p_j}{p_k} % 6
        \dpd{\bar B}{\bar q_j} \dpd{\bar q_j}{q_k} % 7

\end{align*}

\end{document}

% vim: spell spelllang=en tw=79
