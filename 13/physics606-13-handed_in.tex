\documentclass[11pt, english, fleqn, DIV=15, headinclude, BCOR=1.5cm]{scrartcl}

\usepackage[
    bibatend,
    color,
]{../header}

\usepackage{tikz}
\usepackage{pdflscape}

\usepackage[tikz]{mdframed}
\newmdtheoremenv[%
    backgroundcolor=black!5,
    innertopmargin=\topskip,
    splittopskip=\topskip,
]{theorem}{Theorem}[section]

\hypersetup{
    pdftitle=
}

\newcounter{totalpoints}
\newcommand\punkte[1]{#1\addtocounter{totalpoints}{#1}}

\newcounter{problemset}
\setcounter{problemset}{13}

\subject{physics606 -- Advanced Quantum Theory}
\ihead{physics606 -- Problem Set \arabic{problemset}}

\title{Problem Set \arabic{problemset}}

\publishers{Group 2 -- Dilege Gülmez}
\ofoot{Group 2 -- Dilege Gülmez}

\newmdenv[%
    backgroundcolor=black!0,
    frametitlebackgroundcolor=black!0,
    roundcorner=5pt,
    skipabove=\topskip,
    innertopmargin=\topskip,
    splittopskip=\topskip,
    frametitle={Problem statement},
    frametitlerule=true,
    nobreak=true,
]{problem}

\newmdenv[%
    backgroundcolor=white,
    frametitlebackgroundcolor=black!0,
    roundcorner=5pt,
    skipabove=\topskip,
    innertopmargin=\topskip,
    innerbottommargin=8cm,
    splittopskip=\topskip,
    frametitle={Side question},
    frametitlerule=true,
]{question}

\newcommand\an{^\text{angular}}
\newcommand\ra{^\text{radial}}


\author{
    Martin Ueding \\ \small{\href{mailto:mu@martin-ueding.de}{mu@martin-ueding.de}}
    \and
    Lino Lemmer \\ \small{\href{mailto:l2@uni-bonn.de}{l2@uni-bonn.de}}
}
\ifoot{Martin Ueding, Lino Lemmer}

\ohead{\rightmark}

\begin{document}

\maketitle

\vspace{3ex}

\begin{center}
    \begin{tabular}{rrr}
        problem number & achieved points & possible points \\
        \midrule
        1 & & \punkte{10} \\
        2 & & \punkte{4} \\
        3 & & \punkte{16} \\
        \midrule
        total & & \arabic{totalpoints}
    \end{tabular}
\end{center}

\section{Annihilation operator in Heisenberg picture}

\subsection{}

\begin{problem}
    Show that
    \[
        a_j^{\text H \dagger}(t)
        = \exp\del{\frac\iup\hbar H t} a_j^\dagger \exp\del{\frac\iup\hbar H
        t}.
    \]
    Hint: Take the hermitean conjugate of equation~(2), remembering that
    $[AB]^\dagger = B^\dagger A^\dagger$.
\end{problem}

Okay, so we start with the equation~(2) that was mentioned in the problem
statement. This can be found on the problem set and reads the following:
\[
    a_j^{\text H}(t)
    = \exp\del{\frac\iup\hbar H t} a_j \exp\del{- \frac\iup\hbar H
    t}.
\]
We omit the parentheses around the “Heisenberg” “H” because we conform to ISO
80000-2 and set the “H” upright. Therefore it can always be distinguished from
the italic Hamiltonian $H$.

Now we follow the lead of the hint and apply the Hermitean conjugate to this
equation. To make it really clear what happens, we do not make any
simplification steps up to this point.
\begin{align*}
    \iff a_j^{\text H \dagger}(t)
    &= \sbr{
        \exp\del{\frac\iup\hbar H t} a_j \exp\del{- \frac\iup\hbar H t}
    }^\dagger
    \intertext{%
        It can be seen that we have added the Hermitean conjugate “$\dagger$”
        to both sides of the previous equation. Since we have applied this to
        both sides, this equation is equivalent to the previous one. We need
        this equivalence to be able to show that equation~(4) from the problem
        set does indeed follows from equation~(2) from the problem set. Now
        comes the next crucial step in this derivation: We need to apply the
        second part of the hint, the one that said that we should remember that
        $[AB]^\dagger = B^\dagger A^\dagger$ holds. To make it clear which term
        went into which position, we will give them extra labels first.
    }
    &= \sbr{
    \underbrace{\exp\del{\frac\iup\hbar H t}}_A a_j \underbrace{\exp\del{-
    \frac\iup\hbar H t}}_B
    }^\dagger
    \intertext{%
        Now we are all set to apply the hint to this equation.
    }
    &=
    \underbrace{\exp\del{- \frac\iup\hbar H t}^\dagger}_B a_j^\dagger \underbrace{\exp\del{
    \frac\iup\hbar H t}^\dagger}_A
    \intertext{%
        With the extra labels, it should be visible that the terms changed
        their order and all got a Hermitean conjugate “$\dagger$” on their own.
        The next step was not given in the hint. Now we have to simplify the
        Hermitean conjugate of the exponential function. One way to see how
        this works is to split up the exponential function into real and
        imaginary parts. The complex exponential function can be written like
        so:
        \[
            \exp(\iup \phi) = \cos(\phi) + \iup \sin(\phi)
            \eqnsep
            \phi \in \R.
        \]
        There Hermitean conjugate will, just like the complex conjugate, switch
        the sign on the imaginary part. So here, the Hermitean conjugate will
        be given by
        \[
            \exp(\iup \phi)^\dagger = \cos(\phi) - \iup \sin(\phi).
        \]
        To write this back into the exponential form, one has to use that the
        cosine is symmetric, the sine is antisymmetric in its argument. So we
        can write this as
        \[
            \exp(\iup \phi)^\dagger = \cos(\phi) + \iup \sin(-\phi).
        \]
    }
    \intertext{%
        However, using the above formula, we can also write this as
        \[
            \exp(- \iup \phi)
        \]
        and deduce that
        \[
            \exp(\iup \phi)^\dagger = \exp(- \iup \phi)
        \]
    }
    \intertext{%
        So for a pure imaginary number in the exponential, the sign will just
        flip. The Hamiltonian $H$ is a matrix here, or can at least be thought
        of as a matrix. The Hermitean conjugate will also transpose the matrix.
        What we have to do is to apply the Hermitean conjugate to the
        Hamiltonian as well. Luckily, it is a self-adjoined operator, which
        means that $H = H^\dagger$. So we do not have to take care of this.
        Using all this, we can simplify the exponential functions.
    }
    &= \exp\del{\frac\iup\hbar H t} a_j^\dagger \exp\del{- \frac\iup\hbar H t}
\end{align*}
And that is exactly equation~(4) from the problem set.

\subsection{}

The first thing that we will do is to show the identity:
\begin{align*}
    [AB, C]
    &= ABC - CAB \\
    &= ABC - ACB + ACB - CAB \\
    &= A[B, C] + [A, C]B
\end{align*}

Then we can solve the differential equation.
\begin{align*}
    \dot a_k^{\text H}(t)
    &= \frac\iup\hbar \sbr{H, a_k^{\text H}(t)}
    \intertext{%
        The first thing we do is to insert the unitary evolution operators.
    }
    &= \frac\iup\hbar \sbr{H, U^\dagger(t) a_k U(t)}
    \intertext{%
        Since $U$ is a analytic function of $H$, it will commute with $H$. Even
        the $a_j^\dagger a_j$ will, taken together as $n_j$, commute with $H$
        and every sufficiently well behaved function of $H$.
    }
    &= \frac\iup\hbar U^\dagger(t) [H, a_k] U(t)
    \intertext{%
        Then we can insert $H$ in terms of the occupation number operators.
    }
    &= \frac\iup\hbar U^\dagger(t) \sum_j \epsilon_j [a_j^\dagger a_j, a_k] U(t)
    \intertext{%
        This has the form that lets us use the relation that we have shown.
    }
    &= \frac\iup\hbar U^\dagger(t) \sum_j \epsilon_j \sbr{
        a_j^\dagger [a_j, a_k] - [a_j^\dagger, a_k] a_j
    }
    U(t)
    \intertext{%
        The first commutator is zero, the second a negative Kronecker $\delta$.
        For fermions, it will have the anti-commutator here which gives the
        same value.
    }
    &= \frac\iup\hbar \epsilon_k U^\dagger(t) a_k U(t)
    \intertext{%
        This again is the operator in the Heisenberg picture. The differential
        equation therefore is:
    }
    \dot a_k^{\text H}(t)
    &= \frac\iup\hbar \epsilon_k a_k^{\text H}(t).
\end{align*}

With the initial condition that at $t = 0$ the Heisenberg and Schrödinger
operators are the same ($a_k^{\text H}(0) = a_k$) this can be solved by:
\[
    a_k^{\text H}(t) = \exp\del{\frac\iup\hbar \epsilon_k t} a_k
\]

\subsection{}

Trivial. Do the same derivation and note that the fermion anti-commutator has
the same value as the boson commutator.

\end{document}

% vim: spell spelllang=en tw=79
