\documentclass[11pt, english, fleqn, DIV=15, headinclude, BCOR=1.5cm]{scrartcl}

\usepackage[
    bibatend,
    color,
]{../header}

\usepackage{tikz}
\usepackage{pdflscape}

\usepackage[tikz]{mdframed}
\newmdtheoremenv[%
    backgroundcolor=black!5,
    innertopmargin=\topskip,
    splittopskip=\topskip,
]{theorem}{Theorem}[section]

\hypersetup{
    pdftitle=
}

\newcounter{totalpoints}
\newcommand\punkte[1]{#1\addtocounter{totalpoints}{#1}}

\newcounter{problemset}
\setcounter{problemset}{13}

\subject{physics606 -- Advanced Quantum Theory}
\ihead{physics606 -- Problem Set \arabic{problemset}}

\title{Problem Set \arabic{problemset}}

\publishers{Group 2 -- Dilege Gülmez}
\ofoot{Group 2 -- Dilege Gülmez}

\newmdenv[%
    backgroundcolor=black!0,
    frametitlebackgroundcolor=black!0,
    roundcorner=5pt,
    skipabove=\topskip,
    innertopmargin=\topskip,
    splittopskip=\topskip,
    frametitle={Problem statement},
    frametitlerule=true,
    nobreak=true,
]{problem}

\newmdenv[%
    backgroundcolor=white,
    frametitlebackgroundcolor=black!0,
    roundcorner=5pt,
    skipabove=\topskip,
    innertopmargin=\topskip,
    innerbottommargin=8cm,
    splittopskip=\topskip,
    frametitle={Side question},
    frametitlerule=true,
]{question}

\newcommand\an{^\text{angular}}
\newcommand\ra{^\text{radial}}


\author{
    Martin Ueding \\ \small{\href{mailto:mu@martin-ueding.de}{mu@martin-ueding.de}}
    \and
    Lino Lemmer \\ \small{\href{mailto:l2@uni-bonn.de}{l2@uni-bonn.de}}
}
\ifoot{Martin Ueding, Lino Lemmer}

\ohead{\rightmark}

\begin{document}

\maketitle

\vspace{3ex}

\begin{center}
    \begin{tabular}{rrr}
        problem number & achieved points & possible points \\
        \midrule
        1 & & \punkte{10} \\
        2 & & \punkte{4} \\
        3 & & \punkte{16} \\
        \midrule
        total & & \arabic{totalpoints}
    \end{tabular}
\end{center}

\section{Annihilation operator in Heisenberg picture}

\subsection{}

\begin{problem}
    Show that
    \[
        a_j^{\text H \dagger}(t)
        = \exp\del{\frac\iup\hbar H t} a_j^\dagger \exp\del{\frac\iup\hbar H
        t}.
    \]
    Hint: Take the hermitean conjugate of equation~(2), remembering that
    $[AB]^\dagger = B^\dagger A^\dagger$.
\end{problem}

Okay, so we start with the equation~(2) that was mentioned in the problem
statement. This can be found on the problem set and reads the following:
\[
    a_j^{\text H}(t)
    = \exp\del{\frac\iup\hbar H t} a_j \exp\del{- \frac\iup\hbar H
    t}.
\]
We omit the parentheses around the “Heisenberg” “H” because we conform to ISO
80000-2 and set the “H” upright. Therefore it can always be distinguished from
the italic Hamiltonian $H$.

Now we follow the lead of the hint and apply the Hermitean conjugate to this
equation. To make it really clear what happens, we do not make any
simplification steps up to this point.
\begin{align*}
    \iff a_j^{\text H \dagger}(t)
    &= \sbr{
        \exp\del{\frac\iup\hbar H t} a_j \exp\del{- \frac\iup\hbar H t}
    }^\dagger
    \intertext{%
        It can be seen that we have added the Hermitean conjugate “$\dagger$”
        to both sides of the previous equation. Since we have applied this to
        both sides, this equation is equivalent to the previous one. We need
        this equivalence to be able to show that equation~(4) from the problem
        set does indeed follows from equation~(2) from the problem set. Now
        comes the next crucial step in this derivation: We need to apply the
        second part of the hint, the one that said that we should remember that
        $[AB]^\dagger = B^\dagger A^\dagger$ holds. To make it clear which term
        went into which position, we will give them extra labels first.
    }
    &= \sbr{
    \underbrace{\exp\del{\frac\iup\hbar H t}}_A a_j \underbrace{\exp\del{-
    \frac\iup\hbar H t}}_B
    }^\dagger
    \intertext{%
        Now we are all set to apply the hint to this equation.
    }
    &=
    \underbrace{\exp\del{- \frac\iup\hbar H t}^\dagger}_B a_j^\dagger \underbrace{\exp\del{
    \frac\iup\hbar H t}^\dagger}_A
    \intertext{%
        With the extra labels, it should be visible that the terms changed
        their order and all got a Hermitean conjugate “$\dagger$” on their own.
        The next step was not given in the hint. Now we have to simplify the
        Hermitean conjugate of the exponential function. One way to see how
        this works is to split up the exponential function into real and
        imaginary parts. The complex exponential function can be written like
        so:
        \[
            \exp(\iup \phi) = \cos(\phi) + \iup \sin(\phi)
            \eqnsep
            \phi \in \R.
        \]
        There Hermitean conjugate will, just like the complex conjugate, switch
        the sign on the imaginary part. So here, the Hermitean conjugate will
        be given by
        \[
            \exp(\iup \phi)^\dagger = \cos(\phi) - \iup \sin(\phi).
        \]
        To write this back into the exponential form, one has to use that the
        cosine is symmetric, the sine is antisymmetric in its argument. So we
        can write this as
        \[
            \exp(\iup \phi)^\dagger = \cos(\phi) + \iup \sin(-\phi).
        \]
    }
    \intertext{%
        However, using the above formula, we can also write this as
        \[
            \exp(- \iup \phi)
        \]
        and deduce that
        \[
            \exp(\iup \phi)^\dagger = \exp(- \iup \phi)
        \]
    }
    \intertext{%
        So for a pure imaginary number in the exponential, the sign will just
        flip. The Hamiltonian $H$ is a matrix here, or can at least be thought
        of as a matrix. The Hermitean conjugate will also transpose the matrix.
        What we have to do is to apply the Hermitean conjugate to the
        Hamiltonian as well. Luckily, it is a self-adjoined operator, which
        means that $H = H^\dagger$. So we do not have to take care of this.
        Using all this, we can simplify the exponential functions.
    }
    &= \exp\del{\frac\iup\hbar H t} a_j^\dagger \exp\del{- \frac\iup\hbar H t}
\end{align*}
And that is exactly equation~(4) from the problem set.

\subsection{}

The first thing that we will do is to show the identity:
\begin{align*}
    [AB, C]
    &= ABC - CAB \\
    &= ABC - ACB + ACB - CAB \\
    &= A[B, C] + [A, C]B
\end{align*}
If you swap the two middle terms, you can derive this relations for
anti-commutators.

Then we can solve the differential equation.
\begin{align*}
    \dot a_k^{\text H}(t)
    &= \frac\iup\hbar \sbr{H, a_k^{\text H}(t)}
    \intertext{%
        The first thing we do is to insert the unitary evolution operators.
    }
    &= \frac\iup\hbar \sbr{H, U^\dagger(t) a_k U(t)}
    \intertext{%
        Since $U$ is a analytic function of $H$, it will commute with $H$. Even
        the $a_j^\dagger a_j$ will, taken together as $n_j$, commute with $H$
        and every sufficiently well behaved function of $H$.
    }
    &= \frac\iup\hbar U^\dagger(t) [H, a_k] U(t)
    \intertext{%
        Then we can insert $H$ in terms of the occupation number operators.
    }
    &= \frac\iup\hbar U^\dagger(t) \sum_j \epsilon_j [a_j^\dagger a_j, a_k] U(t)
    \intertext{%
        This has the form that lets us use the relation that we have shown.
    }
    &= \frac\iup\hbar U^\dagger(t) \sum_j \epsilon_j \sbr{
        a_j^\dagger [a_j, a_k] - [a_j^\dagger, a_k] a_j
    }
    U(t)
    \intertext{%
        The first commutator is zero, the second a negative Kronecker $\delta$.
        For fermions, it will have the anti-commutator here which gives the
        same value.
    }
    &= \frac\iup\hbar \epsilon_k U^\dagger(t) a_k U(t)
    \intertext{%
        This again is the operator in the Heisenberg picture. The differential
        equation therefore is:
    }
    \dot a_k^{\text H}(t)
    &= \frac\iup\hbar \epsilon_k a_k^{\text H}(t).
\end{align*}

With the initial condition that at $t = 0$ the Heisenberg and Schrödinger
operators are the same ($a_k^{\text H}(0) = a_k$) this can be solved by:
\[
    a_k^{\text H}(t) = \exp\del{\frac\iup\hbar \epsilon_k t} a_k
\]

\subsection{}

Trivial. Do the same derivation and note that the fermion anti-commutator has
the same value as the boson commutator.

\begin{landscape}

\section{Time evolution of fermionic field operators}

On problem set 12 we more or less derived the Hamiltonian in terms of the field
operators. So we will continue from that and just follow the derivation from
\parencite[24]{Schwabl/Quantenmechanik_fuer_Fortgeschrittene}. So the time
evolution of anything is given by
\begin{align*}
    \iup \hbar \pd{}t \psi(\vec x, t)
    &= - [H, \psi(\vec x, t)].
    \intertext{%
        Now we write this in the Schrödinger picture for a little while and
        strip off the time evolution of the state. As shown in problem 1.1 in
        somewhat excruciating detail, they commute with the Hamiltonian.
        Therefore, we can move them out of the commutator.
    }
    &= - U^\dagger(t) [H, \psi(\vec x)] U(t)
    \intertext{%
        Now we can plug in the Hamiltonian that we derived on problem set 12.
    }
    &= - U^\dagger(t)
    \sbr{
        \int \dif^3 y \sbr{
            \frac{\hbar^2}{2m} \vnabla \psi^\dagger(\vec y) \vnabla \psi(\vec y)
            + U(\vec y) \psi^\dagger(\vec y) \psi(\vec y)
        }
        + \frac12 \int \dif^3 y \, \dif^3 z \, \psi^\dagger(\vec y)
        \psi^\dagger(\vec z) V(\vec y, \vec z) \psi(\vec z) \psi(\vec y)
        , \psi(\vec x)
    } U(t)
    \intertext{%
        Now that is some long expression. And it is really hard to see that the
        outer square bracket is a commutator. Well, it should be clear
        \emph{now}, then. A lot can be pulled out of the commutator right here,
        so that might simplify it a bit. Now it does not even fit on a single
        line in landscape, but that does not really hurt that much.
    }
    &= - U^\dagger(t)
    \int \dif^3 y \sbr{
        \frac{\hbar^2}{2m} [\vnabla \psi^\dagger(\vec y) \vnabla \psi(\vec
        y), \psi(\vec x)]
        + U(\vec y) [\psi^\dagger(\vec y) \psi(\vec y) , \psi(\vec x)]
    }
    U(t)
    \\ &\phantom= - U^\dagger(t)
    \frac12 \int \dif^3 y \, \dif^3 z \, [\psi^\dagger(\vec y)
    \psi^\dagger(\vec z) V(\vec y, \vec z) \psi(\vec z) \psi(\vec y)
    , \psi(\vec x)]
    U(t)
    \intertext{%
        We start with the first commutator, the one with the kinetic energy.
        There we can use the $[AB,C]$ commutator relation. Because we can, we
        will do this derivation for both fermions and bosons and denote this
        with a $\pm$, where the upper sign is for fermions. The
        (anti-)commutator for $\phi(\vec y)$ and $\psi(\vec x)$ is always zero,
        so that is not interesting. And we directly drop that term before we
        write it down.
    }
    &= - U^\dagger(t)
    \int \dif^3 y \sbr{
        \frac{\hbar^2}{2m} [\vnabla \psi^\dagger(\vec y), \psi(\vec x)]_\pm \vnabla \psi(\vec
        y)
        + U(\vec y) [\psi^\dagger(\vec y) \psi(\vec y) , \psi(\vec x)]
    }
    U(t)
    \\ &\phantom=
    - U^\dagger(t)
    \frac12 \int \dif^3 y \, \dif^3 z \, [\psi^\dagger(\vec y)
    \psi^\dagger(\vec z) V(\vec y, \vec z) \psi(\vec z) \psi(\vec y)
    , \psi(\vec x)]
    U(t)
    \intertext{%
        Since the partial derivative is supposed to act on $\vec y$
        only---maybe we should have mentioned that earlier, sorry---we can pull
        it out of the commutator.
    }
    &= - U^\dagger(t)
    \int \dif^3 y \sbr{
        \frac{\hbar^2}{2m} \vnabla [\psi^\dagger(\vec y), \psi(\vec x)]_\pm \vnabla \psi(\vec
        y)
        + U(\vec y) [\psi^\dagger(\vec y) \psi(\vec y) , \psi(\vec x)]
    }
    U(t)
    \\ &\phantom=
    - U^\dagger(t)
    \frac12 \int \dif^3 y \, \dif^3 z \, [\psi^\dagger(\vec y)
    \psi^\dagger(\vec z) V(\vec y, \vec z) \psi(\vec z) \psi(\vec y)
    , \psi(\vec x)]
    U(t)
    \intertext{%
        Then this (anti-)commutator is just a negative $\delta$-distribution.
    }
    &= - U^\dagger(t)
    \int \dif^3 y \sbr{
        - \frac{\hbar^2}{2m} \vnabla \delta^{(3)}(\vec x - \vec y) \vnabla \psi(\vec
        y)
        + U(\vec y) [\psi^\dagger(\vec y) \psi(\vec y) , \psi(\vec x)]
    }
    U(t)
    \\ &\phantom=
    - U^\dagger(t)
    \frac12 \int \dif^3 y \, \dif^3 z \, [\psi^\dagger(\vec y)
    \psi^\dagger(\vec z) V(\vec y, \vec z) \psi(\vec z) \psi(\vec y)
    , \psi(\vec x)]
    U(t)
    \intertext{%
        We use partial integration to move the gradient from the
        $\delta$-distribution to the field operator. This will give us yet
        another minus sign that we have to take into account.
    }
    &= - U^\dagger(t)
    \int \dif^3 y \sbr{
        \frac{\hbar^2}{2m} \delta^{(3)}(\vec x - \vec y) \laplace \psi(\vec y)
        + U(\vec y) [\psi^\dagger(\vec y) \psi(\vec y) , \psi(\vec x)]
    }
    U(t)
    \\ &\phantom=
    - U^\dagger(t)
    \frac12 \int \dif^3 y \, \dif^3 z \, [\psi^\dagger(\vec y)
    \psi^\dagger(\vec z) V(\vec y, \vec z) \psi(\vec z) \psi(\vec y)
    , \psi(\vec x)]
    U(t)
    \intertext{%
        Great, now we can integrate over it to eliminate it.
    }
    &= - U^\dagger(t)
    \sbr{
        - \frac{\hbar^2}{2m} \laplace \psi(\vec x)
        + \int \dif^3 y \, U(\vec y) [\psi^\dagger(\vec y) \psi(\vec y) , \psi(\vec x)]
    }
    U(t)
    \\ &\phantom=
    - U^\dagger(t)
    \frac12 \int \dif^3 y \, \dif^3 z \, [\psi^\dagger(\vec y)
    \psi^\dagger(\vec z) V(\vec y, \vec z) \psi(\vec z) \psi(\vec y)
    , \psi(\vec x)]
    U(t)
    \intertext{%
        To finalize the kinetic energy, we sandwich the time evolution
        operators back onto it. That will make it a Heisenberg field operator
        again.
    }
    &=
    - \frac{\hbar^2}{2m} \laplace \psi^\text H(\vec x, t)
    - U^\dagger(t)
    \int \dif^3 y \, U(\vec y) [\psi^\dagger(\vec y) \psi(\vec y) , \psi(\vec x)]
    U(t)
    \\ &\phantom=
    - \frac12 U^\dagger(t)
    \int \dif^3 y \, \dif^3 z \, [\psi^\dagger(\vec y)
    \psi^\dagger(\vec z) V(\vec y, \vec z) \psi(\vec z) \psi(\vec y)
    , \psi(\vec x)]
    U(t)
    \intertext{%
        One down, two more to go. Next up is the scalar potential term. Here
        the same think with the $[AB,C]$ formula applies as well.
    }
    &=
    - \frac{\hbar^2}{2m} \laplace \psi^\text H(\vec x, t)
    - U^\dagger(t)
    \int \dif^3 y \, U(\vec y) [\psi^\dagger(\vec y), \psi(\vec x)]_\pm \psi(\vec y) 
    U(t)
    \\ &\phantom=
    - \frac12 U^\dagger(t)
    \int \dif^3 y \, \dif^3 z \, [\psi^\dagger(\vec y)
    \psi^\dagger(\vec z) V(\vec y, \vec z) \psi(\vec z) \psi(\vec y)
    , \psi(\vec x)]
    U(t)
    \intertext{%
        As before, the (anti-)commutator is a negative $\delta$-distribution. Then we
        can integrate over that as well and get rid of it. Now we can finally
        go back to writing this expression on a single line. It will probably
        not last very long, however.
    }
    &=
    - \frac{\hbar^2}{2m} \laplace \psi^\text H(\vec x, t)
    + U^\dagger(t) U(\vec x) \psi(\vec x) U(t)
    %\\ &\phantom=
    - \frac12 U^\dagger(t)
    \int \dif^3 y \, \dif^3 z \, [\psi^\dagger(\vec y)
    \psi^\dagger(\vec z) V(\vec y, \vec z) \psi(\vec z) \psi(\vec y)
    , \psi(\vec x)]
    U(t)
    \intertext{%
        Now we did not want to make too many steps without an explanation in
        between. So the second term can now be written a Heisenberg state.
    }
    &= - \frac{\hbar^2}{2m} \laplace \psi^\text H(\vec x, t)
    - U(\vec x) \psi(\vec x, t)
    %\\ &\phantom=
    - \frac12 U^\dagger(t)
    \int \dif^3 y \, \dif^3 z \, \sbr{\psi^\dagger(\vec y)
        \psi^\dagger(\vec z) V(\vec y, \vec z) \psi(\vec z) \psi(\vec y)
    , \psi(\vec x)}
    U(t)
    \intertext{%
        This leaves the last summand for us to work on. Of course, we use the
        same trusty (anti-)commutator relation that has served us so well in
        the course of this problem. But first of all, we move everything out of
        the commutator that commuted with $\psi(\vec x)$ anyway.
    }
    &= - \frac{\hbar^2}{2m} \laplace \psi^\text H(\vec x, t)
    - U(\vec x) \psi(\vec x, t)
    %\\ &\phantom=
    - \frac12 U^\dagger(t)
    \int \dif^3 y \, \dif^3 z \,
    \sbr{\psi^\dagger(\vec y) \psi^\dagger(\vec z), \psi(\vec x)}
    V(\vec y, \vec z) \psi(\vec z) \psi(\vec y)
    U(t)
    \intertext{%
        Without much further ado, we use the $[AB,C]$ relation. However, it is
        going to be in two lines again since it is so long because neither
        (anti-)commutator vanishes in this particular case here.
    }
    &= - \frac{\hbar^2}{2m} \laplace \psi^\text H(\vec x, t)
    - U(\vec x) \psi(\vec x, t)
    \\ &\phantom=
    - \frac12 U^\dagger(t)
    \int \dif^3 y \, \dif^3 z \,
    \sbr{
        \psi^\dagger(\vec y) \sbr{\psi^\dagger(\vec z), \psi(\vec x)}_\pm
        \mp
        \sbr{\psi^\dagger(\vec y), \psi(\vec x)}_\pm \psi^\dagger(\vec z)
    }
    V(\vec y, \vec z) \psi(\vec z) \psi(\vec y)
    U(t)
    \intertext{%
        Those (anti-)commutators are again
        $\delta$-distributions---surprise!---and can be integrated away. Since
        this derivation is not long enough yet we will include those steps. It
        would not be consistent either. So we expand the square bracket---the
        one which is not a commutator.
    }
    &= - \frac{\hbar^2}{2m} \laplace \psi^\text H(\vec x, t)
    - U(\vec x) \psi(\vec x, t)
    %\\ &\phantom=
    - \frac12 U^\dagger(t)
    \int \dif^3 y \, \dif^3 z \,
        \psi^\dagger(\vec y) \sbr{\psi^\dagger(\vec z), \psi(\vec x)}_\pm
    V(\vec y, \vec z) \psi(\vec z) \psi(\vec y) U(t)
    \\ &\phantom=
        \pm
    \frac12 U^\dagger(t)
    \int \dif^3 y \, \dif^3 z \,
        \sbr{\psi^\dagger(\vec y), \psi(\vec x)}_\pm \psi^\dagger(\vec z)
    V(\vec y, \vec z) \psi(\vec z) \psi(\vec y)
    U(t)
    \intertext{%
        Now we insert the $\delta$-distributions. Note that the signs flip
        again.
    }
    &= - \frac{\hbar^2}{2m} \laplace \psi^\text H(\vec x, t)
    - U(\vec x) \psi(\vec x, t)
    %\\ &\phantom=
    + \frac12 U^\dagger(t)
    \int \dif^3 y \, \dif^3 z \,
    \psi^\dagger(\vec y) \delta^{(3)}(\vec z - \vec x)
    V(\vec y, \vec z) \psi(\vec z) \psi(\vec y) U(t)
    \\ &\phantom=
        \mp
    \frac12 U^\dagger(t)
    \int \dif^3 y \, \dif^3 z \,
    \delta^{(3)}(\vec y - \vec x) \psi^\dagger(\vec z)
    V(\vec y, \vec z) \psi(\vec z) \psi(\vec y)
    U(t)
    \intertext{%
        With the $\delta$-distributions in place, we can finally carry out the
        integration and remove the number of variables in this expression.
    }
    &= - \frac{\hbar^2}{2m} \laplace \psi^\text H(\vec x, t)
    - U(\vec x) \psi(\vec x, t)
    %\\ &\phantom=
    + \frac12 U^\dagger(t) \int \dif^3 y \, \psi^\dagger(\vec y)
    V(\vec y, \vec x) \psi(\vec x) \psi(\vec y) U(t)
    \\ &\phantom=
    \mp
    \frac12 U^\dagger(t) \int \dif^3 z \, \psi^\dagger(\vec z)
    V(\vec x, \vec z) \psi(\vec z) \psi(\vec x) U(t)
    \intertext{%
        In the last integration, we rename the integration variable from $z$ to
        $y$ to match the other integral.
    }
    &= - \frac{\hbar^2}{2m} \laplace \psi^\text H(\vec x, t)
    - U(\vec x) \psi(\vec x, t)
    %\\ &\phantom=
    + \frac12 U^\dagger(t) \int \dif^3 y \, \psi^\dagger(\vec y)
    V(\vec y, \vec x) \psi(\vec x) \psi(\vec y) U(t)
    \\ &\phantom=
    \mp
    \frac12 U^\dagger(t) \int \dif^3 y \, \psi^\dagger(\vec y)
    V(\vec x, \vec y) \psi(\vec y) \psi(\vec x) U(t)
    \intertext{%
        Since $V$ is symmetric in its arguments, we can switch those in the
        first term.
    }
    &= - \frac{\hbar^2}{2m} \laplace \psi^\text H(\vec x, t)
    - U(\vec x) \psi(\vec x, t)
    %\\ &\phantom=
    + \frac12 U^\dagger(t) \int \dif^3 y \, \psi^\dagger(\vec y)
    V(\vec x, \vec y) \psi(\vec x) \psi(\vec y) U(t)
    \\ &\phantom=
    \mp
    \frac12 U^\dagger(t) \int \dif^3 y \, \psi^\dagger(\vec y)
    V(\vec x, \vec y) \psi(\vec y) \psi(\vec x) U(t)
    \intertext{%
        Now the use that the (anti-)commutator for the field annihilation
        operators vanishes for the two kinds of particles, respectively, and
        switch the very last two field operators in the last term. That will
        also let us get rid of the plus-minus-sign.
    }
    &= - \frac{\hbar^2}{2m} \laplace \psi^\text H(\vec x, t)
    - U(\vec x) \psi(\vec x, t)
    %\\ &\phantom=
    + \frac12 U^\dagger(t) \int \dif^3 y \, \psi^\dagger(\vec y)
    V(\vec x, \vec y) \psi(\vec x) \psi(\vec y) U(t)
    \\ &\phantom=
    +
    \frac12 U^\dagger(t) \int \dif^3 y \, \psi^\dagger(\vec y)
    V(\vec x, \vec y) \psi(\vec x) \psi(\vec y) U(t)
    \intertext{%
        Now we can combine those identical terms.
    }
    &= - \frac{\hbar^2}{2m} \laplace \psi^\text H(\vec x, t)
    - U(\vec x) \psi(\vec x, t)
    + U^\dagger(t) \int \dif^3 y \, \psi^\dagger(\vec y)
    V(\vec x, \vec y) \psi(\vec x) \psi(\vec y) U(t)
    \intertext{%
        As a last step, we can insert ones in the manifestation of $U(t)
        U^\dagger(t)$ between all the field operators.
    }
    &= - \frac{\hbar^2}{2m} \laplace \psi^\text H(\vec x, t)
    - U(\vec x) \psi(\vec x, t)
    + U^\dagger(t) \int \dif^3 y \, \psi^\dagger(\vec y) U(t) U^\dagger(t)
    V(\vec x, \vec y) \psi(\vec x) U(t) U^\dagger(t)\psi(\vec y) U(t)
    \intertext{%
        Then we can write all the field operators as Heisenberg operators.
    }
    &= - \frac{\hbar^2}{2m} \laplace \psi^\text H(\vec x, t)
    - U(\vec x) \psi(\vec x, t)
    + \int \dif^3 y \, \psi^{\text H \dagger}(\vec y, t)
    V(\vec x, \vec y) \psi^\text H(\vec x, t) \psi^\text H(\vec y, t)
\end{align*}
\end{landscape}

\begin{question}
    How do the four points that can be achieved in the problem related to the
    two points that can be achieved in problem 1.1? Problem 1.1 was utterly
    trivial, this is not trivial. Except when one would just say that it is the
    same relation as for the bosons since all the commutators just get replaced
    with anti-commutators in the relation used in 1.2 and 1.3 and is done with
    it. Was \emph{that} asked here?
\end{question}

\end{document}

% vim: spell spelllang=en tw=79
