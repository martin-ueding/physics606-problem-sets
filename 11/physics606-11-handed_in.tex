\documentclass[11pt, english, fleqn, DIV=15, headinclude, BCOR=1.5cm]{scrartcl}

\usepackage[
    bibatend,
    color,
]{../header}

\usepackage{tikz}
\usepackage{pdflscape}

\usepackage[tikz]{mdframed}
\newmdtheoremenv[%
    backgroundcolor=black!5,
    innertopmargin=\topskip,
    splittopskip=\topskip,
]{theorem}{Theorem}[section]

\hypersetup{
    pdftitle=
}

\newcounter{totalpoints}
\newcommand\punkte[1]{#1\addtocounter{totalpoints}{#1}}

\newcounter{problemset}
\setcounter{problemset}{11}

\subject{physics606 -- Advanced Quantum Theory}
\ihead{physics606 -- Problem Set \arabic{problemset}}

\title{Problem Set \arabic{problemset}}

\publishers{Group 2 -- Dilege Gülmez}
\ofoot{Group 2 -- Dilege Gülmez}

\newmdenv[%
    backgroundcolor=black!0,
    frametitlebackgroundcolor=black!0,
    roundcorner=5pt,
    skipabove=\topskip,
    innertopmargin=\topskip,
    splittopskip=\topskip,
    frametitle={Problem statement},
    frametitlerule=true,
    nobreak=true,
]{problem}

\newmdenv[%
    backgroundcolor=white,
    frametitlebackgroundcolor=black!0,
    roundcorner=5pt,
    skipabove=\topskip,
    innertopmargin=\topskip,
    innerbottommargin=8cm,
    splittopskip=\topskip,
    frametitle={Side question},
    frametitlerule=true,
]{question}

\newcommand\an{^\text{angular}}
\newcommand\ra{^\text{radial}}


\author{
    Martin Ueding \\ \small{\href{mailto:mu@martin-ueding.de}{mu@martin-ueding.de}}
    \and
    Lino Lemmer \\ \small{\href{mailto:l2@uni-bonn.de}{l2@uni-bonn.de}}
}
\ifoot{Martin Ueding, Lino Lemmer}

\ohead{\rightmark}

\begin{document}

\maketitle

\vspace{3ex}

\begin{center}
    \begin{tabular}{rrr}
        problem number & achieved points & possible points \\
        \midrule
        1 & & \punkte{19} \\
        2 & & \punkte{12} \\
        3 & & \punkte{5} \\
        \midrule
        total & & \arabic{totalpoints}
    \end{tabular}
\end{center}

\section{Two-particle scattering}

The notation is a bit strange. The center of mass system has an asterisk in the
superscript, the lab frame has an italic (although it should be roman according
to ISO standard) “L” in it. We will drop this “L” and just have asterisk and
non-asterisk variables.

\subsection{Galilean transformation}

In the center of mass frame, there is no total momentum:
\[
    \vec p_1^* + \vec p_2^* = \vec 0.
\]
The given velocities $v$ have no specification whether they are before of after
the scattering. To avoid any confusion, we define the ones before to be $v$,
the after to be $w$. The transformation is a simple boost which is $\vec v_i^*
= \vec v_i - \vec B$. We use the letter $B$ here to avoid mixing it up with any
of the velocities. We plug this in and solve for $\vec B$. The masses of the
particles are only introduced in part four on the problem set, but they are
$m_1$ and $m_2$.
\begin{align*}
    \vec p_1^* + \vec p_2^* &= \vec 0 \\
    \intertext{%
        Write out the momentum.
    }
    \iff m_1 \vec v_1^* + m_2 \vec v_2^* &= \vec 0 \\
    \intertext{%
        Insert the transformation. $\vec v_2 = \vec 0$ because the in the
        laboratory frame, the second particle is initially at rest.
    }
    \iff m_1 [\vec v_1 - \vec B] + m_2 [- \vec B] &= \vec 0 \\
    \iff [m_1 + m_2] \vec B &= m_1 \vec v_1 \\
    \iff \vec B &= \frac{m_1}{m_1 + m_2} \vec v_1
\end{align*}

\subsection{Angular independence in center of mass frame}

In the center of mass frame, there is zero total momentum. For two particles,
like here, this means that they have opposing momentum. The energy is given by
\[
    \sum_i \frac 12 m_i p_i^{*2}.
\]
The total energy is conserved,
\[
    \therefore \forall i, j\colon p_i^* = k_j^*.
\]
That also means that there is no angular dependence, so there is no restriction
in $\theta^*$.

\subsection{Angular dependence in lab frame}

We know that
\[
    \vec p_1 = \vec k_1 + \vec k_2
\]
holds in the lab frame since the second particle was at rest initially. Then we
can write down energy conservation:
\begin{align*}
    m_1 p_1^2 &= m_1 k_1^2 + m_2 k_2^2 \\
    \intertext{%
        Replace $\vec k_2$ with other knowns.
    }
    \iff m_1 p_1^2 &= m_1 k_1^2 + m_2 [\vec p_1 - \vec k_1]^2 \\
    \intertext{%
        Expand.
    }
    \iff m_1 p_1^2 &= m_1 k_1^2 + m_2 p_1^2 + m_2 k_1^2 - 2 m_2 \vec p_1 \vec k_1 \\
    \intertext{%
        Exchange on the sides.
    }
    \iff 2 m_2 \vec p_1 \vec k_1 &= m_1 k_1^2 + m_2 p_1^2 + m_2 k_1^2 - m_1 p_1^2  \\
    \frac{\vec p_1 \vec k_1}{p_1 k_1} &= \frac 12 \sbr{
        \frac{m_1}{m_2} \frac{k_1}{p_1} + \frac{p_1}{k_1} + \frac{k_1}{p_1} -
        \frac{m_1}{m_2} \frac{p_1}{k_1}
    } \\
    &= \cos(\theta)
\end{align*}
So one can see that $k_1$ depends on $\theta$. We could try to solve this
equation for $k_1$, but that was not asked.

\subsection{Equal mass}

The energy has to transfer completely to particle 2. That also means that the
momentum has to transfer completely. This gives us three equations,
\[
    \vec p_1 = \vec k_2
    \eqnsep
    \vec p_2 = \vec k_1 = \vec 0
    \eqnsep
    m_1 p_1^2 + 0 = 0 + m_2 k_2^2.
\]
From that, $m_1 = m_2$ follows pretty directly.

\subsection{Relation of the angles}

\newcommand\w{{v_1^*}}
\newcommand\ct{\cos(\theta^*)}

\begin{align*}
    \cos(\theta)
    &= \frac{\vec p_1 \vec k_1}{p_1 k_1} \\
    \intertext{%
        We can cancel $m_1$ twice and obtain the velocities.
    }
    &= \frac{\vec v_1 \vec w_1}{v_1 w_1} \\
    \intertext{%
        Those velocities can be boosted into the center of mass system.
    }
    &= \frac{[\vec v_1^* + \vec B] [\vec w_1^* + \vec B]}{|\vec v_1^* + \vec B| |\vec w_1^* + \vec B|} \\
    \intertext{%
        We expand.
    }
    &= \frac{\vec v_1^* \vec w_1^* + \vec v_1^* \vec B + \vec v_1^* \vec B + B^2}
    {|\vec v_1^* + \vec B| |\vec w_1^* + \vec B|} \\
    \intertext{%
        The magnitude of the velocities of particle 1 is the same, since all
        momenta (particle 1, 2 and before/after) have the same magnitude.
        Divide by $m_1$ and obtain conserved velocities. The angle between
        $\vec v_1^*$ and $\vec w_1^*$ is $\theta^*$. The boost $\vec B$
        parallel to $\vec v_1^*$. That will let us express the scalar products
        in terms of that angle and the magnitudes of a single velocity.
    }
    &= \frac{\w^2 \ct + \w B + \w B \ct + B^2}{[\w + B] \sqrt{\w^2 + 2 \w B \ct
    + B^2}}
    \intertext{%
        Now we use the proportionality between $B$ and $\w$, where we call the
        proportionality constant $\alpha$.
    }
    &= \frac{\w^2 \ct + \alpha \w^2 + \alpha \w^2 \ct + \alpha^2 \w^2}{\w[1 +
    \alpha] \sqrt{\w^2 + 2 \alpha \w^2 \ct + \alpha^2 \w^2}}
    \intertext{%
        We cancel off a $\w^2$.
    }
    &= \frac{\ct + \alpha + \alpha \ct + \alpha^2}{[1 +
    \alpha] \sqrt{1 + 2 \alpha \ct + \alpha^2}}
    \intertext{%
        We factor the numerator.
    }
    &= \frac{[1 + \alpha] \sbr{\ct + \alpha}}{[1 +
    \alpha] \sqrt{1 + 2 \alpha \ct + \alpha^2}}
    \intertext{%
        Now we can cancel the first bracket.
    }
    &= \frac{\alpha + \ct}{\sqrt{1 + \alpha^2 + 2 \alpha \ct}}
\end{align*}
Now if 
\[
    \alpha = \frac{m_1}{m_2},
\]
then this would would work out by expanding the whole fraction with $m_2$. This
does not match what we had in the very first part of the problem though. So
this is not really solved yet.

% FIXME This could be improved.

In the limit $m_1 \ll m_2$, it simplifies to
\[
    \cos(\theta) = 1
    \quad\Longleftarrow\quad
    \theta = 0.
\]

In the limit $m_1 \gg m_2$, this simplifies to $\cos(\theta) = \cos(\theta^*)$
which is fulfilled given $\theta = \theta^*$.

When the masses are equal, we have
\begin{align*}
    \cos(\theta)
    &= \frac{m_1 + m_1 \cos(\theta^*)}{\sqrt{2 m_1^2 + 2m_1^2 \cos(\theta^*)}} \\
    &= \frac{1 + \cos(\theta^*)}{\sqrt{2 + 2\cos(\theta^*)}} \\
    &= \sqrt{\frac{1 + 1\cos(\theta^*)}2} \\
    \intertext{%
        Now we use the power reduction formula and get
    }
    &= \cos\del{\frac{\theta^*}{2}}.
\end{align*}
This is consistent with
\[
    \theta = \frac{\theta^*}2.
\]


\subsection{Scattering cross section}

% TODO

\section{Two-particle wave function}

\subsection{Normalization}

We have to normalize the given function. We think that this $N$ is meant to be
the normalization, so we put this to the other side already.
\begin{align*}
    \frac 1{N^2}
    &= \frac{1}{N^2} \intop_\R \dif x_1 \, \intop_\R \dif x_2 \, |\psi(x_1, x_2)|^2 \\
    \intertext{%
        So we put in the function.
    }
    &= \intop_\R \dif x_1 \, \intop_\R \dif x_2 \,
    \exp\del{- 2 \frac{[x_1 - x_2]^2}{\sigma^2}}
    \exp\del{- 2 \frac{[x_1 + x_2]^2}{\Sigma^2}}
    \intertext{%
        Now the best way is to do a coordinate transformation to parabolic
        coordinates (I think they are called that). This is just
        \[
            \eta := x_1 - x_2
            \eqnsep
            \xi := x_1 + x_2.
        \]
        The Jacobian of this transformation can be computed to be just 2. That
        transformation has the nice effect to decouple the two integrals.
    }
    &= 2 \intop_\R \dif \xi \exp\del{- 2 \frac{\xi^2}{\sigma^2}}
    \intop_\R \dif \eta \exp\del{- 2 \frac{\eta^2}{\sigma^2}}
    \intertext{%
        Those integrals can be solved using the solution formulas for the
        Gaussian integrals that we have missed somewhat in the last few
        homework problems. Having $a = 2 / \sigma^2$, we get
    }
    &= 2 \sqrt{\frac{\piup}{2/\sigma^2}} \sqrt{\frac{\piup}{2/\Sigma^2}} \\
    &= \piup \sigma \Sigma
\end{align*}

\subsection{Simple product}

You can apply the binomial equation in the exponent and see that you will
always have terms like $\exp(x_1 x_2)$ which cannot be factored into functions
of one of the variables only.


\begin{landscape}
\subsection{Decomposition}

Here we will calculate the Fourier transform of the given $\psi$. It then
should fulfill the equation~(4) from the problem set.
\begin{align*}
    \tilde \psi
    &= \mathcal F_{x_1} \mathcal F_{x_2} \psi(x_1, x_2) \\
    \intertext{%
        We insert the integrals for the Fourier transform.
    }
    &= \frac{1}{2\piup} \intop_\R \dif x_1 \intop_\R \dif x_2 \, \psi(x_1, x_2)
    \exp(\iup k_1 x_1) \exp(\iup k_2 x_2) \\
    \intertext{%
        We insert $\psi$.
    }
    &= \frac{N}{2\piup} \intop_\R \dif x_1 \intop_\R \dif x_2 \, 
    \exp\del{- \frac{[x_1 - x_2]^2}{\sigma^2}}
    \exp\del{- \frac{[x_1 + x_2]^2}{\Sigma^2}}
    \exp(\iup k_1 x_1) \exp(\iup k_2 x_2) \\
    \intertext{%
        The next thing to do is to convert this into a form where the beloved
        formulas can be used. First, we combine all the exponentials.
    }
    &= \frac{N}{2\piup} \intop_\R \dif x_1 \intop_\R \dif x_2 \, 
    \exp\del{- \frac{[x_1 - x_2]^2}{\sigma^2} - \frac{[x_1 + x_2]^2}{\Sigma^2} + \iup k_1 x_1 + \iup k_2 x_2} \\
    \intertext{%
        Then we expand the squares.
    }
    &= \frac{N}{2\piup} \intop_\R \dif x_1 \intop_\R \dif x_2 \, 
    \exp\del{- \frac{x_1^2 - 2 x_1 x_2 + x_2^2}{\sigma^2} - \frac{x_1^2 + 2 x_1
    x_2 + x_2^2}{\Sigma^2} + \iup k_1 x_1 + \iup k_2 x_2} \\
    \intertext{%
        Then grouping by the powers of $x_1$ and $x_2$ is in order. We define 
        \[
            Z := \frac{1}{\sigma^2} + \frac{1}{\Sigma^2}
        \]
        and use that.
    }
    &= \frac{N}{2\piup} \intop_\R \dif x_1 \intop_\R \dif x_2 \, 
    \exp\del{- Z x_1^2 + \sbr{\frac{2 x_2}{\sigma^2} - \frac{2 x_2}{\Sigma^2} +
    \iup k_1} x_1 - Z x_2^2 + \iup k_2 x_2} \\
    \intertext{%
        Using
        \[
            a = Z
            \eqnsep
            b = \frac{2 x_2}{\sigma^2} - \frac{2 x_2}{\Sigma^2} + \iup k_1
        \]
        for the $x_1$ integral, we obtain:
    }
    &= \frac{N}{2\piup} \sqrt{\frac{\piup}{Z}} \intop_\R \dif x_2 \exp\del{- Z x_2^2 + \iup k_2 x_2}
    \exp\del{\frac{1}{4 Z}\sbr{\frac{2 x_2}{\sigma^2} - \frac{2 x_2}{\Sigma^2} +
    \iup k_1}^2} \\
    \intertext{%
        Now all the $x_1$ are integrated over and gone. The same steps have to
        be done for the $x_2$ integral. So the exponential functions have to be
        combined and sorted by powers of $x_2$.
    }
    &= \frac{N}{2\piup} \sqrt{\frac{\piup}{Z}} \intop_\R \dif x_2 \exp\del{- Z x_2^2 + \iup k_2 x_2
    + \frac{1}{4 Z}\sbr{\frac{2 x_2}{\sigma^2} - \frac{2 x_2}{\Sigma^2} +
    \iup k_1}^2} \\
    \intertext{%
        Here we define
        \[
            Y := \frac{1}{\sigma^2} - \frac{1}{\Sigma^2}
        \]
        similarly to $Z$, except for the sign.
    }
    &= \frac{N}{2\piup} \sqrt{\frac{\piup}{Z}} \intop_\R \dif x_2 \exp\del{- Z x_2^2 + \iup k_2 x_2
    + \frac{1}{4 Z}\sbr{2 Y x_2 + \iup k_1}^2} \\
    &= \frac{N}{2\piup} \sqrt{\frac{\piup}{Z}} \intop_\R \dif x_2 \exp\del{- Z x_2^2 + \iup k_2 x_2
    + \frac{1}{4 Z}\sbr{4 Y^2 x_2^2 + 4 \iup Y k_2 x_2 - k_1^2}} \\
    &= \frac{N}{2\piup} \sqrt{\frac{\piup}{Z}} \intop_\R \dif x_2 \exp\del{
    - \sbr{Z+\frac{Y^2}Z} x_2^2
    + \sbr{\iup k_2 + \iup \frac YZ k_2} x_2
    - \frac{k_1^2}{4 Z}} \\
    \intertext{%
        Now we can apply the Gaussian integration again.
    }
    &= \frac{N}{2\piup} \sqrt{\frac{\piup}{Z}} \sqrt{\frac{\piup}{Z +
    \frac{Y^2}Z}}
    \exp\del{- \frac{k_1^2}{4 Z}}
    \exp\del{
        \frac{1}{4 \sbr{Z+\frac{Y^2}Z}} \sbr{\iup k_2 + \iup \frac YZ k_2}^2
    } \\
    \intertext{%
        We can simplify this again.
    }
    &= \frac{N}{2[Z^2 + Y^2]}
    \exp\del{- \frac{k_1^2}{4 Z}
    -
    \frac{k_2^2}{4 \sbr{Z+\frac{Y^2}Z}} \sbr{1 + 2 \frac YZ + \frac{Y^2}{Z^2}}
    } \\
    \intertext{%
        We move the $k$s to the end of each summand.
    }
    &= \frac{N}{2[Z^2 + Y^2]} \exp\del{- \frac{1}{4 Z} k_1^2
    - \frac{1 + 2 \frac YZ + \frac{Y^2}{Z^2}}{4 \sbr{Z+\frac{Y^2}Z}} k_2^2} \\
    \intertext{%
        We expand with a factor of $Z$ in the second summand.
    }
    &= \frac{N}{2[Z^2 + Y^2]} \exp\del{- \frac{1}{4 Z} k_1^2
    - \frac{Z + 2 Y + \frac{Y^2}{Z}}{4 \sbr{Z^2+Y^2}} k_2^2}
\end{align*}
We would expect this to be more symmetric in $k_1$ and $k_2$ since the original
function was pretty symmetric in $x_1$ and $x_2$ as well.
\end{landscape}

% TODO

\subsection{Use as boson or fermion function}

In general, you just have to (anti)symmetrize the hell out the function and you
get a completely (anti)symmetric function. In this particular case, the
function already is symmetric in its coordinates. So it the antisymmetric part
is zero. Therefore one can only use it a bosonic function.

\section{Totally symmetric $N$-particle state}

The modulus squared of this thing can be written like this, where the dots mean
every possible combination of all permutations with all permutations:
\[
    \abs{S_+ \ket{i_1, i_2, \ldots, i_N}}^2
    = \frac{1}{N} \sbr{
        \braket{\ldots | \ldots}
        +
        \braket{\ldots | \ldots}
        +
        \ldots
    }
\]
Assume first, that all $i_j$ are different. Then there are $[N!]^2$ scalar
products, and only $N!$ are nonzero, because the permutations are exactly the
same for both terms. Therefore, the norm will be 1. Now assume that just one of
the $i$s has a multiplicity of $n_i$. Then terms that were orthogonal before,
are now identical, giving a scalar product of 1. The number of possible
permutations of this particular $i$ is given by $n_i$. Therefore, the number of
nonzero scalar products will go up by a factor $n_i$. This can be done for the
other $i$s as well, giving the normalization of
\[
    \prod_{i = 0}^N n_i.
\]


\end{document}

% vim: spell spelllang=en tw=79
