\documentclass[11pt, english, fleqn, DIV=15, headinclude, BCOR=1.5cm]{scrartcl}

\usepackage[
    bibatend,
    color,
]{../header}

\usepackage{tikz}
\usepackage{pdflscape}

\usepackage[tikz]{mdframed}
\newmdtheoremenv[%
    backgroundcolor=black!5,
    innertopmargin=\topskip,
    splittopskip=\topskip,
]{theorem}{Theorem}[section]

\hypersetup{
    pdftitle=
}

\newcounter{totalpoints}
\newcommand\punkte[1]{#1\addtocounter{totalpoints}{#1}}

\newcounter{problemset}
\setcounter{problemset}{11}

\subject{physics606 -- Advanced Quantum Theory}
\ihead{physics606 -- Problem Set \arabic{problemset}}

\title{Problem Set \arabic{problemset}}

\publishers{Group 2 -- Dilege Gülmez}
\ofoot{Group 2 -- Dilege Gülmez}

\newmdenv[%
    backgroundcolor=black!0,
    frametitlebackgroundcolor=black!0,
    roundcorner=5pt,
    skipabove=\topskip,
    innertopmargin=\topskip,
    splittopskip=\topskip,
    frametitle={Problem statement},
    frametitlerule=true,
    nobreak=true,
]{problem}

\newmdenv[%
    backgroundcolor=white,
    frametitlebackgroundcolor=black!0,
    roundcorner=5pt,
    skipabove=\topskip,
    innertopmargin=\topskip,
    innerbottommargin=8cm,
    splittopskip=\topskip,
    frametitle={Side question},
    frametitlerule=true,
]{question}

\newcommand\an{^\text{angular}}
\newcommand\ra{^\text{radial}}


\author{
    Martin Ueding \\ \small{\href{mailto:mu@martin-ueding.de}{mu@martin-ueding.de}}
    \and
    Lino Lemmer \\ \small{\href{mailto:l2@uni-bonn.de}{l2@uni-bonn.de}}
}
\ifoot{Martin Ueding, Lino Lemmer}

\ohead{\rightmark}

\begin{document}

\maketitle

\vspace{3ex}

\begin{center}
    \begin{tabular}{rrr}
        problem number & achieved points & possible points \\
        \midrule
        1 & & \punkte{19} \\
        2 & & \punkte{12} \\
        3 & & \punkte{5} \\
        \midrule
        total & & \arabic{totalpoints}
    \end{tabular}
\end{center}

\section{Two-particle scattering}

\section{Two-particle wave function}

\subsection{Normalization}

We have to normalize the given function. We think that this $N$ is meant to be
the normalization, so we put this to the other side already.
\begin{align*}
    \frac 1{N^2}
    &= \frac{1}{N^2} \intop_\R \dif x_1 \, \intop_\R \dif x_2 \, |\psi(x_1, x_2)|^2 \\
    \intertext{%
        So we put in the function.
    }
    &= \intop_\R \dif x_1 \, \intop_\R \dif x_2 \,
    \exp\del{- 2 \frac{[x_1 - x_2]^2}{\sigma^2}}
    \exp\del{- 2 \frac{[x_1 + x_2]^2}{\Sigma^2}}
    \intertext{%
        Now the best way is to do a coordinate transformation to parabolic
        coordinates (I think they are called that). This is just
        \[
            \eta := x_1 - x_2
            \eqnsep
            \xi := x_1 + x_2.
        \]
        The Jacobian of this transformation can be computed to be just 2. That
        transformation has the nice effect to decouple the two integrals.
    }
    &= 2 \intop_\R \dif \xi \exp\del{- 2 \frac{\xi^2}{\sigma^2}}
    \intop_\R \dif \eta \exp\del{- 2 \frac{\eta^2}{\sigma^2}}
    \intertext{%
        Those integrals can be solved using the solution formulas for the
        Gaussian integrals that we have missed somewhat in the last few
        homework problems. Having $a = 2 / \sigma^2$, we get
    }
    &= 2 \sqrt{\frac{\piup}{2/\sigma^2}} \sqrt{\frac{\piup}{2/\Sigma^2}} \\
    &= \piup \sigma \Sigma
\end{align*}

\subsection{Simple product}

You can apply the binomial equation in the exponent and see that you will
always have terms like $\exp(x_1 x_2)$ which cannot be factored into functions
of one of the variables only.

\subsection{Decomposition}

% TODO

\subsection{Use as boson or fermion function}

In general, you just have to (anti)symmetrize the hell out the function and you
get a completely (anti)symmetric function. In this particular case, the
function already is symmetric in its coordinates. So it the antisymmetric part
is zero. Therefore one can only use it a bosonic function.

\section{Totally symmetric $N$-particle state}

The modulus squared of this thing can be written like this, where the dots mean
every possible combination of all permutations with all permutations:
\[
    \abs{S_+ \ket{i_1, i_2, \ldots, i_N}}^2
    = \frac{1}{N} \sbr{
        \braket{\ldots | \ldots}
        +
        \braket{\ldots | \ldots}
        +
        \ldots
    }
\]
Assume first, that all $i_j$ are different. Then there are $[N!]^2$ scalar
products, and only $N!$ are nonzero, because the permutations are exactly the
same for both terms. Therefore, the norm will be 1. Now assume that just one of
the $i$s has a multiplicity of $n_i$. Then terms that were orthogonal before,
are now identical, giving a scalar product of 1. The number of possible
permutations of this particular $i$ is given by $n_i$. Therefore, the number of
nonzero scalar products will go up by a factor $n_i$. This can be done for the
other $i$s as well, giving the normalization of
\[
    \prod_{i = 0}^N n_i.
\]


\end{document}

% vim: spell spelllang=en tw=79
